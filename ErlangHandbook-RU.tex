% !TEX encoding = UTF-8 Unicode
%% NOTE: Due to using fontspec and UTF8 to avoid encoding problems and stuff, Russian
%% version of this document builds using xe(la)tex and TexStudio
\documentclass[russian,14pt,oneside]{book}
\usepackage{extsizes} % additional font sizes for documentclass

%%%%%%
%% package includes
\usepackage{ifxetex}
\ifxetex
    \usepackage{fontspec}
    \defaultfontfeatures{Scale=MatchLowercase,Mapping=tex-text}
    \setmainfont[Path=fonts/,ItalicFont=PTF56F.ttf,
        BoldFont=PTF75F.ttf]{PTF55F.ttf}
    \setmonofont[Ligatures={NoRequired,NoCommon,NoContextual},
        Path=fonts/,BoldFont=FiraMono-Bold.ttf]{FiraMono-Regular.ttf}
\else
	\usepackage[utf8]{inputenc}
	\usepackage[T2A]{fontenc}
\fi

\usepackage[russian]{babel}
\usepackage{hyperref}
\usepackage{textcomp}
\usepackage{array}
\usepackage{color}
\usepackage[hmargin=2.5cm,vmargin=2.5cm]{geometry}
\usepackage{minted}
\usepackage{todonotes}
\usepackage{parskip}
\usepackage[Bjornstrup]{fncychap}
\usepackage{listings}

%%%%%%
%% package setup
\hypersetup{
    colorlinks=true,
    linktoc=all,
    linkcolor=red,
}

%%%%%%
%% hacks and macros
\newminted[erlang]{erlang}{xleftmargin=0cm,xrightmargin=0cm,frame=single,framerule=1pt}

%% Framed code fragment coloring, for code examples where Russian variable 
%% names break syntax and coloring
\usepackage{fancyvrb}
\DefineVerbatimEnvironment{erlangru}{Verbatim}{
	fontsize=\small,frame=lines,samepage=true}

%% hack to fix the margin issue in \item s in 1.2.1
\newminted[erlangim]{erlang}{xleftmargin=-1.5cm}

%%% display todo notes or not?
\newcommand{\todonote}[1]{\todo[inline]{\textbf{Note:} #1}}
%\newcommand{\todonote}[1]{}

%% typesetting for the "results in" notation used.
\def\resultingin{$\quad\Rightarrow\quad$}

\begin{document}

%%%%%%
%% Title
\begin{titlepage}
\centering

\vspace*{70pt}
\includegraphics[scale=0.3]{includes/erlang-logo.png}\\[0.8\baselineskip]
{\Huge \sffamily НАСТОЛЬНАЯ КНИГА}\\
\vspace{250pt}
{\LARGE Бьярн Дэкер (Bjarne D\"acker)}\par
{\LARGE Роберт Вирдинг (Robert Virding)}\par

\end{titlepage}


%%%%%%
%% Info Page
\clearpage
\thispagestyle{empty}
{\Huge Настольная книга по Erlang}\\[0.1\baselineskip]
\hspace{10pt}{\large{Бьярн Дэкер (Bjarne Däcker) и Роберт Вирдинг (Robert Virding)}}

\vspace{10pt}
{\Large \textbf{Версия:}\\[0.2\baselineskip]
\immediate\write18{git log -n1 | grep 'Date:' | sed 's/Date:   //g' > _revision.tex}
\input{_revision}
\immediate\write18{rm _revision.tex}
}

\vspace{10pt}
{\large Последняя версия этой книги находится по адресу:\\
\url{http://opensource.erlang-solutions.com/erlang-handbook}}

\vfill

\textbf{Главный редактор}\\[0.1\baselineskip]
%\begin{tabular}{@{\hspace{3ex}}p{42em}}
Омер Килич (Omer Kilic)
%\end{tabular}

\textbf{Прочие участники}\\[0.1\baselineskip]
%\begin{tabular}{@{\hspace{3ex}}p{42em}}
Список людей, которые помогли с изменениями и исправлениями находится
\href{https://github.com/esl/erlang-handbook/graphs/contributors}{в репозитории 
проекта}.
%\end{tabular}

\vspace{10pt}
\textbf{Соглашения}\\
Спецификации синтаксиса задаются с помощью \texttt{этого моноширинного шрифта}.
В квадратные скобки ([ ]) заключаются необязательные части. Термы, начинающиеся
с заглавной буквы, как например \textit{Integer} должны быть заменены на 
какое-нибудь подходящее значение. Термы, начинающиеся со строчной буквы, как
например \texttt{end}, являются зарезервированными словами языка Erlang. 
Вертикальная черта (\textbar{}) разделяет альтернативные варианты, как например 
Integer \textbar{} Float.

\vspace{10pt}
\textbf{Ошибки и улучшения}\\
Это живой документ, пожалуйста присылайте исправления и рекомендации по улучшению
содержимого используя систему учёта задач Github по адресу
\url{https://github.com/esl/erlang-handbook}. Вы также можете создать свою ветвь
репозитория (fork) и прислать нам запрос на соединение ветвей (pull request) с
вашими предлагаемыми исправлениями и предложениями. Новые ревизии этого документа
будут публиковаться по мере накопления больших изменений.

\vspace{10pt}
\includegraphics[scale=0.7]{includes/cc-by-sa.png}\\ Этот текст доступен согласно
условиям лицензии Creative Commons Attribution-ShareAlike 3.0 License. Вы имеете
право копировать, распространять и передавать эту книгу по условиям лицензии, 
описанным по адресу \url{http://creativecommons.org/licenses/by-sa/3.0}

\newpage


%%%%%%
%% TOC
\tableofcontents


%%%%%%
%% Chapters

\chapter{Background, or Why Erlang is that it is}
\label{background}

Erlang is the result of a project at Ericsson's Computer Science Laboratory to
improve the programming of telecommunication applications.  A critical
requirement was supporting the characteristics of such applications, that include:

\begin{itemize}
\item Massive concurrency

\item Fault-tolerance

\item Isolation

\item Dynamic code upgrading at runtime

\item Transactions
\end{itemize}

Throughout the whole of Erlang's history the development process has
been extremely pragmatic. The characteristics and properties of the
types of systems in which Ericsson were interested drove Erlang's
development.  These properties were considered to be so
fundamental that it was decided to build support for them into the
language itself, rather than in libraries.  Because of the pragmatic development
process, rather than a result of prior planning, Erlang ``became'' a functional language --- since
the features of functional languages fitted well with the properties of the systems being developed.


\chapter{Structure of an Erlang program}

\section{Module syntax}

An Erlang program is made up of \textbf{modules} where each module is
a text file with the extension \textbf{.erl}. For small programs, all
modules typically reside in one directory.
A module consists of module attributes and function definitions.

\begin{erlang}
-module(demo).
-export([double/1]).

double(X) -> times(X, 2).

times(X, N) -> X * N.
\end{erlang}

The above module \texttt{demo} consists of the function \texttt{times/2}
which is local to the module and the function \texttt{double/1} which
is exported and can be called from outside the module.

\texttt{demo:double(10)} \resultingin \texttt{20}\hfill
(the arrow $\Rightarrow$ should be read as ``resulting in'')

\texttt{double/1} means the function ``double'' with \textit{one}
argument. A function \texttt{double/2} taking \textit{two} arguments
is regarded as a different function. The number of arguments is called
the \textbf{arity} of the function.


\section{Module attributes}
A \textbf{module attribute} defines a certain property of a module and
consists of a \textbf{tag} and a \textbf{value}:

\texttt{-Tag(Value).}

\texttt{Tag} must be an atom, while \texttt{Value} must be a literal
term (see chapter \ref{datatypes}). Any module attribute can be specified. The
attributes are stored in the compiled code and can be retrieved by
calling the function \texttt{Module:module\_info(attributes).}

\subsection{Pre-defined module attributes}
Pre-defined module attributes must be placed before any function
declaration.

\begin{itemize}

	\item \begin{erlangim}
	-module(Module).
	\end{erlangim}
	This attribute is mandatory and must be specified first. It
        defines the name of the module. The name Module, an atom (see section \ref{datatypes:atom}),
        should be the same as the filename without the `\texttt{.erl}' extension.

	\item \begin{erlangim}
	-export([Func1/Arity1, ..., FuncN/ArityN]).
	\end{erlangim}
	This attribute specifies which functions in the module that
        can be called from outside the module. Each function name
        \texttt{FuncX} is an atom and \texttt{ArityX} an integer.

	\item \begin{erlangim}
	-import(Module,[Func1/Arity1, ..., FuncN/ArityN]).
	\end{erlangim}
	This attribute indicates a \texttt{Module} from which a list of functions
        are imported.  For example:

	\begin{erlangim}
	-import(demo, [double/1]).
	\end{erlangim}
	This means that it is possible to write \texttt{double(10)} instead of
        the longer \texttt{demo:double(10)} which can be impractical if the
        function is used frequently.

	\item \begin{erlangim}
	-compile(Options).
	\end{erlangim}
	Compiler options.

	\item \begin{erlangim}
	-vsn(Vsn).
	\end{erlangim}
	Module version. If this attribute is not specified, the
        version defaults to the checksum of the module.

	\item \begin{erlangim}
	-behaviour(Behaviour).
	\end{erlangim}
	This attribute either specifies a user defined behaviour or
        one of the OTP standard behaviours \texttt{gen\_server},
        \texttt{gen\_fsm}, \texttt{gen\_event} or
        \texttt{supervisor}. The spelling ``behavior'' is also accepted.

\end{itemize}


\subsection{Macro and record definitions}

Records and macros are defined in the same way as module attributes:

\begin{erlang}
-record(Record,Fields).

-define(Macro,Replacement).
\end{erlang}

Records and macro definitions are also allowed between functions, as
long as the definition comes before its first use. (About records see
section \ref{datatypes:record} and about macros see chapter \ref{macros}.)

\subsection{File inclusion}

File inclusion is specified in the same way as module attributes:

\begin{erlang}
-include(File).

-include_lib(File).
\end{erlang}

\texttt{File} is a string that represents a file name. Include files
are typically used for record and macro definitions that are shared by
several modules. By convention, the extension \texttt{.hrl} is used
for include files.

\begin{erlang}
-include("my_records.hrl").
-include("incdir/my_records.hrl").
-include("/home/user/proj/my_records.hrl").
\end{erlang}

If \texttt{File} starts with a path component \texttt{\$Var}, then the value of
the environment variable \texttt{Var} (returned by
\texttt{os:getenv(Var)}) is substituted for \texttt{\$Var}.

\begin{erlang}
-include("$PROJ_ROOT/my_records.hrl").
\end{erlang}
%%$ texmaker parser bug

\texttt{include\_lib} is similar to \texttt{include}, but the
first path component is assumed to be the name of an application.

\begin{erlang}
-include_lib("kernel/include/file.hrl").
\end{erlang}

The code server uses \texttt{code:lib\_dir(kernel)} to find the
directory of the current (latest) version of \texttt{kernel}, and then
the subdirectory \texttt{include} is searched for the file \texttt{file.hrl}.


\section{Comments}
Comments may appear anywhere in a module except within strings and
quoted atoms.  A comment begins with the percentage character
(\texttt{\%}) and covers the rest of the line but not the
end-of-line. The terminating end-of-line has the effect of a blank.


\section{Character Set}
Erlang handles the full Latin-1 (ISO-8859-1) character set. Thus all
Latin-1 printable characters can be used and displayed without the
escape backslash. Atoms and variables can use all Latin-1 characters.

\vspace*{12pt}
\begin{center}
\begin{tabular}{|>{\raggedright}p{52pt}|>{\raggedright}p{53pt}|>{\raggedright}p{103pt}|>{\raggedright}p{87pt}|}
\hline
\multicolumn{4}{|p{297pt}|}{Character classes}\tabularnewline
\hline
Octal & Decimal~ &   & Class\tabularnewline
\hline
40 -  57 & 32 - 47 &  ! \texttt{"} \# \$ \% \& ' / & Punctuation
characters\tabularnewline
\hline
60 -  71 & 48 - 57 & 0 - 9 & Decimal digits\tabularnewline
\hline
72 - 100 & 58 - 64 & : ; \texttt{<} = \texttt{>} @ & Punctuation characters\tabularnewline
\hline
101 - 132 &  65 - 90 & A - Z & Uppercase letters\tabularnewline
\hline
133 - 140 &  91 - 96 & [ \textbackslash{} ] \textasciicircum{} \_ ` & Punctuation
characters\tabularnewline
\hline
141 - 172 &  97 - 122 & a  -  z & Lowercase letters\tabularnewline
\hline
173 - 176 & 123 - 126 & \{ \textbar{} \} \textasciitilde{} & Punctuation characters\tabularnewline
\hline
200 - 237 & 128 - 159 ~ &   & Control characters \tabularnewline
\hline
240 - 277 & 160 - 191 & - ¿  & Punctuation characters \tabularnewline
\hline
300 - 326 & 192 - 214 & À - Ö  & Uppercase letters \tabularnewline
\hline
327  & 215 & ×  & Punctuation character \tabularnewline
\hline
330 - 336 & 216 - 222 & Ø - Þ  & Uppercase letters \tabularnewline
\hline
337 - 366 & 223 - 246 & ß - ö  & Lowercase letters \tabularnewline
\hline
367  & 247 & ÷  & Punctuation character \tabularnewline
\hline
370 - 377 & 248 - 255 & ø - ÿ  & Lowercase letters \tabularnewline
\hline
\end{tabular}
\end{center}

% because of where this lands, force a page break to avoid orphan.
\newpage
\section{Reserved words}

%\vspace{12pt}

The following are reserved words in Erlang:

\begin{erlang}
after and andalso band begin bnot bor bsl bsr bxor case catch cond
div end fun if let not of or orelse receive rem try when xor
\end{erlang}

\chapter{Data types (terms)}
\label{datatypes}

\section{Unary data types}

\subsection{Atoms}
\label{datatypes:atom}
An \textbf{atom} is a symbolic name, also known as a
\textit{literal}.  Atoms begin with a lower-case letter, and may contain alphanumeric characters, underscores (\texttt{\_}) or at-signs (\texttt{@}).
Alternatively atoms can be specified by enclosing them in single quotes (\texttt{'}), necessary when they start with an uppercase character or contain
characters other than underscores and at-signs.  For example:

\texttt{hello}

\texttt{phone\_number}

\texttt{'Monday'}

\texttt{'phone number'}

\texttt{'Anything inside quotes \textbackslash n\textbackslash 012'}
\hfill(see section \ref{datatypes:escapeseq})


\subsection{Booleans}
\label{datatypes:boolean}
There is no \textbf{boolean} data type in Erlang. The atoms \texttt{true} and
\texttt{false} are used instead.

\texttt{2 =< 3} \resultingin \texttt{true} \\
\texttt{true or false} \resultingin \texttt{true}


\subsection{Integers}
\label{datatypes:integer}
In addition to the normal way of writing \textbf{integers} Erlang
provides further notations. \texttt{\$Char} is the Latin-1 numeric value of the
character `\texttt{Char}' (that may be an escape sequence) and \texttt{Base\#Value} is an integer in base \texttt{Base}, which
must be an integer in the range $2..36$.

\texttt{42} \resultingin \texttt{42} \\
\$A  \resultingin \texttt{65} \\
\texttt{\$\textbackslash n} \resultingin \texttt{10}\hfill(see section \ref{datatypes:escapeseq}) \\
\texttt{2\#101} \resultingin \texttt{5} \\
\texttt{16\#1f} \resultingin \texttt{31}


\subsection{Floats}
\label{datatypes:float}
A \textbf{float} is a real number written \texttt{Num}[\texttt{eExp}]
where \texttt{Num} is a decimal number between 0.01 and 10000 and
\texttt{Exp} (optional) is a signed integer specifying the power-of-10 exponent.  For example:

\texttt{2.3e-3} \resultingin \texttt{2.30000e-3}\hfill
(corresponding to 2.3*10\textsuperscript{-3})


\subsection{References}
\label{datatypes:reference}
A \textbf{reference} is a term which is unique in an Erlang runtime
system, created by the built-in function \texttt{make\_ref/0}.  (For more information on built-in functions, or \textit{BIF}s, see section \ref{functions:bifs}.)


\subsection{Ports}
\label{datatypes:port}
A \textbf{port identifier} identifies a port (see chapter \ref{ports}).


\subsection{Pids}
\label{datatypes:pid}
A \textbf{process identifier}, \textit{pid}, identifies a process (see
chapter \ref{processes}).


\subsection{Funs}
\label{datatypes:fun}
A \textit{fun} identifies a \textbf{functional object} (see section
\ref{functions:funs}).


\section{Compound data types}

\subsection{Tuples}
\label{datatypes:tuple}
A \textbf{tuple} is a compound data type that holds a \textbf{fixed
number of terms} enclosed within curly braces.

\texttt{\{Term1,...,TermN\}}

Each \texttt{TermX} in the tuple is called an \textbf{element}. The
number of elements is called the \textbf{size} of the tuple.

\begin{center}
\begin{tabular}{|>{\raggedright}p{134pt}|>{\raggedright}p{186pt}|}
\hline
\multicolumn{2}{|p{321pt}|}{BIFs to manipulate tuples}\tabularnewline
\hline
\texttt{size(Tuple)} & Returns the size of \texttt{Tuple}\tabularnewline
\hline
\texttt{element(N,Tuple)} & Returns the \texttt{N}\textsuperscript{th} element in \texttt{Tuple}\tabularnewline
\hline
\texttt{setelement(N,Tuple,Expr)} & Returns a new tuple copied from \texttt{Tuple} except that the
\texttt{N}\textsuperscript{th} element is replaced by \texttt{Expr}\tabularnewline
\hline
\end{tabular}
\end{center}

\texttt{P = \{adam, 24, \{july, 29\}\}} \resultingin \texttt{P} is bound to \texttt{\{adam, 24, \{july, 29\}\}} \\
\texttt{element(1, P)} \resultingin \texttt{adam} \\
\texttt{element(3, P)} \resultingin \texttt{ \{july,29\}} \\
\texttt{P2 = setelement(2, P, 25)} \resultingin \texttt{P2} is bound to \texttt{\{adam, 25, \{july, 29\}\}} \\
\texttt{size(P)} \resultingin \texttt{3} \\
\texttt{size(\{\})} \resultingin \texttt{0} \\


\subsection{Records}
\label{datatypes:record}
A \textbf{record} is a \textit{named tuple} with named elements
called \textbf{fields}.  A record type is defined as a module attribute, for example:

\begin{erlang}
-record(Rec, {Field1 [= Value1],
              ...
              FieldN [= ValueN]}).
\end{erlang}

\texttt{Rec} and \texttt{Fields} are atoms and each \texttt{FieldX}
can be given an optional default \texttt{ValueX}. This definition may
be placed amongst the functions of a module, but only before it is used.  If a record type is
used by several modules it is advisable to put it in a separate file for inclusion.

A new record of type \texttt{Rec} is created using an expression like this:

\begin{erlang}
#Rec{Field1=Expr1, ..., FieldK=ExprK [, _=ExprL]}
\end{erlang}

The fields need not be in the same order as in the record
definition. Fields omitted will get their respective default
values. If the final clause is used, omitted fields will get the value
\texttt{ExprL}. Fields without default values and that are omitted
will have an undefined value.

The value of a field is retrieved using the expression
``\texttt{Variable\#Rec.Field}''.

\begin{erlang}
-module(employee).
-export([new/2]).
-record(person, {name, age, employed=erixon}).

new(Name, Age) -> #person{name=Name, age=Age}.
\end{erlang}

The function \texttt{employee:new/2} can be used in another module
which must also include the same record definition of \texttt{person}.

\texttt{\{P = employee:new(ernie,44)\}} \resultingin \texttt{\{person, ernie, 44,
erixon\}} \\
\texttt{P\#person.age} \resultingin \texttt{44} \\
\texttt{P\#person.employed} \resultingin \texttt{erixon}

When working with records in the Erlang shell, the functions \texttt{rd(RecordName, RecordDefinition)} and \texttt{rr(Module)} can be used to
define and load record definitions.  Refer to the \textit{Erlang Reference Manual} for more information.


\subsection{Lists}
\label{datatypes:list}
A \textbf{list} is a compound data type that holds a \textit{variable}
number of \textbf{terms} enclosed within square brackets.

\texttt{[Term1,...,TermN]}

Each term \texttt{TermX} in the list is called an
\textbf{element}.  The \textbf{length} of a list refers to the number of elements.  Common in functional programming,
the first element is called the \textbf{head} of the list
and the remainder (from the 2\textsuperscript{nd} element onwards) is called
the \textbf{tail} of the list.  Note that individual elements within a list do not have to have the same type, although it is common (and perhaps good)
practice to do so --- where mixed types are involved, \textbf{records} are more commonly used.

\begin{center}
\begin{tabular}{|>{\raggedright}p{90pt}|>{\raggedright}p{230pt}|}
\hline
\multicolumn{2}{|p{321pt}|}{BIFs to manipulate lists}\tabularnewline
\hline
\texttt{length(List)} & Returns the length of \texttt{List}\tabularnewline
\hline
\texttt{hd(List)} & Returns the 1\textsuperscript{st} (head) element of \texttt{List}\tabularnewline
\hline
\texttt{tl(List)} & Returns \texttt{List} with the 1\textsuperscript{st} element removed (tail)\tabularnewline
\hline
\end{tabular}
\end{center}

The vertical bar operator (\textbar{}) separates the leading elements of a list (one or more) from the remainder.  For example:

\texttt{[H | T]  = [1, 2, 3, 4, 5]} \resultingin \texttt{H=1} and \texttt{T=[2, 3, 4, 5]} \\
\texttt{[X, Y | Z] = [a, b, c, d, e]} \resultingin \texttt{X=a}, \texttt{Y=b} and \texttt{Z=[c, d, e]}

Implicitly a list will end with an empty list, i.e.~\texttt{[a, b]} is
the same as \texttt{[a, b | []]}.  A list looking like \texttt{[a, b | c]}
is \textbf{badly formed} and should be avoided (because the atom '\texttt{c}' is \textit{not} a list).
Lists lend themselves naturally to recursive functional programming.  For example, the following
function `\texttt{sum}' computes the sum of a list, and `\texttt{double}' multiplies each element in a list by 2, constructing and returning a new list
as it goes.

\begin{erlang}
sum([]) -> 0;
sum([H | T]) -> H + sum(T).

double([]) -> [];
double([H | T]) -> [H*2 | double(T)].
\end{erlang}

The above definitions introduce \textit{pattern matching}, described in chapter \ref{patterns}.  Patterns of this form are common in recursive programming,
explicitly providing a ``base case'' (for the empty list in these examples).

For working with lists, the operator \texttt{++} joins two lists together (appends the second argument to the first) and
returns the resulting list.  The operator \texttt{--} produces a list that
is a copy of its first argument, except that for each element in
the second argument, the first occurrence of this element (if any) in the resulting list is
removed.

\texttt{[1,2,3] ++ [4,5]} \resultingin \texttt{[1,2,3,4,5]}
\texttt{[1,2,3,2,1,2] -- [2,1,2]} \resultingin \texttt{[3,1,2]}

A collection of list processing functions can be found in the
\texttt{STDLIB} module \texttt{lists}.


\subsection{Strings}
\label{datatypes:string}
\textbf{Strings} are character strings enclosed within double quotes
but are, in fact, stored as lists of characters.

\texttt{"abcdefghi"} is the same as \texttt{[97,98,99,100,101,102,103,104,105]}

\texttt{""} is the same as \texttt{[]}

Two adjacent strings will be concatenated into one at compile-time and
do not incur any runtime overhead.

\texttt{"string" "42"} \resultingin \texttt{"string42"}


\subsection{Binaries}
\label{datatypes:binary}
A binary is a chunk of untyped memory by default a sequence of 8-bit
bytes.

\texttt{<}\texttt{<Elem1,...,ElemN>}\texttt{>}

Each \texttt{ElemX} is specified as \texttt{Value}[\texttt{:Size}][\texttt{/TypeSpecifierList}].

\begin{center}
\begin{tabular}{|>{\raggedright}p{73pt}|>{\raggedright}p{81pt}|>{\raggedright}p{147pt}|}
\hline
\multicolumn{3}{|p{297pt}|}{Element specification}\tabularnewline
\hline
\texttt{Value} & \texttt{Size} & \texttt{TypeSpecifierList}\tabularnewline
\hline
Should evaluate to an integer, float or binary & Should
evaluate to an integer & A sequence of optional type specifiers, in any order,
separated by hyphens (-)\tabularnewline
\hline
\end{tabular}
\end{center}

% page break at this point, so turned into two chunks.
\begin{center}
\begin{tabular}{|>{\raggedright}p{47pt}|>{\raggedright}p{115pt}|>{\raggedright}p{147pt}|}
\hline
\multicolumn{3}{|p{297pt}|}{Type specifiers}\tabularnewline
\hline
Type & \texttt{integer} \textbar{} \texttt{float} \textbar{} \texttt{binary} & Default
is \texttt{integer}\tabularnewline
\hline
Signedness & \texttt{signed} \textbar{} \texttt{unsigned} & Default is
\texttt{unsigned}\tabularnewline
\hline
Endianness & \texttt{big} \textbar{} \texttt{little} \textbar{} \texttt{native} & CPU
dependent. Default is \texttt{big}\tabularnewline
\hline
Unit & \texttt{unit:}\textit{IntegerLiteral} & Allowed range is $1..256$.
Default is 1 for integer and float, and 8 for binary\tabularnewline
\hline
\end{tabular}
\end{center}

The value of \texttt{Size} multiplied by the unit gives the number of
bits for the segment. Each segment can consist of zero or more bits
but the total number of bits must be a multiple of 8, or a
\texttt{badarg} run-time error will occur. Also, a segment of type
binary must have a size evenly divisible by 8.

Binaries cannot be nested.

\begin{erlang}
<<1, 17, 42>>       % <<1, 17, 42>>
<<"abc">>           % <<97, 98, 99>> (The same as <<$a, $b, $c>>)
<<1, 17, 42:16>>    % <<1,17,0,42>>
<<>>                % <<>>
<<15:8/unit:10>>    % <<0,0,0,0,0,0,0,0,0,15>>
<<(-1)/unsigned>>   % <<255>>
\end{erlang}


\section{Escape sequences}
\label{datatypes:escapeseq}
Escape sequences are allowed in strings and quoted atoms.

\begin{center}
\begin{tabular}{|>{\raggedright}p{91pt}|>{\raggedright}p{229pt}|}
\hline
\multicolumn{2}{|p{321pt}|}{Escape sequences}\tabularnewline
\hline
\textbackslash{}b & Backspace\tabularnewline
\hline
\textbackslash{}d & Delete\tabularnewline
\hline
\textbackslash{}e & Escape\tabularnewline
\hline
\textbackslash{}f & Form feed\tabularnewline
\hline
\textbackslash{}n & New line\tabularnewline
\hline
\textbackslash{}r & Carriage return\tabularnewline
\hline
\textbackslash{}s & Space\tabularnewline
\hline
\textbackslash{}t & Tab\tabularnewline
\hline
\textbackslash{}v & Vertical tab\tabularnewline
\hline
\textbackslash{}XYZ, \textbackslash{}XY, \textbackslash{}X & Character with octal
representation XYZ, XY or X\tabularnewline
\hline
\textbackslash{}\textasciicircum{}A .. \textbackslash{}\textasciicircum{}Z & Control
A to control Z\tabularnewline
\hline
\textbackslash{}\textasciicircum{}a .. \textbackslash{}\textasciicircum{}z & Control
A to control Z\tabularnewline
\hline
\textbackslash{}' & Single quote\tabularnewline
\hline
\textbackslash{}\textbf{\texttt{"}} & Double quote\tabularnewline
\hline
\textbackslash{}\textbackslash{} & Backslash\tabularnewline
\hline
\end{tabular}
\end{center}

\section{Type conversions}
There are a number of built-in functions for type conversion:

\begin{center}
\begin{tabular}{|>{\raggedright}p{63pt}|>{\raggedright}p{21pt}|>{\raggedright}p{25pt}|>{\raggedright}p{21pt}|>{\raggedright}p{21pt}|>{\raggedright}p{21pt}|>{\raggedright}p{21pt}|>{\raggedright}p{21pt}|>{\raggedright}p{24pt}|}
\hline
\multicolumn{9}{|p{243pt}|}{Type conversions}\tabularnewline
\hline
 & atom & integer & float & pid & fun & tuple & list & binary\tabularnewline
\hline
atom &  & - & - & - & - & - & X & X\tabularnewline
\hline
integer & - &  & X & - & - & - & X & X\tabularnewline
\hline
float & - & X &  & - & - & - & X & X\tabularnewline
\hline
pid & - & - & - &  & - & - & X & X\tabularnewline
\hline
fun & - & - & - & - &  & - & X & X\tabularnewline
\hline
tuple & - & - & - & - & - &  & X & X\tabularnewline
\hline
list & X & X & X & X & X & X &  & X\tabularnewline
\hline
binary & X & X & X & X & X & X & X & \tabularnewline
\hline
\end{tabular}
\end{center}

The BIF \texttt{float/1} converts integers to floats. The BIFs
\texttt{round/1} and \texttt{trunc/1} convert floats to integers.

The BIFs \texttt{Type\_to\_list/1} and \texttt{list\_to\_Type/1}
convert to and from lists.

The BIFs \texttt{term\_to\_binary/1} and \texttt{binary\_to\_term/1}
convert to and from binaries.

\begin{erlang}
atom_to_list(hello)        % "hello"
list_to_atom("hello")      % hello
float_to_list(7.0)         % "7.00000000000000000000e+00"
list_to_float("7.000e+00") % 7.00000
integer_to_list(77)        % "77"
list_to_integer("77")      % 77
tuple_to_list({a, b ,c})   % [a,b,c]
list_to_tuple([a, b, c])   % {a,b,c}
pid_to_list(self())        % "<0.25.0>"
term_to_binary(<<17>>)     % <<131,109,0,0,0,1,17>>
term_to_binary({a, b ,c})  % <<131,104,3,100,0,1,97,100,0,1,98,100,0,1,99>>
binary_to_term(<<131,104,3,100,0,1,97,100,0,1,98,100,0,1,99>>)  % {a,b,c}
term_to_binary(math:pi())  % <<131,99,51,46,49,52,49,53,57,50,54,53,51,...>>
\end{erlang}


\chapter{Сопоставление с образцом}
\label{patterns}

\section{Переменные}
\label{patterns:variables}

\textbf{Переменные} представлены, как аргументы функции или как результат 
сопоставления с образцом. Переменные начинаются с заглавной буквы или символа
подчёркивания (\texttt{\_}) и могут содержить буквенно-цифровые символы, 
подчёркивания и символы \emph{at} (\texttt{@}). Переменные могут быть связаны со 
значением (присвоены) только один раз.

\begin{erlang}
Abc
A_long_variable_name
AnObjectOrientatedVariableName
_Height
\end{erlang}

\textbf{Анонимная переменная} объявляется с помощью одного символа подчёркивания
(\texttt{\_}) и может использоваться там, где требуется переменная, но её значение
нас не интересует и может быть проигнорировано.

\begin{erlang}
[H|_] = [1,2,3]         % H=1 и всё остальное игнорируется
\end{erlang}

Переменные, начинающиеся с символа подчёркивания, как, например, 
\texttt{\_Height}, являются обычными не анонимными переменными.  Однако они
игнорируются компилятором в том смысле, что они не произведут предупреждений
компилятора о неиспользуемых переменных.  Таким образом, возможна следующая
запись:

\begin{erlang}
member(_Elem, []) ->
    false.
\end{erlang}

вместо:

\begin{erlang}
member(_, []) ->
    false.
\end{erlang}

что улучшает читаемость кода.

\textit{Область видимости} для переменной --- это её уравнение функции.  
Переменные, связанные со значением в ветке \texttt{if}, \texttt{case} или 
\texttt{receive}, должны быть связаны с чем-нибудь во всех ветвях этого 
оператора, чтобы иметь значение за пределами выражения, иначе компилятор будет
считать это значение \textit{небезопасным} (unsafe) (вероятно, не присвоенным)
за пределами этого выражения, и выдаст соответствующее предупреждение.


\section{Сопоставление с образцом}

\textbf{Образец} имеет такую же структуру, как и терм, но может содержать новые
свободные переменные. Например:

\begin{erlang}
Name1
[H|T]
{error,Reason}
\end{erlang}

Образцы могут встречаться в \textit{заголовках функций}, выражениях 
\textit{case}, \textit{receive}, и \textit{try} и в выражениях оператора
сопоставления (\texttt{=}). Образцы вычисляются посредством 
\textbf{сопоставления образца} с выражением, и таким образом новые переменные
определяются и связываются со значением.

\begin{erlangru}
Образец = Выражение
\end{erlangru}

Обе стороны выражения должны иметь одинаковую структуру. Если сопоставление 
проходит успешно, то все свободные переменные (если такие были) в образце слева
становятся связанными. Если сопоставление не проходит, то возникает ошибка
времени исполнения \texttt{badmatch}.


\begin{minted}{console}
> {A, B} = {answer, 42}.
{answer,42}
> A.
answer
> B.
42
\end{minted}


\subsection{Оператор сопоставления (\texttt{=}) в образцах}

Если \texttt{Образец1} и \texttt{Образец2} являются действительными образцами, тогда следующая запись тоже действительный образец:

\begin{erlangru}
Образец1 = Образец2
\end{erlangru}

Оператор \texttt{=} представляет собой \textbf{подмену} (alias), при
сопоставлении которой с выражением, оба и \texttt{Образец1} и \texttt{Образец2}
также сопоставляются с ней. Цель этого --- избежать необходимости повторно
строить термы, которые были разобраны на составляющие в сопоставлении.

\begin{erlang}
foo({connect,From,To,Number,Options}, To) ->
    Signal = {connect,From,To,Number,Options},
    fox(Signal),
    ...;
\end{erlang}

можно более эффективно записать, как:

\begin{erlang}
foo({connect,From,To,Number,Options} = Signal, To) ->
    fox(Signal),
    ...;
\end{erlang}


\subsection{Строковой префикс в образцах}

При сопоставлении строк с образцом, следующая запись является действительным 
образцом:

\begin{erlang}
f("prefix" ++ Str) -> ...
\end{erlang}

что эквивалентно и легче читается, чем следующая запись:

\begin{erlang}
f([$p,$r,$e,$f,$i,$x | Str]) -> ...
\end{erlang}

Вы можете использовать строки только как префикс; варианты с постфиксом для 
образцов, такие как \texttt{Str ++ "postfix"} не разрешаются. 


\subsection{Выражения в образцах}

Арифметическое выражение может быть использовано внутри образца, если оно 
использует только числовые, битовые операторы, и его значение является 
константой, которая может быть вычислена во время компиляции.

\begin{erlang}
case {Value, Result} of
    {?Threshold+1, ok} -> ...   % ?Threshold - это макрос
\end{erlang}


\subsection{Сопоставление двоичных данных}

\begin{erlang}
Bin = <<1, 2, 3>>      % <<1,2,3>> Все элементы - 8-битные байты
<<A, B, C>> = Bin      % A=1, B=2 и C=3
<<D:16, E>> = Bin      % D=258 и E=3
<<F, G/binary>> = Bin  % F=1 и G=<<2,3>>
\end{erlang}

В последней строке переменная \texttt{G} неуказанного размера сопоставляется с 
остатком двоичных данных \texttt{Bin}.

Всегда ставьте пробел между оператором (\texttt{=}) и (\verb|<<|), чтобы 
избежать возможной путаницы с оператором (\texttt{=<}).


\chapter{Functions}

\section{Function definition}
A function is defined as a sequence of one or more \textbf{function
clauses}. The function name is an atom.

\vspace*{4pt}
\begin{erlang}
Func(Pattern11,...,Pattern1N) [when GuardSeq1] -> Body1;
    ...;
    ...;
Func(PatternK1,...,PatternKN) [when GuardSeqK] -> BodyK.

\end{erlang}
\vspace*{4pt}

The function clauses are separated by semicolons (\texttt{;}) and
terminated by full stop (\texttt{.}). A function clause consists of a
\textbf{clause head} and a \textbf{clause body} separated by an arrow
(\texttt{->}). A clause head consists of the function name (an atom),
arguments within parentheses and an optional guard sequence beginning
with the keyword \texttt{when}.  Each argument is a pattern.  A clause
body consists of a sequence of expressions separated by commas
(\texttt{,}).

\vspace*{4pt}
\begin{erlang}
Expr1,
...,
ExprM
\end{erlang}
\vspace*{4pt}

The number of arguments \texttt{N} is the \textbf{arity} of the
function. A function is uniquely defined by the module name, function name
and arity. Two different functions in the same module with different
arities may have the same name. A function \texttt{Func} in \texttt{Module}
with arity \texttt{N} is often denoted as \texttt{Module:Func/N}.

\vspace*{4pt}
\begin{erlang}
-module(mathStuff).
-export([area/1]).

area({square, Side}) -> Side * Side;
area({circle, Radius}) -> math:pi() * Radius * Radius;
area({triangle, A, B, C}) ->
    S = (A + B + C)/2,
    math:sqrt(S*(S-A)*(S-B)*(S-C)).
\end{erlang}

% push this section onto the next page (orphan line)
\newpage
\section{Function calls}
A function is called using:

\begin{erlang}
[Module:]Func(Expr1, ..., ExprN)
\end{erlang}

\texttt{Module} evaluates to a module name and \texttt{Func} to
a function name or a \textit{fun}. When calling a function in another
module, the module name must be provided and the function must be
exported. This is referred to as a \textbf{fully qualified function name}.

\begin{erlang}
lists:keysearch(Name, 1, List)
\end{erlang}

The module name can be omitted if \texttt{Func} evaluates to the name
of a local function, an imported function, or an auto-imported
BIF.  In such cases, the function is called using an \textbf{implicitly qualified function name}.

Before calling a function the arguments \texttt{ExprX} are
evaluated.  If the function cannot be found, an \texttt{undef} run-time
error will occur. Next the function clauses are scanned sequentially
until a clause is found such that the patterns in the clause head can
be successfully matched against the given arguments and that the guard
sequence, if any, is true. If no such clause can be found, a
\texttt{function\_clause} run-time error will occur.

If a matching clause is found, the corresponding clause body is evaluated,
i.e.~the expressions in the body are evaluated sequentially and the
value of the last expression is returned.

The fully qualified function name must be used when calling a function
with the same name as a BIF (built-in function, see section
\ref{functions:bifs}). The compiler does not allow defining a function
with the same name as an imported function. When calling a local
function, there is a difference between using the implicitly or fully
qualified function name, as the latter always refers to the latest
version of the module (see chapter \ref{code}).


\section{Expressions}
\label{functions:expressions}
An \textbf{expression} is either a term or the invocation of an
operator, for example:

\begin{erlang}
Term
op Expr
Expr1 op Expr2
(Expr)
begin
   Expr1,
   ...,
   ExprM            % no comma (,) before end
end

\end{erlang}

There are both \textit{unary} and \textit{binary} operators. The
simplest form of expression is a term, i.e.~an \textit{integer},
\textit{float}, \textit{atom}, \textit{string}, \textit{list} or
\textit{tuple} and the return value is the term itself. An expression
may contain \textit{macro} or \textit{record} expressions which will
expanded at compile time.

Parenthesised expressions are useful to override operator precedence (see section \ref{functions:expressions:precedence}):

\begin{erlang}
1 + 2 * 3           % 7
(1 + 2) * 3         % 9
\end{erlang}

Block expressions within \texttt{begin...end} can be used to group a
sequence of expressions and the return value is the value of the last
expression \texttt{ExprM}.

All subexpressions are evaluated before the expression itself is
evaluated, but the order in which subexpressions are evaluated is undefined.

Most operators can only be applied to arguments of a certain type. For
example, arithmetic operators can only be applied to integers or
floats. An argument of the wrong type will cause a \texttt{badarg}
run-time error.


\subsection{Term comparisons}
\begin{erlang}
Expr1 op Expr2
\end{erlang}

A \textbf{term comparison} returns a \textit{boolean} value,
in the form of atoms \texttt{true} or \texttt{false}.

\begin{center}
\begin{tabular}{|>{\raggedright}p{40pt}|>{\raggedright}p{105pt}|>{\raggedright}p{26pt}|>{\raggedright}p{124pt}|}
\hline
\multicolumn{4}{|p{297pt}|}{Comparison operators}\tabularnewline
\hline
\texttt{==} & Equal to & \texttt{=<} & Less than or equal to\tabularnewline
\hline
\texttt{/=} & Not equal to & \texttt{<} & Less than\tabularnewline
\hline
\texttt{=:=} & Exactly equal to & \texttt{>}= & Greater than or equal to\tabularnewline
\hline
\texttt{=/=} & Exactly not equal to & \texttt{>} & Greater than\tabularnewline
\hline
\end{tabular}
\end{center}

\begin{erlang}
1==1.0              % true
1=:=1.0             % false
1 > a               % false
\end{erlang}

The arguments may be of different data types. The following order is
defined:

\texttt{number < atom < reference < fun < port < pid < tuple < list < binary}

Lists are compared element by element. Tuples are ordered by size, two
tuples with the same size are compared element by element. When
comparing an integer and a float, the integer is first converted to a
float. In the case of \texttt{=:=} or \texttt{=/=} there is no type conversion.


\subsection{Arithmetic expressions}

\begin{erlang}
op Expr
Expr1 op Expr2
\end{erlang}

An \textbf{arithmetic expression} returns the result after applying
the operator.

\begin{center}
\begin{tabular}{|>{\raggedright}p{35pt}|>{\raggedright}p{145pt}|>{\raggedright}p{128pt}|}
\hline
\multicolumn{3}{|p{309pt}|}{Arithmetic operators}\tabularnewline
\hline
\texttt{+}  & Unary +  & \texttt{Integer \textbar{} Float} \tabularnewline
\hline
\texttt{-}  & Unary -  & \texttt{Integer \textbar{} Float}\tabularnewline
\hline
\texttt{+}  & Addition & \texttt{Integer} \textbar{} \texttt{Float}\tabularnewline
\hline
\texttt{-}  & Subtraction & \texttt{Integer} \textbar{} \texttt{Float}\tabularnewline
\hline
\texttt{*}  & Multiplication & \texttt{Integer} \textbar{} \texttt{Float}\tabularnewline
\hline
\texttt{/}  & Floating point division  & \texttt{Integer} \textbar{} \texttt{Float}\tabularnewline
\hline
\texttt{bnot}  & Unary bitwise not  & \texttt{Integer} \tabularnewline
\hline
\texttt{div}  & Integer division  & \texttt{Integer}\tabularnewline
\hline
\texttt{rem}  & Integer remainder of X/Y  & \texttt{Integer} \tabularnewline
\hline
\texttt{band}  & Bitwise and  & \texttt{Integer}\tabularnewline
\hline
\texttt{bor}  & Bitwise or  & \texttt{Integer} \tabularnewline
\hline
\texttt{bxor}  & Arithmetic bitwise xor  & \texttt{Integer}\tabularnewline
\hline
\texttt{bsl}  & Arithmetic bitshift left  & \texttt{Integer} \tabularnewline
\hline
\texttt{bsr}  & Bitshift right  & \texttt{Integer}\tabularnewline
\hline
\end{tabular}
\end{center}

\begin{erlang}
+1                  % 1
4/2                 % 2.00000
5 div 2             % 2
5 rem 2             % 1
2#10 band 2#01      % 0
2#10 bor 2#01       % 3
\end{erlang}


\subsection{Boolean expressions}

\begin{erlang}
op Expr
Expr1 op Expr2
\end{erlang}

A \textbf{boolean expression} returns the value \texttt{true} or
\texttt{false} after applying the operator.

\begin{center}
\begin{tabular}{|>{\raggedright}p{79pt}|>{\raggedright}p{241pt}|}
\hline
\multicolumn{2}{|p{321pt}|}{Boolean operators}\tabularnewline
\hline
\texttt{not}  & Unary logical not \tabularnewline
\hline
\texttt{and}  & Logical and \tabularnewline
\hline
\texttt{or}  & Logical or \tabularnewline
\hline
\texttt{xor}  & Logical exclusive or\tabularnewline
\hline
\end{tabular}
\end{center}

\begin{erlang}
not true            % false
true and false      % false
true xor false      % true
\end{erlang}


\subsection{Short-circuit boolean expressions}

\begin{erlang}
Expr1 orelse Expr2
Expr1 andalso Expr2
\end{erlang}

These are boolean expressions where \texttt{Expr2} is evaluated only
if necessary. In an \texttt{orelse} expression \texttt{Expr2} will be
evaluated only if \texttt{Expr1} evaluates to false. In an
\texttt{andalso} expression \texttt{Expr2} will be evaluated only if
\texttt{Expr1} evaluates to true.

\begin{erlang}
if A >= 0 andalso math:sqrt(A) > B -> ...

if is_list(L) andalso length(L) == 1 -> ...
\end{erlang}


\subsection{Operator precedences}
\label{functions:expressions:precedence}
In an expression consisting of subexpressions the operators will be
applied according to a defined \textbf{operator precedence} order:

\begin{center}
\begin{tabular}{|>{\raggedright}p{221pt}|>{\raggedright}p{99pt}|}
\hline
\multicolumn{2}{|p{321pt}|}{Operator precedence (from high to low)}\tabularnewline
\hline
\texttt{:} ~ &  \tabularnewline
\hline
\texttt{\#} ~ &  \tabularnewline
\hline
\texttt{Unary + - bnot not ~} &  \tabularnewline
\hline
\texttt{/ * div rem band and}  & Left associative \tabularnewline
\hline
\texttt{+ - bor bxor bsl bsr or xor} & Left associative \tabularnewline
\hline
\texttt{++ --}  & Right associative \tabularnewline
\hline
\texttt{== /= =< < >= > =:= =/=} & \tabularnewline
\hline
\texttt{andalso}  &  \tabularnewline
\hline
\texttt{orelse} &  \tabularnewline
\hline
\texttt{= !}  & Right associative \tabularnewline
\hline
\texttt{catch ~} &  \tabularnewline
\hline
\end{tabular}
\end{center}

The operator with the highest priority is evaluated first. Operators
with the same priority are evaluated according to their
\textbf{associativity}.  The left associative arithmetic operators
are evaluated left to right:

\texttt{6 + 5 * 4 - 3 / 2  \resultingin  6 + 20 - 1.5  \resultingin  26 - 1.5  \resultingin  24.5}


\section{Compound expressions}


\subsection{If}

\begin{erlang}
if
    GuardSeq1 ->
        Body1;
    ...;
    GuardSeqN ->
        BodyN                   % Note no semicolon (;) before end
end
\end{erlang}

The branches of an \texttt{if} expression are scanned sequentially
until a guard sequence \texttt{GuardSeq} which evaluates to
\texttt{true} is found.  The corresponding \texttt{Body} (sequence
of expressions separated by commas) is then evaluated.  The return value of
\texttt{Body} is the return value of the \texttt{if} expression.

If no guard sequence is true, an \texttt{if\_clause} run-time error
will occur. If necessary, the guard expression \texttt{true} can be used in the
last branch, as that guard sequence is always true (known as a ``catch
all'').

\begin{erlang}
is_greater_than(X, Y) ->
    if
        X>Y ->
            true;
        true ->                 % works as an 'else' branch
            false
    end
\end{erlang}

It should be noted that pattern matching in function clauses can be used to replace \texttt{if} cases (most of the time).
Overuse of \texttt{if} sentences withing function bodies is considered a bad Erlang practice.

\subsection{Case}

Case expressions provide for inline pattern matching, similar to the way in which function clauses are matched.

\begin{erlang}
case Expr of
    Pattern1 [when GuardSeq1] ->
        Body1;
        ...;
    PatternN [when GuardSeqN] ->
        BodyN                   % Note no semicolon (;) before end
end
\end{erlang}

The expression \texttt{Expr} is evaluated and the patterns
\texttt{Pattern1}...\texttt{PatternN} are sequentially matched against the result. If a
match succeeds and the optional guard sequence \texttt{GuardSeqX} is
\texttt{true}, then the corresponding \texttt{BodyX} is evaluated. The return value
of \texttt{BodyX} is the return value of the case expression.

If there is no matching pattern with a true guard sequence, a
\texttt{case\_clause} run-time error will occur.

\begin{erlang}
is_valid_signal(Signal) ->
    case Signal of
        {signal, _What, _From, _To} ->
            true;
        {signal, _What, _To} ->
            true;
        _Else ->                % 'catch all'
            false
    end.
\end{erlang}


\subsection{List comprehensions}
List comprehensions are analogous to the \texttt{setof} and
\texttt{findall} predicates in Prolog.

\begin{erlang}
[Expr || Qualifier1,...,QualifierN]
\end{erlang}

\texttt{Expr} is an arbitrary expression, and each \texttt{QualifierX}
is either a \textbf{generator} or a \textbf{filter}. A generator is
written as:

\begin{erlang}
Pattern <- ListExpr
\end{erlang}

where \texttt{ListExpr} must be an expression which evaluates to a
list of terms. A filter is an expression which evaluates to
\texttt{true} or \texttt{false}. Variables in list generator
expressions \textit{shadow} variables in the function clause
surrounding the list comprehension.

The qualifiers are evaluated from left to right, the generators
creating values and the filters constraining them. The list
comprehension then returns a list where the elements are the result of
evaluating \texttt{Expr} for each combination of the resulting values.

\begin{minted}{console}
> [{X, Y} || X <- [1,2,3,4,5,6], X > 4, Y <- [a,b,c]].
[{5,a},{5,b},{5,c},{6,a},{6,b},{6,c}]
\end{minted}


\section{Guard sequences}
A \textbf{guard sequence} is a set of \textbf{guards} separated by
semicolons (\texttt{;}). The guard sequence is \texttt{true} if at
least one of the guards is \texttt{true}.

\begin{erlang}
Guard1; ...; GuardK
\end{erlang}

A \textbf{guard} is a set of \textbf{guard expressions} separated by
commas (\texttt{,}). The guard is \texttt{true} if all guard
expressions evaluate to \texttt{true}.

\begin{erlang}
GuardExpr1, ..., GuardExprN
\end{erlang}

The permitted \textbf{guard expressions} (sometimes called guard
tests) are a subset of valid Erlang expressions, since the
evaluation of a guard expression must be guaranteed to be free of side-effects.

\begin{center}
\begin{tabular}{|>{\raggedright}p{154pt}|>{\raggedright}p{166pt}|}
\hline
\multicolumn{2}{|p{321pt}|}{Valid guard expressions:}\tabularnewline
\hline
\multicolumn{2}{|p{321pt}|}{The atom \texttt{true};}\tabularnewline
\hline
\multicolumn{2}{|p{321pt}|}{Other constants (terms and bound variables), are all regarded
as \texttt{false};}\tabularnewline
\hline
\multicolumn{2}{|p{321pt}|}{Term comparisons;}\tabularnewline
\hline
\multicolumn{2}{|p{321pt}|}{Arithmetic and boolean expressions;}\tabularnewline
\hline
\multicolumn{2}{|p{321pt}|}{Calls to the BIFs specified below.}\tabularnewline
\hline
Type test BIFs & Other BIFs allowed in guards:\tabularnewline
\hline
\texttt{is\_atom/1} & \texttt{abs(Integer} \textbar{} \texttt{Float)}\tabularnewline
\hline
\texttt{is\_constant/1} & \texttt{float(Term)}\tabularnewline
\hline
\texttt{is\_integer/1} & \texttt{trunc(Integer} \textbar{} \texttt{Float)}\tabularnewline
\hline
\texttt{is\_float/1} & \texttt{round(Integer} \textbar{} \texttt{Float)}\tabularnewline
\hline
\texttt{is\_number/1} & \texttt{size(Tuple} \textbar{} \texttt{Binary)}\tabularnewline
\hline
\texttt{is\_reference/1} & \texttt{element(N, Tuple)}\tabularnewline
\hline
\texttt{is\_port/1} & \texttt{hd(List)}\tabularnewline
\hline
\texttt{is\_pid/1} & \texttt{tl(List)}\tabularnewline
\hline
\texttt{is\_function/1} & \texttt{length(List)}\tabularnewline
\hline
\texttt{is\_tuple/1} & \texttt{self()}\tabularnewline
\hline
\texttt{is\_record/2} The 2\textsuperscript{nd} argument is \linebreak{}
the record name & \texttt{node(})\tabularnewline
\hline
\texttt{is\_list/1} & \texttt{node(Pid} \textbar{} \texttt{Ref} \textbar \texttt{Port)}\tabularnewline
\hline
\texttt{is\_binary/1} & \tabularnewline
\hline
\end{tabular}
\end{center}

A small example:

\begin{erlang}
fact(N) when N>0 ->             % first clause head
    N * fact(N-1);              % first clause body
fact(0) ->                      % second clause head
    1.                          % second clause body
\end{erlang}


\section{Tail recursion}
If the last expression of a function body is a function call, a
\textbf{tail recursive} call is performed in such a way that no system
resources (like the call stack) are consumed. This means that an
infinite loop like a server can be programmed provided it only uses
tail recursive calls.

The function \texttt{fact/1} above could be rewritten using tail
recursion in the following manner:

 \begin{erlang}
fact(N) when N>1 -> fact(N, N-1);
fact(N) when N==1; N==0 -> 1.

fact(F,0) -> F;                 % The variable F is used as an accumulator
fact(F,N) -> fact(F*N, N-1).
\end{erlang}


\section{Funs}
\label{functions:funs}
A \textbf{fun} defines a \textit{functional object}. Funs make it
possible to pass an entire function, not just the function name, as an
argument. A `fun' expression begins with the keyword \texttt{fun} and
ends with the keyword \texttt{end} instead of a full stop
(\texttt{.}).  Between these should be a regular function
declaration, except that no function name is specified.

\begin{erlang}
fun
    (Pattern11,...,Pattern1N) [when GuardSeq1] ->
        Body1;
        ...;
    (PatternK1,...,PatternKN) [when GuardSeqK] ->
        BodyK
end
\end{erlang}

Variables in a \texttt{fun} head \textit{shadow} variables in the function
clause surrounding the \texttt{fun} but variables bound in a \texttt{fun} body are local
to the body.  The return value of the expression is the resulting function. The expression
\texttt{fun Name/N is} equivalent to:

\begin{erlang}
fun (Arg1,...,ArgN) -> Name(Arg1,...,ArgN) end
\end{erlang}

The expression \texttt{fun Module:Func/Arity} is also allowed, provided that \texttt{Func} is exported
from \texttt{Module}.

\begin{erlang}
Fun1 = fun (X) -> X+1 end.
Fun1(2)         % 3

Fun2 = fun (X) when X>=1000 -> big; (X) -> small end.
Fun2(2000)      % big
\end{erlang}

Since a \texttt{fun} is anonymous, i.e.~there is no function name in the
definition of the \texttt{fun}, the definition of a recursive \texttt{fun} has to be
done in two steps.  This example shows how to define the function
\texttt{sum(List)} (see section \ref{datatypes:list}) as a \texttt{fun}.

\begin{erlang}
Sum1 = fun ([], _Foo) -> 0;([H|T], Foo) -> H + Foo(T, Foo) end.
Sum = fun (List) -> Sum1(List, Sum1) end.
Sum([1,2,3,4,5])    % 15
\end{erlang}

The definition of \texttt{Sum1} is done in a way such that it takes \textit{itself} as a parameter, matched to \texttt{\_Foo} (empty list) or \texttt{Foo},
which it then calls recursively.  The definition of \texttt{Sum} calls \texttt{Sum1}, also passing \texttt{Sum1} as a parameter.

\section{BIFs --- Built-in functions}
\label{functions:bifs}
The \textbf{built-in functions}, BIFs, are implemented in the C code of
the runtime system and do things that are difficult or impossible to
implement in Erlang. Most of the built-in functions belong to the
module \texttt{erlang} but there are also built-in functions that belong
to other modules like \texttt{lists} and \texttt{ets}. The most
commonly used BIFs belonging to the module \texttt{erlang} are
\textbf{auto-imported}, i.e.~they do not need to be prefixed with the
module name.

\begin{center}
\begin{tabular}{|>{\raggedright}p{103pt}|>{\raggedright}p{217pt}|}
\hline
\multicolumn{2}{|p{321pt}|}{Some useful BIFs}\tabularnewline
\hline
\texttt{date()} & Returns today's date as \texttt{\{Year, Month, Day\}}\tabularnewline
\hline
\texttt{now()} & Returns current time in microseconds. System dependent\tabularnewline
\hline
\texttt{time()} & Returns current time as \texttt{\{Hour, Minute, Second\}} System dependent\tabularnewline
\hline
\texttt{halt()} & Stops the Erlang system\tabularnewline
\hline
\texttt{processes()} & Returns a list of all processes currently known to the system\tabularnewline
\hline
\texttt{process\_info(Pid)} & Returns a dictionary containing information about \texttt{Pid}\tabularnewline
\hline
\texttt{Module:module\_info()} & Returns a dictionary containing information about the code
in Module\tabularnewline
\hline
\end{tabular}
\end{center}

A \textbf{dictionary} is a list of \texttt{\{Key, Value\} terms (see
also section \ref{processes:dicts}).}

\begin{erlang}
size({a, b, c})             % 3
atom_to_list('Erlang')      % "Erlang"
date()                      % {2013,5,27}
time()                      % {01,27,42}
\end{erlang}

\chapter{Processes}
\label{processes}

A \textbf{process} corresponds to one \textit{thread of control}.
Erlang permits very large numbers of concurrent processes,
each executing like it had an own virtual processor.  When a process
executing \texttt{functionA} calls another \texttt{functionB}, it
will wait until \texttt{functionB} is finished and then retrieve its
result. If instead it \textit{spawns} another process executing
\texttt{functionB}, both will continue in parallel
(concurrently). \texttt{functionA} will not wait for \texttt{functionB}
and the only way they can communicate is through \textit{message passing}.

Erlang processes are light-weight with a small memory footprint, fast to
create and shut-down, and the scheduling overhead is low.  A
\textbf{process identifier}, \texttt{Pid}, identifies a process. The
BIF \texttt{self/0} returns the \texttt{Pid} of the calling process.


\section{Process creation}
A process is created using the BIF \texttt{spawn/3}.

\begin{erlang}
spawn(Module, Func, [Expr1, ..., ExprN])
\end{erlang}

\texttt{Module} should evaluate to a module name and \texttt{Func} to
a function name in that module. The list \texttt{Expr1}$...$\texttt{ExprN} are the
arguments to the function.  \texttt{spawn} creates a new process and
returns the process identifier, \texttt{Pid}. The new process starts
by executing:

\begin{erlang}
Module:Func(Expr1, ..., ExprN)
\end{erlang}

The function \texttt{Func} has to be exported even if it is spawned by
another function in the same module. There are other spawn BIFs, for
example \texttt{spawn/4} for spawning a process on another node.


\section{Registered processes}
A process can be associated with a name. The name must be an atom and
is automatically unregistered if the process terminates. Only static
(cyclic) processes should be registered.

\begin{center}
\begin{tabular}{|>{\raggedright}p{117pt}|>{\raggedright}p{204pt}|}
\hline
\multicolumn{2}{|p{321pt}|}{Name registration BIFs}\tabularnewline
\hline
\texttt{register(Name, Pid)}  & Associates the atom \texttt{Name} with the process \texttt{Pid}\tabularnewline
\hline
\texttt{registered()}  & Returns a list of names which have been registered \tabularnewline
\hline
\texttt{whereis(Name)}  & Returns the \texttt{Pid} registered under \texttt{Name} or \texttt{undefined} if the name
is not registered\tabularnewline
\hline
\end{tabular}
\end{center}


\section{Process communication}
Processes communicate by sending and receiving
\textbf{messages}. Messages are sent using the send operator
(\texttt{!}) and are received using \texttt{receive}. Message passing
is asynchronous and reliable, i.e.~the message is guaranteed to
eventually reach the recipient, provided that the recipient exists.

\subsection{Send}
\begin{erlang}
Pid ! Expr
\end{erlang}

The send (\texttt{!}) operator sends the value of \texttt{Expr} as a
message to the process specified by \texttt{Pid} where it will be
placed last in its \textbf{message queue}.  The value of \texttt{Expr}
is also the return value of the (\texttt{!}) expression. \texttt{Pid}
must evaluate to a process identifier, a registered name or a tuple
\texttt{\{Name,Node\}}, where \texttt{Name} is a registered process at
\texttt{Node} (see chapter \ref{distribution}). The message sending operator
(\texttt{!}) never fails, even if it addresses a non-existent process.


\subsection{Receive}

\begin{erlang}
receive
    Pattern1 [when GuardSeq1] ->
        Body1;
    ...
    PatternN [when GuardSeqN] ->
        BodyN                   % Note no semicolon (;) before end
end
\end{erlang}

This expression receives messages sent to the process using the send
operator (\texttt{!}). The patterns \texttt{PatternX} are sequentially
matched against the first message in time order in the message queue,
then the second and so on.  If a match succeeds and the optional guard
sequence \texttt{GuardSeqX} is true, then the message is removed from
the message queue and the corresponding \texttt{BodyX} is
evaluated.  It is the order of the pattern clauses that decides the
order in which messages will be received prior to the order in which
they arrive.  This is called \textit{selective receive}. The
return value of \texttt{BodyX} is the return value of the receive
expression.

\texttt{receive} never fails. The process may be suspended, possibly
indefinitely, until a message arrives that matches one of the patterns
and with a true guard sequence.

\newpage
\begin{erlang}
wait_for_onhook() ->
    receive
        onhook ->
            disconnect(),
            idle();
        {connect, B} ->
            B ! {busy, self()},
            wait_for_onhook()
    end.
\end{erlang}


\subsection{Receive with timeout}

\begin{erlang}
receive
    Pattern1 [when GuardSeq1] ->
        Body1;
        ...;
    PatternN [when GuardSeqN] ->
        BodyN
after
    ExprT ->
        BodyT
end
\end{erlang}

\texttt{ExprT} should evaluate to an integer between \texttt{0} and
\texttt{16\#ffffffff} (the value must fit in 32 bits). If no matching
message has arrived within \texttt{ExprT} milliseconds, then
\texttt{BodyT} will be evaluated and its return value becomes the
return value of the receive expression.

\begin{erlang}
wait_for_onhook() ->
    receive
        onhook ->
            disconnect(),
            idle();
        {connect, B} ->
            B ! {busy, self()},
            wait_for_onhook()
    after
        60000 ->
            disconnect(),
            error()
    end.
\end{erlang}

A \texttt{receive...after} expression with no branches can be used to
implement simple timeouts.

\begin{erlang}
receive
after
    ExprT ->
        BodyT
end
\end{erlang}

\begin{center}
\begin{tabular}{|>{\raggedright}p{47pt}|>{\raggedright}p{273pt}|}
\hline
\multicolumn{2}{|p{321pt}|}{Two special cases for the timeout value \texttt{ExprT}}\tabularnewline
\hline
\texttt{infinity} & This is equivalent to not using a timeout and can be useful for timeout
values that are calculated at run-time\tabularnewline
\hline
\texttt{0} & If there is no matching message in the mailbox, the timeout will occur immediately\tabularnewline
\hline
\end{tabular}
\end{center}


\section{Process termination}
\label{processes:termination}
A process always terminates with an \textbf{exit reason} which may be
any term.  If a process terminates normally, i.e.~it has run
to the end of its code, then the reason is the atom \texttt{normal}. A process
can terminate itself by calling one of the following BIFs.

\begin{erlang}
exit(Reason)

erlang:error(Reason)

erlang:error(Reason, Args)
\end{erlang}

A process terminates with the exit reason \texttt{\{Reason,Stack\}} when a
run-time error occurs.

A process may also be terminated if it receives an exit signal with
a reason other than \texttt{normal} (see section \ref{processes:recvexitsignals}).


\section{Process links}
\label{processes:links}
Two processes can be \textbf{linked} to each other. Links are
bidirectional and there can only be one link between two distinct processes (unique \texttt{Pid}s). A
process with \texttt{Pid1} can link to a process with \texttt{Pid2}
using the BIF \texttt{link(Pid2)}.  The BIF \texttt{spawn\_link(Module, Func, Args)}
spawns and links a process in one atomic operation.

A link can be removed using the BIF \texttt{unlink(Pid)}.


\subsection{Error handling between processes}
When a process terminates it will send \textbf{exit signals} to all
processes that it is linked to.  These in turn will also be terminated
\textit{or handle the exit signal in some way}. This feature can be
used to build hierarchical program structures where some processes are
supervising other processes, for example restarting them if they
terminate abnormally.


\subsection{Sending exit signals}
\label{processes:sendexitsignals}
A process always terminates with an exit reason which is sent as an
exit signal to all linked processes. The BIF \texttt{exit(Pid, Reason)} sends
an exit signal with the reason \texttt{Reason} to \texttt{Pid}, without
affecting the calling process.


\subsection{Receiving exit signals}
\label{processes:recvexitsignals}
If a process receives an exit signal with an exit reason other than
\texttt{normal} it will also be terminated, and will send exit signals with the
same exit reason to its linked processes.  An exit signal with reason
\texttt{normal} is ignored.  This behaviour can be changed using the BIF
\texttt{process\_flag(trap\_exit, true)}.

The process is then able to \textbf{trap exits}.  This means that an
exit signal will be transformed into a message \texttt{\{'EXIT', FromPid, Reason\}} which is
put into the process's mailbox and can be handled by the process like a regular
message using \texttt{receive}.

However, a call to the BIF \texttt{exit(Pid, kill)} unconditionally
terminates the process \texttt{Pid} regardless whether it is able to
trap exit signals or not.


\section{Monitors}
A process \texttt{Pid1} can create a \textbf{monitor} for
\texttt{Pid2} using the BIF:

\begin{erlang}
erlang:monitor(process, Pid2)
\end{erlang}

which returns a reference \texttt{Ref}. If \texttt{Pid2} terminates
with exit reason \texttt{Reason}, a message as follows will be sent to
\texttt{Pid1}:

\begin{erlang}
{'DOWN', Ref, process, Pid2, Reason}
\end{erlang}

If \texttt{Pid2} does not exist, the \texttt{'DOWN'} message is sent
immediately with \texttt{Reason} set to \texttt{noproc}.  Monitors are
unidirectional in that if \texttt{Pid1} monitors \texttt{Pid2} then it
will receive a message when \texttt{Pid2} dies but \texttt{Pid2} will
\textbf{not} receive a message when \texttt{Pid1} dies. Repeated calls
to \texttt{erlang:monitor(process, Pid)} will create several,
independent monitors and each one will be sent a \texttt{'DOWN'} message
when \texttt{Pid} terminates.

A monitor can be removed by calling \texttt{erlang:demonitor(Ref)}. It
is possible to create monitors for processes with registered names,
also at other nodes.


\section{Process priorities}
The BIF \texttt{process\_flag(priority, Prio)} defines the priority of
the current process. \texttt{Prio} may have the value \texttt{normal},
which is the default, \texttt{low}, \texttt{high} or \texttt{max}. 

Modifying a process's priority is discouraged and should only be done in 
special circumstances.  A problem that requires changing process priorities
can generally be solved by another approach.


\section{Process dictionary}
\label{processes:dicts}
Each process has its own process dictionary which is a list of
\texttt{\{Key, Value\}} terms.

\begin{center}
\begin{tabular}{|>{\raggedright}p{79pt}|>{\raggedright}p{247pt}|}
\hline
\multicolumn{2}{|p{326pt}|}{Process dictionary BIFs}\tabularnewline
\hline
\texttt{put(Key, Value)} & Saves the \texttt{Value} under the \texttt{Key} or replaces an older value\tabularnewline
\hline
\texttt{get(Key)} & Retrieves the value stored under \texttt{Key} or \texttt{undefined}\tabularnewline
\hline
\texttt{get()} & Returns the entire process dictionary as a list of \texttt{\{Key, Value\}} terms\tabularnewline
\hline
\texttt{get\_keys(Value)} & Returns a list of keys that have the value \texttt{Value}\tabularnewline
\hline
\texttt{erase(Key)} & Deletes \texttt{\{Key, Value\}}, if any, and returns \texttt{Key}\tabularnewline
\hline
\texttt{erase()} & Returns the entire process dictionary and deletes it\tabularnewline
\hline
\end{tabular}
\end{center}

Process dictionaries could be used to keep global variables within an application,
but the extensive use of them for this is usually regarded as poor programming style.


\chapter{Обработка ошибок}
\label{errorhandling}

Эта глава описывает обработку ошибок внутри процесса. Такие ошибки известны ещё, 
как \textbf{исключения}.


\section{Классы исключений и причины ошибок}

\begin{center}
\begin{tabular}{|>{\raggedright}p{55pt}|>{\raggedright}p{350pt}|}
\hline
\multicolumn{2}{|p{326pt}|}{Классы исключений}\tabularnewline
\hline
\texttt{error} &
Ошибка времени выполнения, например, при применении оператора к недопустимым типам
аргументов. Ошибки времени выполнения могут быть вызваны программистом с помощью
встроенной функции \texttt{erlang:error(Причина)} или 
\texttt{erlang:error(Причина, Аргументы)} \tabularnewline
\hline
\texttt{exit}  &
Процесс вызвал \texttt{exit(Причина)}, см. раздел \ref{processes:termination}
о завершении процессов \tabularnewline
\hline
\texttt{throw}  & 
Процесс вызвал \texttt{throw(Выражение)}, см. раздел 
\ref{errorhandling:catchthrow} о бросках исключений\tabularnewline
\hline
\end{tabular}
\end{center}

Появление исключения приводит к аварийной остановке процесса, то есть его 
исполнение останавливается и он и его данные удаляются из системы. Также это 
действие называется \emph{уничтожением} (termination). После этого сигналы выхода
посылаются всем связанным процессам. Исключение состоит из класса, причины выхода
и копии стека вызовов. Стек вызовов можно сформировать и получить в удобном виде с 
помощью функции \texttt{erlang:get\_stacktrace/0}.

Ошибки времени исполнения и другие исключения могут не приводить к смерти 
процесса, если использовать выражения \texttt{try} и \texttt{catch}.

Для исключений, имеющих класс \texttt{error}, например для обычных ошибок времени 
выполнения, \textbf{причиной выхода} будет кортеж \texttt{\{Причина, Стек\}} где
\texttt{Причина} --- это терм, указывающий более точно на тип ошибки.

\begin{center}
\begin{tabular}{|>{\raggedright}p{110pt}|>{\raggedright}p{320pt}|}
\hline
\multicolumn{2}{|p{326pt}|}{Причины выхода}\tabularnewline
\hline
\texttt{badarg}  & 
Передан аргумент недопустимого типа. \tabularnewline
\hline
\texttt{badarith}  &
Аргумент в арифметическом выражении имеет недопустимый тип. \tabularnewline
\hline
\texttt{\{badmatch, Значение\}}  & 
Вычисление сопоставления с образцом прошло неудачно. \texttt{Значение} не совпало
с образцом.
\tabularnewline
\hline
\texttt{function\_clause}  & 
Не найдено ни одного подходящего уравнения функции по образцам аргументов и
охранных выражений при вычислении вызова функции. \tabularnewline
\hline
\texttt{\{case\_clause, Значение\}}  & 
При вычислении выражения \texttt{case} \texttt{Значение} не совпало ни с чем. \tabularnewline
\hline
\texttt{if\_clause}  & 
Ни одна из веток выражения \texttt{if} не оказалась равна \texttt{true}.
\tabularnewline
\hline
\texttt{\{try\_clause, Value\}}  &
При вычислении секции выражения \texttt{try} со \texttt{Значением} не совпала ни
одна из веток. \tabularnewline
\hline
\texttt{undef}  & 
Функция не была найдена при попытке исполнить вызов функции \tabularnewline
\hline
\texttt{\{badfun, Fun\}}  & 
Что-то не так с анонимной функцией \texttt{Fun} \tabularnewline
\hline
\texttt{\{badarity, Fun\}}  & 
Функция вызвана с неверным количеством аргументов. Значение \texttt{Fun} описывает
и её и переданные аргументы \tabularnewline
\hline
\texttt{timeout\_value}  & 
Значение таймаута в \texttt{receive}$...$\texttt{after} было вычислено и оказалось
не целым 32-битным числом и не атомом \texttt{infinity} \tabularnewline
\hline
\texttt{noproc}  & 
Попытка создать связь с несуществующим процессом \tabularnewline
\hline
\texttt{\{nocatch, Значение\}}  & 
Попытка выполнить \texttt{throw} за пределами кода, защищённого оператором
\texttt{catch}. \texttt{Значение} --- терм, который был брошен \tabularnewline
\hline
\texttt{system\_limit}  & Сработало одно из системных ограничений, заданных 
реализацией виртуальной машины или операционной системы \tabularnewline
\hline
\end{tabular}
\end{center}

\texttt{Стек} --- это цепочка вызовов функций, которые исполнялись в тот момент,
когда произошла ошибка, даётся в виде списка кортежей \texttt{\{Модуль, Имя, 
Арность\}}, где первым идёт самый недавний вызов. Иногда самый последний вызов 
вместо \texttt{Арности} будет содержать \texttt{Аргументы}: \texttt{\{Модуль, 
Имя, Аргументы\}}.


\section{Catch и throw}
\label{errorhandling:catchthrow}
\begin{erlangru}
catch Выражение
\end{erlangru}

Такая запись возвращает результат вычисления \texttt{Выражения} если во время 
вычисления не возникло исключение. В случае исключения возвращаемое значение будет
кортежем с информацией о нём.

\begin{erlangru}
{'EXIT', {Причина, Стек}}
\end{erlangru}

Такое исключение считается \emph{пойманным}. Непойманное исключение приведёт к 
уничтожению процесса. Если исключение было вызвано вызовом функции 
\texttt{exit(Терм)} то будет возвращён кортеж с причиной в виде 
\texttt{\{'EXIT',Терм\}}. Если исключение возникло при вызове \texttt{throw(Терм)},
тогда будет возвращено значение \texttt{Терма}.

\texttt{catch 1+2} \resultingin \texttt{3}\\
\texttt{catch 1+a } \resultingin \texttt{\{'EXIT',\{badarith,[...]\}\}}

Оператор \texttt{catch} имеет низкий приоритет и выражения, использующие его часто
требуют заключения в блок \texttt{begin..end} или круглые скобки.

\texttt{A = (catch 1+2)} \resultingin \texttt{3}

Встроенная функция \texttt{throw(Выражение)} используется для \emph{нелокального}
выхода из функций. Её можно вызывать только под защитой \texttt{catch}, что 
возвратит результат вычисления \texttt{Выражения}.

\texttt{catch begin 1,2,3,throw(four),5,6 end} \resultingin \texttt{four}

Если \texttt{throw/1} вычисляется за пределами оператора \texttt{catch}, то 
возникнет ошибка времени выполнения \texttt{nocatch}.

Оператор \texttt{catch} не спасёт процесс от завершения по сигналу выхода из 
другого связанного с ним процесса (если только не был включен режим перехвата 
сигналов выхода, trap\_exit).


\section{Try}
\label{errorhandling:try}

Выражение \texttt{try} способно различить различные классы исключений.  Следующий
пример эмулирует поведение описанного чуть выше \texttt{catch Выражение}:

\begin{erlang}
try Expr
catch
    throw:Term -> Term;
    exit:Reason -> {'EXIT', Reason};
    error:Reason -> {'EXIT',{Reason, erlang:get_stacktrace()}}
end
\end{erlang}

Полное описание \texttt{try} следующее:

\begin{erlangru}
try Выражение [of
    Образец1 [when ОхранныеВыражения1] -> Тело1;
    ...;
    ОбразецN [when ОхранныеВыраженияN] -> ТелоN]
[catch
    [КлассОшибки1:]ОбразецИсключения1 [when ОхранаИсключения1] ->
        ТелоИсключения1;
    ...;
    [КлассОшибкиN:]ОбразецИсключенияN [when ОхранаИсключенияN] ->
        ТелоИсключенияN]
[after ТелоПосле]
end
\end{erlangru}

В наличии должны обязательно быть как минимум одна строка \texttt{catch} или одно
предложение \texttt{after}.  Может дополнительно использоваться оператор
\texttt{of} после \texttt{Выражения}, который добавляет вычисление \texttt{case}
к значению \texttt{Выражения}.

\texttt{try} возвратит значение \texttt{Выражения}, если только не произойдёт 
исключение во время его вычисления.  Тогда исключение \emph{ловится} и ряд 
\texttt{ОбразцовИсключения} с подходящим \texttt{КлассомОшибки} сопоставляются
один за другим с пойманным исключением.  Если \texttt{КлассОшибки} не указан, то
подразумевается \texttt{throw}. Если сопоставление удачно проходит и необязательные
выражения \texttt{ОхраныИсключения} тоже равны \texttt{true}, то вычисляется 
соответствующее \texttt{ТелоИсключения} и его результат становится возвращаемым
значением.

Если не найдено подходящего \texttt{ОбразцаИсключения} с таким же 
\texttt{КлассомОшибки} и выражениями \texttt{ОхраныИсключения}, равными 
\texttt{true}, то исключение передаётся дальше, как если бы начальное 
\texttt{Выражение} не было заключено в \texttt{try}.  Исключение, происходящее
во время вычисления \texttt{ТелаИсключения} не будет поймано.

Если ни один из \texttt{Образцов} не совпал, то произойдёт ошибка времени 
исполнения \texttt{try\_clause}.

Если определено, то \texttt{ТелоПосле} всегда вычисляется \textbf{после} всех 
операций в\linebreak
\texttt{try..catch}, независимо от возникновения ошибки. Его 
возвращаемое значение игнорируется и не влияет на значение всего оператора 
\texttt{try} (как если бы \texttt{after} не было). \texttt{ТелоПосле} вычисляется
даже если исключение произошло в \texttt{Теле} или \texttt{ТелеИсключения}, в этом
случае исключение передаётся выше по коду.

%% The \texttt{AfterBody} is evaluated after either \texttt{Body} or
%% \texttt{ExceptionBody} no matter which one. The evaluated value of the
%% \texttt{AfterBody} is lost; the return value of the try expression is
%% the same with an after section as without. Even if an exception occurs
%% during evaluation of \texttt{Body} or \texttt{ExceptionBody}, the
%% \texttt{AfterBody} is evaluated. In this case the exception is caught
%% and passed on after the \texttt{AfterBody} has been evaluated, so the
%% exception from the try expression is the same with an after section as
%% without.

Исключение, которое происходит во время вычисления \texttt{ТелаПосле} не ловится,
то есть если \texttt{ТелоПосле} вычислилось по причине исключения в 
\texttt{Выражении}, \texttt{Теле} или \texttt{ТелеИсключения}, то оригинальное
исключение будет потеряно и вместо него полёт вверх по стеку вызовов продолжит уже
новое исключение.


\chapter{Распределённый Erlang}
\label{distribution}

\textbf{Распределённая Erlang-система} состоит из некоторого количества систем 
времени исполнения Erlang, которые сообщаются друг с другом.  Каждая такая система
называется \textbf{узлом} (node). Узлы могут находиться на одной физической машине
или на разных и быть соединёнными посредством сети.  Стандартный механизм 
распределения реализован на основе TCP/IP сокетов, но могут быть реализованы и 
другие механизмы.

Передача сообщений между процессами на разных узлах, так же как и связи между
процессами и мониторы, прозрачно реализована используя идентификаторы процессов 
(pid).  Однако, зарегистрированные имена локальны для каждого из узлов.  
На зарегистрированный процесс на конкретном узле можно ссылаться с помощью кортежа
\texttt{\{Имя,Узел\}}.

Служба отображения портов Erlang (Erlang Port Mapper Daemon, или \textbf{epmd}) 
автоматически стартует на каждом компьютере, где имеется запущенный узел Erlang.
Он отвечает за отображение имён узлов в сетевые адреса компьютеров.


\section{Узлы}

\textbf{Узел} --- это исполняемая в данный момент Erlang-система, которой было 
назначено имя, используя параметр командной строки \texttt{-name} (длинное имя)
или \texttt{-sname} (короткое имя).

Формат имени узла --- атом вида \texttt{Имя@Адрес}, где \texttt{Имя} задаётся 
пользователем, запустившим изул, а \texttt{Адрес} --- полное имя сервера, если
были включены длинные имена, или первая часть имени сервера (если были 
использованы короткие имена). Функция \texttt{node()} возвращает имя узла.  Узлы,
использующие длинные имена не могут связываться с узлами, использующими короткие
имена.


\section{Соединение между узлами}

Узлы распределённой Erlang-системы полностью соединены (каждый с каждым). Первый
раз, когда используется новое имя узла, производится попытка подключения к этому
узлу. Если узел \texttt{A} подключается к узлу \texttt{B}, и узел \texttt{B} 
имел открытое подключение к узлу \texttt{C}, то узел \texttt{А} тоже попытается
подключиться к узлу \texttt{C}.  Эта возможность может быть отключена используя
параметр командной строки:

\qquad\texttt{-connect\_all false}

Если узел прекращает работу или теряет сеть, все подключения к нему удаляются.
Встроенная функция:

\begin{erlangru}
erlang:disconnect(Узел)
\end{erlangru}

отключает заданный \texttt{Узел}. Встроенная функция \texttt{nodes()} вернёт 
список подключенных в данный момент (видимых) узлов.


\section{Скрытые узлы}

Иногда полезно подключиться к нужному узлу, не инициируя веер подключений ко всем
остальным узлам.  Для этой цели можно использовать \textbf{скрытый узел}. 
Скрытый узел это узел, запущенный с параметром командной строки \texttt{-hidden}. 
Подключения между скрытыми узлами и другими узлами должны устанавливаться вручную 
и явно.  Скрытые узлы не видны в списке узлов, возвращаемом функцией 
\texttt{nodes()}.  Вместо этого следует использовать \texttt{nodes(hidden)} или 
\texttt{nodes(connected)}.  Скрытый узел не будет включён в набор узлов, за 
которыми следит модуль \texttt{global}.

\textbf{Узел на языке С} это С-программа, написанная, чтобы действовать и 
выглядеть, как скрытый узел в распределённой Erlang-системе.  Библиотека 
\texttt{erl\_interface} содержит необходимые для этого функции.


\section{Секретный куки (cookie)}

Каждый узел имеет свой собственный \textbf{магический куки} (cookie), который 
является атомом. Сервер сетевой аутентикации Erlang (под названием \texttt{auth})
читает содержимое куки из файла \texttt{\$HOME/.erlang.cookie}.  Если файл не
существовал, он будет создан и в него будет записана случайная строка.

% FRMB CHECK: the implication here is that "erlang:set_cookie(node(), Cookie)"
% sets *this* node's cookie, as well as the cookie that will be used to connect
% to other nodes, if not explicitly set otherwise.

Права доступа к файлу должны быть установлены в восьмеричное 0400 (только для
чтения владельцем).  Куки локального узла также можно установить с помощью 
встроенной функции \texttt{erlang:set\_cookie(node(), Куки)}.

Текущему узлу позволяется подключаться к другому узлу \texttt{Узел2}, если он
знает значение его куки.  Если оно отличается от куки текущего узла (чей куки 
будет использован по умолчанию), то его надо явно установить с помощью встроенной 
функции \texttt{erlang:set\_cookie(Узел2, Куки2)}.


\section{Встроенные функции для распределения}

\begin{center}
\begin{tabular}{|>{\raggedright}p{150pt}|>{\raggedright}p{280pt}|}
\hline
\multicolumn{2}{|p{326pt}|}{Встроенные функции для распределения}\tabularnewline
\hline
\texttt{node()}  & 
Возвращает имя текущего узла. Позволяется использовать в охранных выражениях
\tabularnewline
\hline
\texttt{is\_alive()}  & 
Возвращает \texttt{true} если система является узлом и может подключаться к другим 
узлам, иначе \texttt{false} \tabularnewline
\hline
\texttt{erlang:get\_cookie()}  & 
Возвращает магический куки текущего узла \tabularnewline
\hline
\texttt{set\_cookie(\\
	\qquad{}Узел, Куки)} & 
Устанавливает магический \texttt{Куки}, который будет использован при подключении 
к \texttt{Узлу}. Если \texttt{Узел} --- текущий узел, то \texttt{Куки} будет 
использован для всех подключений к новым узлам \tabularnewline
\hline
\texttt{nodes()}  & 
Возвращает список всех видимых узлов, к которым подключен текущий \tabularnewline
\hline
\texttt{nodes(connected)\\
	nodes(hidden)}  & 
Возвращает список не только видимых, но и скрытых и ранее известных узлов, и т.д. 
\tabularnewline
\hline
\texttt{monitor\_node(Узел,}\\
\texttt{\qquad{}true\textbar{}false)}  & 
Отслеживает статус \texttt{Узла}. Сообщение \texttt{\{nodedown, Узел\}} будет
прислано процессу, если подключение к узлу потеряно \tabularnewline
\hline
\texttt{node(Pid\textbar{}Ref\textbar{}Port)}  & 
Возвращает имя узла, на котором зарегистрирован аргумент \tabularnewline
\hline
\texttt{erlang:disconnect\_node\\
	\qquad(Узел)}  & 
Принудительно отключает \texttt{Узел} от кластера \tabularnewline
\hline
\texttt{spawn[\_link\textbar{}\_opt](}\\
\texttt{Узел, Модуль, Функция, 
	Аргументы)}  & Создаёт процесс на другом (удалённом) узле \tabularnewline
\hline
\texttt{spawn[\_link\textbar{}\_opt](\\
	Узел, Функция)}  & 
Создаёт процесс на удалённом узле \tabularnewline
\hline
\end{tabular}
\end{center}


\section{Параметры командной строки}

\begin{center}
\begin{tabular}{|>{\raggedright}p{120pt}|>{\raggedright}p{310pt}|}
\hline
\multicolumn{2}{|p{430pt}|}{Параметры командной строки для распределённого  
	Erlang}\tabularnewline
\hline
\texttt{-connect\_all false}  & 
Подключение новых узлов только вручную и явно перечисляется каждый узел 
\tabularnewline
\hline
\texttt{-hidden}  & 
Стартует узел как скрытый \tabularnewline
\hline
\texttt{-name Имя}  & 
Превращает систему Erlang в узел кластера, используя длинные имена узлов 
\tabularnewline
\hline
\texttt{-setcookie Куки}  & Аналогично вызову \linebreak{}
\texttt{erlang:set\_cookie(node(), Куки))}\tabularnewline
\hline
\texttt{-sname Имя}  & 
Превращает систему Erlang в узел кластера, используя короткие имена узлов 
\tabularnewline
\hline
\end{tabular}
\end{center}


\section{Модули с поддержкой распределённых систем}

Есть несколько доступных модулей, которые пригодятся при программировании 
распределённых систем:

\begin{center}
\begin{tabular}{|>{\raggedright}p{93pt}|>{\raggedright}p{233pt}|}
\hline
\multicolumn{2}{|p{326pt}|}{Модули с поддержкой распределённых 
	систем}\tabularnewline
\hline
\texttt{global}  & 
Глобальное средство регистрации имён \tabularnewline
\hline
\texttt{global\_group}  & 
Соединение узлов в глобальные группы регистрации имён \tabularnewline
\hline
\texttt{net\_adm}  & 
Различные функции для управления сетью в Erlang-системе \tabularnewline
\hline
\texttt{net\_kernel}  & 
Ядро работы с сетью \tabularnewline
\hline
\multicolumn{2}{|p{326pt}|}{Модули стандартной библиотеки, полезные для 
	разработки распределённых систем}\tabularnewline
\hline
\texttt{slave}  & Запуск и управление ведомыми узлами \tabularnewline
\hline
\end{tabular}
\end{center}

\chapter{Порты и драйверы портов}
\label{ports}

\textbf{Порты} предоставляют байто-ориентированный интерфейс к внешним 
программам и связывается с процессами Erlang посылая и принимая сообщения в
виде списков байтов. Процесс Erlang, который создаёт порт, называется
\textbf{владельцем порта} или \textbf{подключенным к порту процессом}.  Все
коммуникации в и из порта должны пройти через владельца порта.  Если владелец
порта завершает работу, порт тоже закроется (а также и внешняя программа,
подключенная к порту, если она была правильно написана и среагирует на закрытие 
ввода/вывода).

Внешняя программа является другим процессом операционной системы. По умолчанию, 
она должна считывать данные из стандартного входа (файловый дескриптор 0) и
отвечать на стандартный вывод (файловый дескриптор 1).  Внешняя программа
должна завершать свою работу когда порт закрывается (ввод/вывод закрывается).


\section{Драйверы портов}

Драйверы портов обычно пишутся на языке С и динамически подключаются к системе
исполнения Erlang. Встроенный драйвер ведёт себя как порт и называется 
\textbf{драйвером порта}.  Однако, ошибка в драйвере порта может привести к 
нестабильности во всей системе Erlang, утечкам памяти, зависаниям и краху 
системы.


\section{Встроенные функции для портов}

\begin{center}
\begin{tabular}{|>{\raggedright}p{140pt}|>{\raggedright}p{300pt}|}
\hline
\multicolumn{2}{|p{440pt}|}{Функция для создания порта} \tabularnewline
\hline
\texttt{open\_port(ИмяПорта, НастройкиПорта)} & 
Возвращает \textbf{идентификатор порта} \texttt{Порт}, как результат открытия 
нового Erlang-порта.  Сообщения могут быть отправлены в и получены через 
идентификатор порта, так же как это можно делать с идентификаторами процессов.
Идентификаторы портов могут быть связаны с процессами, или зарегистрированы под
каким-либо именем с помощью \texttt{link/1} и \texttt{register/2}. \tabularnewline
\hline
\end{tabular}
\end{center}

\texttt{ИмяПорта} обычно является кортежем вида \texttt{\{spawn,Команда\}}, где
строка \texttt{Команда} является именем внешней программы.  Внешняя программа 
выполняется за пределами Erlang-системы, если только не найден драйвер порта с
именем \texttt{Команда}.  Если драйвер найден, он будет активирован вместо 
команды.

\texttt{НастройкиПорта} --- это список настроек (опций) для порта. Список обычно
содержит как минимум один кортеж \texttt{\{packet,N\}}, указывающий, что данные,
пересылаемые между портом и внешней программой, предваряются N-байтовым 
индикатором длины.  Разрешённые значения для \texttt{N} --- 1, 2 или 4.  Если 
двоичные данные должны использоваться вместо списков байтов, то должна быть 
включена опция \texttt{binary}.

Владелец порта Pid связывается с \texttt{Портом} с помощью отправки и получения
Erlang-сообщений.  (Любой процесс может послать сообщение в порт, но ответы от 
порта всегда будут отправлены только владельцу порта).

\begin{center}
\begin{tabular}{|>{\raggedright}p{140pt}|>{\raggedright}p{300pt}|}
\hline
\multicolumn{2}{|p{440pt}|}{Сообщения, отсылаемые в порт}\tabularnewline
\hline
\texttt{\{Pid, \{command, Данные\}\}}  & 
Посылает \texttt{Данные} в порт. \tabularnewline
\hline
\texttt{\{Pid, close\}}  & 
Закрывает порт. Если порт был открыт, он отвечает сообщением 
\texttt{\{Порт, closed\}}, когда все буферы были сброшены и порт закрылся.
\tabularnewline
\hline
\texttt{\{Pid, \{connect, НовыйPid\}\}}  & 
Устанавливает владельца \texttt{Порта} равным \texttt{НовомуPid}. Если порт был
открыт, он отвечает сообщением \texttt{\{Порт, connected\}} старому владельцу. 
Заметьте, что старый владелец порта остаётся связанным с портом, тогда как новый
--- нет. \tabularnewline
\hline
\end{tabular}
\end{center}

Данные должны быть списком ввода-вывода. \textbf{Список ввода-вывода} (iolist) 
--- это либо двоичные данные, либо смешанный (возможно вложенный) список
двоичных данных и целых чисел в диапазоне от 0 до 255.

\begin{center}
\begin{tabular}{|>{\raggedright}p{140pt}|>{\raggedright}p{300pt}|}
\hline
\multicolumn{2}{|p{440pt}|}{Сообщения, получаемые из порта}\tabularnewline
\hline
\texttt{\{Порт, \{data, Данные\}\}}  & 
Данные получены от внешней программы \tabularnewline
\hline
\texttt{\{Порт, closed\}}  & 
Ответ на команду \texttt{Порт ! \{Pid,close\}} \tabularnewline
\hline
\texttt{\{Порт, connected\}}  & 
Ответ на команду \texttt{Порт ! \{Pid,\{connect, NewPid\}\}} \tabularnewline
\hline
\texttt{\{'EXIT', Порт, Причина\}}  &
Присылается, если порт был отключен по какой-либо причине. \tabularnewline
\hline
\end{tabular}
\end{center}

Вместо того, чтобы отправлять и получать сообщения, имеется ряд встроенных 
функций, которые можно использовать.  Они могут быть вызваны любым процессом, а
не только владельцем порта.

\begin{center}
\begin{tabular}{|>{\raggedright}p{140pt}|>{\raggedright}p{300pt}|}
\hline
\multicolumn{2}{|p{440pt}|}{Встроенные функции для работы с портами} 
	\tabularnewline
\hline
\texttt{port\_command(Порт, Данные)}  & 
Отправляет \texttt{Данные} в \texttt{Порт} \tabularnewline
\hline
\texttt{port\_close(Порт)}  & 
Закрывает \texttt{Порт} \tabularnewline
\hline
\texttt{port\_connect(Порт, НовыйPid)}  & 
Устанавливает владельца \texttt{Порта} равным \texttt{НовомуPid}. Старый 
владелец остаётся связанным с портом и должен сам вызвать \texttt{unlink(Порт)}
если связь не требуется. \tabularnewline
\hline
\texttt{erlang:port\_info(\\
Порт, Элемент)}  & 
Возвращает информацию о \texttt{Порте} с ключом \texttt{Элемент} \tabularnewline
\hline
\texttt{erlang:ports()}  & 
Возвращает список всех открытых портов на текущем узле \tabularnewline
\hline
\end{tabular}
\end{center}

Есть несколько дополнительных встроенных функций, которые применимы только к 
драйверам портов: это \texttt{port\_control/3} и \texttt{erlang:port\_call/3}.
\chapter{Code loading}
\label{code}

Erlang supports code updating in a running system. Code replacement is
performed at module level.

The code of a module can exist in two versions in a system:
\textbf{current} and \textbf{old}. When a module is
loaded into the system for the first time, the code becomes
\textit{current}. If a new instance of the module is loaded, the code
of the previous instance becomes \textit{old} and the new instance
becomes \textit{current}. Normally a module is automatically loaded
the first time a function in it is called. If the module is already
loaded then it must explicitly be loaded again to a new version.

Both old and current code are valid, and may be used
concurrently. Fully qualified function calls will always refer to the
current code. However, the old code may still be run by other
processes.

If a third instance of the module is loaded, the code server will
remove (\textit{purge}) the old code and any processes lingering in it
are terminated. Then the third instance becomes \textit{current} and
the previously current code becomes \textit{old}.

To change from old code to current code, a process must make a fully
qualified function call.

\begin{erlang}
-module(mod).
-export([loop/0]).

loop() ->
    receive
        code_switch ->
            mod:loop();
        Msg ->
            ...
            loop()
    end.
\end{erlang}

To make the process change code, send the message
\texttt{code\_switch} to it. The process then will make a fully
qualified call to \texttt{mod:loop()} and change to the current
code. Note that \texttt{mod:loop/0} must be exported.
\chapter{Макросы}
\label{macros}

\section{Определение и использование макросов}

\begin{erlangru}
-define(Константа, Замена).
-define(Функция(Переменная1, ..., ПеременнаяN), Замена).
\end{erlangru}

\textbf{Макрос} должен быть определён перед тем, как он используется, но 
определение макроса можно поместить где угодно среди атрибутов и определений 
функций в модуле.  Если макрос используется в нескольких модулях, рекомендуется 
поместить его определение во включаемый файл.  Макрос используется так:

\begin{erlangru}
?Константа
?Функция(Переменная1,...,ПеременнаяN)
\end{erlangru}

Макросы разворачиваются во время компиляции на самом раннем этапе.
Ссылка на макрос \texttt{?Константа} будет заменена на текст \texttt{Замена} 
так:

\begin{erlang}
-define(TIMEOUT, 200).
...
call(Request) ->
    server:call(refserver, Request, ?TIMEOUT).
\end{erlang}

разворачивается перед компиляцией в:

\begin{erlang}
call(Request) ->
    server:call(refserver, Request, 200).
\end{erlang}

Ссылка на макрос \texttt{?Функция(Аргумент1, ..., АргументN)} будет заменена на
\texttt{Замену}, где все вхождения переменной \texttt{ПеременнаяX} из 
определения макроса будут заменены на соответствующий \texttt{АргументX}.

\begin{erlang}
-define(MACRO1(X, Y), {a, X, b, Y}).
...
bar(X) ->
    ?MACRO1(a, b),
    ?MACRO1(X, 123).
\end{erlang}

будет развёрнуто в:

\begin{erlang}
bar(X) ->
    {a, a, b, b},
    {a, X, b, 123}.
\end{erlang}

Для просмотра результата разворачивания макросов, модуль можно скомпилировать с
параметром \texttt{'P'} таким образом:

\begin{erlangru}
compile:file(Файл, ['P']).
\end{erlangru}

Это производит распечатку разобранного кода после применения к нему 
предварительной обработки и трансформаций разбора (parse transform), в файле с 
именем \texttt{Файл.P}.


\section{Предопределённые макросы}

\begin{center}
\begin{tabular}{|>{\raggedright}p{110pt}|>{\raggedright}p{250pt}|}
\hline
\multicolumn{2}{|p{330pt}|}{Предопределённые макросы}\tabularnewline
\hline
\texttt{?MODULE} & 
Атом, имя текущего модуля \tabularnewline
\hline
\texttt{?MODULE\_STRING} & 
Строка, имя текущего модуля \tabularnewline
\hline
\texttt{?FILE} & 
Имя исходного файла текущего модуля \tabularnewline
\hline
\texttt{?LINE} & 
Текущий номер строки \tabularnewline
\hline
\texttt{?MACHINE} & 
Имя виртуальной машины, 'BEAM'\tabularnewline
\hline
\end{tabular}
\end{center}


\section{Управление исполнением макросов}

\begin{erlang}
-undef(Macro).      % Это отменяет определённый ранее макрос

-ifdef(Macro).
    %% Строки, которые будут скомпилированы, если Macro существует
-else.
    %% Иначе будут скомпилированы эти строки
-endif.
\end{erlang}

\texttt{ifndef(Macro)} можно использовать вместо \texttt{ifdef} и имеет обратный
смысл.

\begin{erlang}
-ifdef(debug).
-define(LOG(X), io:format("{~p,~p}:~p~n",[?MODULE,?LINE,X])).
-else.
-define(LOG(X), true).
-endif.
\end{erlang}

Если макрос \texttt{debug} определён в то время, когда идёт компиляция модуля, 
то макрос \texttt{?LOG(Arg)} развернётся в вызов печати текста
\texttt{io:format/2} и обеспечит пользователя отладочным выводом в консоль.


\section{Превращение аргументов макроса в строку}

Запись вида \texttt{??Аргумент}, где \texttt{Аргумент} --- это параметр, передаваемый в макрос, развернётся в представление аргумента в строковом виде.

\begin{erlang}
-define(TESTCALL(Call), io:format("Call ~s: ~w~n", [??Call, Call])).

?TESTCALL(myfunction(1,2)),
?TESTCALL(you:function(2,1)),
\end{erlang}

\pagebreak
Разворачивается в:

\begin{erlang}
io:format("Call ~s: ~w~n",
          ["myfunction(1,2)", m:myfunction(1,2)]),
io:format("Call ~s: ~w~n",
          ["you:function(2,1)", you:function(2,1)]),
\end{erlang}

Таким образом, получается отладочный вывод как вызванной функции так и 
результата.

\chapter{Дальнейшие материалы для чтения}

Следующие вебсайты предлагают более подробное объяснение тем и концепций, кратко
описанных в данном документе:

\section{Русскоязычные ресурсы}

\begin{itemize}
\item Русские новости из мира Erlang \url{http://erlanger.ru}
\item Группа \texttt{erlang-russian} и \texttt{erlang-in-ukraine} на сервере 
Google Groups.
\item Книга Ф. Чезарини "<Программирование в Erlang">, русский перевод
\url{http://dmkpress.com/catalog/computer/programming/functional/978-5-94074-617-1/}
\end{itemize}

Также обратите внимание на русскоязычный канал \texttt{erlang} на сервере 
\url{http://jabber.ru}

\section{Англоязычные ресурсы}

\begin{itemize}
\item Официальная документация по Erlang: \url{http://www.erlang.org/doc/}
\item Learn You Some Erlang for Great Good: \url{http://learnyousomeerlang.com/}
\item Раздел с уроками на Erlang Central: 
	\url{https://erlangcentral.org/wiki/index.php?title=Category:HowTo}
\end{itemize}

Ещё вопросы? Список рассылки \texttt{erlang-questions} (адрес для подписки
\url{http://erlang.org/mailman/listinfo/erlang-questions}) является хорошим местом
для неспешных общих дискуссий об Erlang/OTP, языке, реализации, использовании и
вопросы новичков. Если вы не планируете писать в рассылку, её можно прочесть без 
подписки на сервере Google Groups, группа \texttt{erlang-programming}.

Также обратите внимание на англоязычный IRC канал \texttt{\#erlang} в сети 
Freenode.



\end{document}
