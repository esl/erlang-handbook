\chapter{Error handling}
\label{errorhandling}

This chapter deals with error handling within a process. Such errors
are known as \textbf{exceptions}.


\section{Exception classes and error reasons}

\begin{center}
\begin{tabular}{|>{\raggedright}p{49pt}|>{\raggedright}p{277pt}|}
\hline
\multicolumn{2}{|p{326pt}|}{Exception classes}\tabularnewline
\hline
\texttt{error} & Run-time error for example when applying an operator to the wrong types of arguments. Run-time errors can be emulated by calling the BIFs \texttt{erlang:error(Reason)} or \texttt{erlang:error(Reason, Args)} \tabularnewline
\hline
\texttt{exit}  & The process calls \texttt{exit(Reason)}, see section \ref{processes:termination}\tabularnewline
\hline
\texttt{throw}  & The process calls \texttt{throw(Expr)}, see section \ref{errorhandling:catchthrow}\tabularnewline
\hline
\end{tabular}
\end{center}

An exception will cause the process to crash, i.e. its execution is
stopped and it is removed from the system. It is also said to
\textit{terminate}. Then exit signals will be sent to any linked
processes.\textit{ }An exception consists of its class, an exit reason
and a stack. The stack trace can be retrieved using the BIF
\texttt{erlang:get\_stacktrace/0}.

Run-time errors and other exceptions can be prevented from causing the
process to terminate by using the expressions catch or try.

For exceptions of class error, for example normal run-time errors, the
\textbf{exit reason} is a tuple \texttt{\{Reason, Stack\}} where
Reason is a term indicating which type of error.

\begin{center}
\begin{tabular}{|>{\raggedright}p{100pt}|>{\raggedright}p{226pt}|}
\hline
\multicolumn{2}{|p{326pt}|}{Exit reasons}\tabularnewline
\hline
\texttt{badarg}  & Argument is of wrong type. \tabularnewline
\hline
\texttt{badarith}  & Argument is of wrong type in an arithmetic expression. \tabularnewline
\hline
\texttt{\{badmatch, Value\}}  & Evaluation of a match expression failed. Value did not match.
\tabularnewline
\hline
\texttt{function\_clause}  & No matching function clause is found when evaluating a function
call. \tabularnewline
\hline
\texttt{\{case\_clause, Value\}}  & No matching branch is found when evaluating a case expression.
Value did not match. \tabularnewline
\hline
\texttt{if\_clause}  & No true branch is found when evaluating an if expression. \tabularnewline
\hline
\texttt{\{try\_clause, Value\}}  & No matching branch is found when evaluating the of section
of a \texttt{try} expression. Value did not match. \tabularnewline
\hline
\texttt{undef}  & The function cannot be found when evaluating a function call\tabularnewline
\hline
\texttt{\{badfun, Fun\}}  & There is something wrong with a Fun\tabularnewline
\hline
\texttt{\{badarity, Fun\}}  & A fun is applied to the wrong number of arguments. Fun describes
it and the arguments\tabularnewline
\hline
\texttt{timeout\_value}  & The timeout value in a receive..after expression is evaluated
to something else than an integer or infinity\tabularnewline
\hline
\texttt{noproc}  & Trying to link to a non-existing process\tabularnewline
\hline
\texttt{\{nocatch, Value\}}  & Trying to evaluate a throw outside a catch. Value is the thrown
term\tabularnewline
\hline
\texttt{system\_limit}  & A system limit has been reached\tabularnewline
\hline
\end{tabular}
\end{center}

Stack is the stack of function calls being evaluated when the error
occurred, given as a list of tuples \texttt{\{Module, Name, Arity\}}
with the most recent function call first.  The most recent function
call tuple may in some cases be \texttt{\{Module, Name, Args\}}


\section{Catch and throw}
\label{errorhandling:catchthrow}
\begin{erlang}
catch Expr
\end{erlang}

This returns the value of \texttt{Expr} unless an exception occurs
during its evaluation. Then the return value will be a tuple
containing information about the exception.

\begin{erlang}
{'EXIT', {Reason, Stack}}
\end{erlang}

Then the exception is \textit{caught}. Otherwise it would terminate
the process. If the exception is caused by a function call
\texttt{exit(Term)} the tuple \texttt{\{'EXIT',Term\}} is returned. If
the exception is caused by calling \texttt{throw(Term)} then
\texttt{Term} will be returned.

\texttt{catch 1+2} $\Rightarrow$ \texttt{3}\\
\texttt{catch 1+a } $\Rightarrow$ \texttt{\{'EXIT',\{badarith,[...]\}\}}

\texttt{catch} has low precedence and catch subexpressions often need
to be enclosed in a block expression or in parentheses.

\texttt{A = (catch 1+2)} $\Rightarrow$ \texttt{3}

The BIF \texttt{throw(Expr)} is used for \textit{non-local} return
from a function. It must be evaluated within a catch, which returns
the result from evaluating \texttt{Expr}.

\texttt{catch begin 1,2,3,throw(four),5,6 end}  $\Rightarrow$  \texttt{four}

If \texttt{throw/1} is not evaluated within a catch, a
\texttt{nocatch} run-time error will occur.

A catch will not prevent a process from terminating due to an exit
signal from another linked process (unless it has been set to trap
exits).


\section{Try}
\label{errorhandling:try}
The try expression is able to distinguish between different exception
classes. The following example emulates \texttt{catch Expr}:

\begin{erlang}
try Expr
catch
    throw:Term -> Term;
    exit:Reason -> {'EXIT', Reason};
    error:Reason -> {'EXIT',{Reason, erlang:get_stacktrace()}}
end
\end{erlang}

The full description of \texttt{try} is as follows:

\begin{erlang}
try Expr [of
    Pattern1 [when GuardSeq1] -> Body1;
    ...;
    PatternN [when GuardSeqN] -> BodyN]
[catch
    [Class1:]ExceptionPattern1 [when ExceptionGuardSeq1] -> ExceptionBody1;
    ...;
    [ClassN:]ExceptionPatternN [when ExceptionGuardSeqN] -> ExceptionBodyN]
[after AfterBody]
end
\end{erlang}

There has to be at least a \texttt{catch} or an \texttt{after}
clause. There may be an \texttt{of} clause following the \texttt{Expr}
which adds a \texttt{case} expression on the value of \texttt{Expr}.

\texttt{try} returns the value of \texttt{Expr} unless an exception
occurs during its evaluation. Then the exception is \textit{caught}
and the patterns \texttt{ExceptionPattern} with the right exception
Class are sequentially matched against the caught exception. An
omitted Class is shorthand for throw. If a match succeeds and the
optional guard sequence \texttt{ExceptionGuardSeq} is true, the
corresponding \texttt{ExceptionBody} is evaluated to become the return
value.

If there is no matching \texttt{ExceptionPattern} of the right Class
with a true guard sequence, the exception is passed on as if
\texttt{Expr} had not been enclosed in a try expression. An exception
occurring during the evaluation of an \texttt{ExceptionBody} it is not
caught.

If none of the of Patterns match, a \texttt{try\_clause} run-time
error will occur.

If defined then \texttt{AfterBody} is always evaluated \textbf{last}
irrespective of whether and error occurred or not. Its return value is
ignored and the return value of the \texttt{try} is the same as
without an \texttt{after} section. \texttt{AfterBody} is evaluated
even if an exception occurs in \texttt{Body} or
\texttt{ExceptionBody}, in which case the exception is passed on.

%% The \texttt{AfterBody} is evaluated after either \texttt{Body} or
%% \texttt{ExceptionBody} no matter which one. The evaluated value of the
%% \texttt{AfterBody} is lost; the return value of the try expression is
%% the same with an after section as without. Even if an exception occurs
%% during evaluation of \texttt{Body} or \texttt{ExceptionBody}, the
%% \texttt{AfterBody} is evaluated. In this case the exception is caught
%% and passed on after the \texttt{AfterBody} has been evaluated, so the
%% exception from the try expression is the same with an after section as
%% without.

An exception that occurs during the evaluation of \texttt{AfterBody}
itself is not caught, so if the \texttt{AfterBody} is evaluated due to
an exception in \texttt{Expr}, \texttt{Body} or
\texttt{ExceptionBody}, that exception is lost and masked by the new
exception.
