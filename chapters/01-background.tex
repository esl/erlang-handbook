\chapter{Background, or Why Erlang is that it is}
\label{background}

Erlang is a result of a project at Ericsson's Computer Science Lab to
improve the programming of telecoms type of applications. A critical
requirement was that the characteristics of these types applications
had to be supported. These characteristics include:

\begin{itemize}
\item Massive concurrency

\item Fault-tolerance

\item Isolation

\item Dynamic code upgrading at runtime

\item Transactions
\end{itemize}

Throughout the whole of Erlang's history the development process has
been extremely pragmatic. The characteristics and properties of the
types of systems in which we were interested drove Erlang's
development. For example these properties were considered to be so
fundamental that it was decided to build support for them into the
language itself, rather than in libraries. Another example is that,
rather than it being planned, Erlang ``became'' a functional language
as the features of functional languages fit well with the system
properties.
