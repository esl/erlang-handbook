%&pdfLaTeX
% !TEX encoding = UTF-8 Unicode
\documentclass[oneside]{book}

%%%%%%
%% package includes
\usepackage[T1]{fontenc}
\usepackage[utf8]{inputenc}
\usepackage{hyperref}
\usepackage{textcomp}
\usepackage{array}
\usepackage{color}
\usepackage[hmargin=2.5cm,vmargin=2.5cm]{geometry}
\usepackage{minted}
\usepackage{todonotes}
\usepackage{parskip}
\usepackage[Bjornstrup]{fncychap}
\usepackage{listings}

%%%%%%
%% package setup
\hypersetup{
    colorlinks=true,
    linktoc=all,
    linkcolor=red,
}

%%%%%%
%% hacks and macros
\newminted[erlang]{erlang}{xleftmargin=1cm,xrightmargin=1cm,frame=single,framerule=1pt}

%% hack to fix the margin issue in \item s in 1.2.1
\newminted[erlangim]{erlang}{xleftmargin=-1.5cm}

%%% display todo notes or not?
\newcommand{\todonote}[1]{\todo[inline]{\textbf{Note:} #1}}
%\newcommand{\todonote}[1]{}

%% typesetting for the "results in" notation used.
\def\resultingin{$\quad\Rightarrow\quad$}

\begin{document}

%%%%%%
%% Title
\begin{titlepage}
\centering

\vspace*{70pt}
\includegraphics[scale=0.3]{includes/erlang-logo.png}\\[0.8\baselineskip]
{\Huge \sffamily HANDBOOK}\\
\vspace{250pt}
{\LARGE \sffamily Bjarne D\"acker}\par
{\LARGE \sffamily Robert Virding}\par

\end{titlepage}


%%%%%%
%% Info Page
\clearpage
\thispagestyle{empty}
{\Huge Erlang Handbook}\\[0.1\baselineskip]
\hspace{10pt}{\large{by Bjarne Däcker and Robert Virding}}

\vspace{20pt}
{\Large \textbf{Revision:}\\[0.2\baselineskip]
\immediate\write18{git log -n1 | grep 'Date:' | sed 's/Date:   //g' > _revision.tex}
\input{_revision}
\immediate\write18{rm _revision.tex}
}

\vspace{20pt}
{\large Latest version of this handbook can be found at:\\
\url{http://opensource.erlang-solutions.com/erlang-handbook}}

\vfill

\textbf{Editor}\\[0.1\baselineskip]
%\begin{tabular}{@{\hspace{3ex}}p{42em}}
Omer Kilic
%\end{tabular}

\textbf{Contributors}\\[0.1\baselineskip]
%\begin{tabular}{@{\hspace{3ex}}p{42em}}
The list of contributors can be found \href{https://github.com/esl/erlang-handbook/graphs/contributors}{on the project repository}.
%\end{tabular}

\vspace{20pt}
\textbf{Conventions}\\
Syntax specifications are set using \texttt{this monotype font}. Square brackets
([ ]) enclose optional parts. Terms beginning with an uppercase letter like
\textit{Integer} shall then be replaced by some suitable value. Terms beginning
with a lowercase letter like \texttt{end} are reserved words in Erlang. A
vertical bar (\textbar{}) separates alternatives, like Integer \textbar{} Float.

\vspace{20pt}
\textbf{Errata and Improvements}\\
This is a live document so please file corrections and suggestions for
improvement about the content using the issue tracker at
\url{https://github.com/esl/erlang-handbook}. You may also fork this repository
and send a pull request with your suggested fixes and improvements. New
revisions of this document will be published after major corrections.

\vspace{20pt}
\includegraphics[scale=0.7]{includes/cc-by-sa.png}\\ This text is made available
under a Creative Commons Attribution-ShareAlike 3.0 License. You are free to
copy, distribute and transmit it under the license terms defined at
\url{http://creativecommons.org/licenses/by-sa/3.0}

\newpage


%%%%%%
%% TOC
\tableofcontents


%%%%%%
%% Chapters

\chapter{Background, or Why Erlang is that it is}
\label{background}

Erlang is a result of a project at Ericsson's Computer Science Lab to
improve the programming of telecoms type of applications. A critical
requirement was that the characteristics of these types applications
had to be supported. These characteristics include:

\begin{itemize}
\item Massive concurrency

\item Fault-tolerance

\item Isolation

\item Dynamic code upgrading at runtime

\item Transactions
\end{itemize}

Throughout the whole of Erlang's history the development process has
been extremely pragmatic. The characteristics and properties of the
types of systems in which we were interested drove Erlang's
development. For example these properties were considered to be so
fundamental that it was decided to build support for them into the
language itself, rather than in libraries. Another example is that,
rather than it being planned, Erlang ``became'' a functional language
as the features of functional languages fit well with the system
properties.


\chapter{Structure of an Erlang program}

\section{Module syntax}

An Erlang program is made up of \textbf{modules} where each module is
a text file with the extension \textbf{.erl}. For small programs, all
modules typically reside in one directory.
A module consists of module attributes and function definitions.

\begin{erlang}
-module(demo).
-export([double/1]).

double(X) -> times(X, 2).

times(X, N) -> X * N.
\end{erlang}

The above module \texttt{demo} consists of the function \texttt{times/2}
which is local to the module and the function \texttt{double/1} which
is exported and can be called from outside the module.

\texttt{demo:double(10)} \resultingin \texttt{20}\hfill
(the arrow $\Rightarrow$ should be read as ``resulting in'')

\texttt{double/1} means the function ``double'' with \textit{one}
argument. A function \texttt{double/2} taking \textit{two} arguments
is regarded as a different function. The number of arguments is called
the \textbf{arity} of the function.


\section{Module attributes}
A \textbf{module attribute} defines a certain property of a module and
consists of a \textbf{tag} and a \textbf{value}:

\texttt{-Tag(Value).}

\texttt{Tag} must be an atom, while \texttt{Value} must be a literal
term (see chapter \ref{datatypes}). Any module attribute can be specified. The
attributes are stored in the compiled code and can be retrieved by
calling the function \texttt{Module:module\_info(attributes).}

\subsection{Pre-defined module attributes}
Pre-defined module attributes must be placed before any function
declaration.

\begin{itemize}

	\item \begin{erlangim}
	-module(Module).
	\end{erlangim}
	This attribute is mandatory and must be specified first. It
        defines the name of the module. The name Module, an atom (see section \ref{datatypes:atom}),
        should be the same as the filename without the `\texttt{.erl}' extension.

	\item \begin{erlangim}
	-export([Func1/Arity1, ..., FuncN/ArityN]).
	\end{erlangim}
	This attribute specifies which functions in the module that
        can be called from outside the module. Each function name
        \texttt{FuncX} is an atom and \texttt{ArityX} an integer.

	\item \begin{erlangim}
	-import(Module,[Func1/Arity1, ..., FuncN/ArityN]).
	\end{erlangim}
	This attribute indicates a \texttt{Module} from which a list of functions
        are imported.  For example:

	\begin{erlangim}
	-import(demo, [double/1]).
	\end{erlangim}
	This means that it is possible to write \texttt{double(10)} instead of
        the longer \texttt{demo:double(10)} which can be impractical if the
        function is used frequently.

	\item \begin{erlangim}
	-compile(Options).
	\end{erlangim}
	Compiler options.

	\item \begin{erlangim}
	-vsn(Vsn).
	\end{erlangim}
	Module version. If this attribute is not specified, the
        version defaults to the checksum of the module.

	\item \begin{erlangim}
	-behaviour(Behaviour).
	\end{erlangim}
	This attribute either specifies a user defined behaviour or
        one of the OTP standard behaviours \texttt{gen\_server},
        \texttt{gen\_fsm}, \texttt{gen\_event} or
        \texttt{supervisor}. The spelling ``behavior'' is also accepted.

\end{itemize}


\subsection{Macro and record definitions}

Records and macros are defined in the same way as module attributes:

\begin{erlang}
-record(Record,Fields).

-define(Macro,Replacement).
\end{erlang}

Records and macro definitions are also allowed between functions, as
long as the definition comes before its first use. (About records see
section \ref{datatypes:record} and about macros see chapter \ref{macros}.)

\subsection{File inclusion}

File inclusion is specified in the same way as module attributes:

\begin{erlang}
-include(File).

-include_lib(File).
\end{erlang}

\texttt{File} is a string that represents a file name. Include files
are typically used for record and macro definitions that are shared by
several modules. By convention, the extension \texttt{.hrl} is used
for include files.

\begin{erlang}
-include("my_records.hrl").
-include("incdir/my_records.hrl").
-include("/home/user/proj/my_records.hrl").
\end{erlang}

If \texttt{File} starts with a path component \texttt{\$Var}, then the value of
the environment variable \texttt{Var} (returned by
\texttt{os:getenv(Var)}) is substituted for \texttt{\$Var}.

\begin{erlang}
-include("$PROJ_ROOT/my_records.hrl").
\end{erlang}
%%$ texmaker parser bug

\texttt{include\_lib} is similar to \texttt{include}, but the
first path component is assumed to be the name of an application.

\begin{erlang}
-include_lib("kernel/include/file.hrl").
\end{erlang}

The code server uses \texttt{code:lib\_dir(kernel)} to find the
directory of the current (latest) version of \texttt{kernel}, and then
the subdirectory \texttt{include} is searched for the file \texttt{file.hrl}.


\section{Comments}
Comments may appear anywhere in a module except within strings and
quoted atoms.  A comment begins with the percentage character
(\texttt{\%}) and covers the rest of the line but not the
end-of-line. The terminating end-of-line has the effect of a blank.


\section{Character Set}
Erlang handles the full Latin-1 (ISO-8859-1) character set. Thus all
Latin-1 printable characters can be used and displayed without the
escape backslash. Atoms and variables can use all Latin-1 characters.

\vspace*{12pt}
\begin{center}
\begin{tabular}{|>{\raggedright}p{52pt}|>{\raggedright}p{53pt}|>{\raggedright}p{103pt}|>{\raggedright}p{87pt}|}
\hline
\multicolumn{4}{|p{297pt}|}{Character classes}\tabularnewline
\hline
Octal & Decimal~ &   & Class\tabularnewline
\hline
40 -  57 & 32 - 47 &  ! \texttt{"} \# \$ \% \& ' / & Punctuation
characters\tabularnewline
\hline
60 -  71 & 48 - 57 & 0 - 9 & Decimal digits\tabularnewline
\hline
72 - 100 & 58 - 64 & : ; \texttt{<} = \texttt{>} @ & Punctuation characters\tabularnewline
\hline
101 - 132 &  65 - 90 & A - Z & Uppercase letters\tabularnewline
\hline
133 - 140 &  91 - 96 & [ \textbackslash{} ] \textasciicircum{} \_ ` & Punctuation
characters\tabularnewline
\hline
141 - 172 &  97 - 122 & a  -  z & Lowercase letters\tabularnewline
\hline
173 - 176 & 123 - 126 & \{ \textbar{} \} \textasciitilde{} & Punctuation characters\tabularnewline
\hline
200 - 237 & 128 - 159 ~ &   & Control characters \tabularnewline
\hline
240 - 277 & 160 - 191 & - ¿  & Punctuation characters \tabularnewline
\hline
300 - 326 & 192 - 214 & À - Ö  & Uppercase letters \tabularnewline
\hline
327  & 215 & ×  & Punctuation character \tabularnewline
\hline
330 - 336 & 216 - 222 & Ø - Þ  & Uppercase letters \tabularnewline
\hline
337 - 366 & 223 - 246 & ß - ö  & Lowercase letters \tabularnewline
\hline
367  & 247 & ÷  & Punctuation character \tabularnewline
\hline
370 - 377 & 248 - 255 & ø - ÿ  & Lowercase letters \tabularnewline
\hline
\end{tabular}
\end{center}

% because of where this lands, force a page break to avoid orphan.
\newpage
\section{Reserved words}

%\vspace{12pt}

The following are reserved words in Erlang:

\begin{erlang}
after and andalso band begin bnot bor bsl bsr bxor case catch cond
div end fun if let not of or orelse receive rem try when xor
\end{erlang}


\chapter{Data types (terms)}
\label{datatypes}

\section{Unary data types}

\subsection{Atoms}
\label{datatypes:atom}
An \textbf{atom} is a symbolic name, also known as a
\textit{literal}. They begin with a lower-case letter or must be
enclosed within single quotes (') if it contains other characters than
alphanumeric characters, underscore (\texttt{\_}) or commercial at
(\texttt{@}).

\texttt{hello}

\texttt{phone\_number}

\texttt{'Monday'}

\texttt{'phone number'}

\texttt{'Anything inside quotes \textbackslash n\textbackslash 012'}
(see section \ref{datatypes:escapeseq})


\subsection{Booleans}
\label{datatypes:boolean}
There is no \textbf{boolean} data type in Erlang. The atoms true and
false are used instead.

\texttt{2 =< 3} $\Rightarrow$ \texttt{true} \\
\texttt{true or false} $\Rightarrow$ \texttt{true}


\subsection{Integers}
\label{datatypes:integer}
In addition to the normal way of writing \textbf{integers} Erlang
provides further notations. \$Char is the Latin-1 numeric value of the
character Char and Base\#Value is an integer of the base Base, which
must be an integer in the range 2..36. Also escape sequences are
allowed.


\texttt{42} $\Rightarrow$ \texttt{42} \\
\$A  $\Rightarrow$ \texttt{65} \\
\texttt{\$\textbackslash n} $\Rightarrow$  \texttt{10}
(see section \ref{datatypes:escapeseq}) \\
\texttt{2\#101} $\Rightarrow$ \texttt{5} \\
\texttt{16\#1f} $\Rightarrow$ \texttt{31}


\subsection{Floats}
\label{datatypes:float}
A \textbf{float} is a real number written like \texttt{Num[eExp]}
where \texttt{Num} is a decimal number between 0.01 and 10000 and
\texttt{Exp} is a signed integer.

\texttt{2.3e-3} $\Rightarrow$ \texttt{2.30000e-3}
(corresponding to 2.3*10\textsuperscript{-3})


\subsection{References}
\label{datatypes:reference}
A \textbf{reference} is a term which is unique in an Erlang runtime
system, created by the \textit{BIF} (see section \ref{functions:bifs})
\texttt{make\_ref/0}


\subsection{Ports}
\label{datatypes:port}
A \textbf{port identifier} identifies a port (see chapter \ref{ports}).


\subsection{Pids}
\label{datatypes:pid}
A \textbf{process identifier}, \textit{pid}, identifies a process (see
chapter \ref{processes}).


\subsection{Funs}
\label{datatypes:fun}
A fun identifies a \textbf{functional object} (see section
\ref{functions:funs}).


\section{Compound data types}

\subsection{Tuples}
\label{datatypes:tuple}
A \textbf{tuple} is a compound data type that holds a \textbf{fixed
number of terms} enclosed within curly braces.

\texttt{\{Term1,...,TermN\}}

Each \texttt{TermX} in the tuple is called an \textbf{element}. The
number of elements is called the \textbf{size} of the tuple.

\begin{center}
\begin{tabular}{|>{\raggedright}p{134pt}|>{\raggedright}p{186pt}|}
\hline
\multicolumn{2}{|p{321pt}|}{BIFs to manipulate tuples}\tabularnewline
\hline
size(Tuple) & Returns the size of Tuple\tabularnewline
\hline
element(N,Tuple) & Returns the N\textsuperscript{th} element in Tuple\tabularnewline
\hline
setelement(N,Tuple,Expr) & Returns a new tuple copied from Tuple except that the
N\textsuperscript{th} element is replaced by Expr\tabularnewline
\hline
\end{tabular}
\end{center}

\texttt{P = \{adam, 24, \{july, 29\}\}} $\Rightarrow$ \texttt{P} is bound to \texttt{\{adam, 24, \{july, 29\}\}} \\
\texttt{element(1, P)} $\Rightarrow$ \texttt{adam} \\
\texttt{element(3, P)} $\Rightarrow$ \texttt{ \{july,29\}} \\
\texttt{P2 = setelement(2, P, 25)} $\Rightarrow$ \texttt{\{adam, 25, \{july, 29\}\}} \\
\texttt{size(P)} $\Rightarrow$ \texttt{3} \\
\texttt{size(\{\})} $\Rightarrow$ \texttt{0} \\


\subsection{Records}
\label{datatypes:record}
A \textbf{record} is a \textit{named tuple} with named elements,
called \textbf{fields}. A record named \texttt{Rec} is defined as a
module attribute.

\begin{erlang}
-record(Rec, {Field1 [= Value1],
              ...
              FieldN [= ValueN]}).
\end{erlang}

\texttt{Rec} and \texttt{Fields} are atoms and each \texttt{FieldX}
can be given an optional default \texttt{ValueX}. This definition may
be placed among the functions of a module but before it is used. If a record is used by several modules it is advisable to put it in a
separate file for inclusion.

A new record of type \texttt{Rec} is created using an expression like this:

\begin{erlang}
#Rec{Field1=Expr1, ..., FieldK=ExprK [, _=ExprL]}
\end{erlang}

The fields need not be in the same order as in the record
definition. Fields omitted will get their respective default
values. If the final clause is used, omitted fields will get the value
\texttt{ExprL}. Fields without default values and that are omitted
will have an undefined value.

The value of a field is retrieved using the expression
\texttt{Variable\#Rec.Field}

\begin{erlang}
-module(employee).
-export([new/2]).
-record(person, {name, age, employed=erixon}).

new(Name, Age) -> #person{name=Name, age=Age}.
\end{erlang}

The function \texttt{employee:new/2} can be used in another module
which must also include the same record definition of person.

\texttt{\{P = employee:new(ernie,44)\}} $\Rightarrow$ \texttt{\{person, ernie, 44,
erixon\}} \\
\texttt{P\#person.age} $\Rightarrow$ \texttt{44} \\
\texttt{P\#person.employed} $\Rightarrow$ \texttt{erixon}

When working with records in the Erlang shell, the functions \texttt{rd(RecordName, RecordDefinition)} and \texttt{rr(Module)} could be used to define and load record definitions. Refer to the Erlang Reference Manual for more information.


\subsection{Lists}
\label{datatypes:list}
A \textbf{list} is a compound data type that holds a \textit{variable}
number of \textbf{terms} enclosed within square brackets.

\texttt{[Term1,...,TermN]}

Each term \texttt{TermX} in the list is called an
\textbf{element}. The number of elements is called the \textbf{length}
of the list. The first element is called the \textbf{head} of the list
and the remainder from the 2\textsuperscript{nd} element on is called
the \textbf{tail} of the list.

\begin{center}
\begin{tabular}{|>{\raggedright}p{90pt}|>{\raggedright}p{230pt}|}
\hline
\multicolumn{2}{|p{321pt}|}{BIFs to manipulate lists}\tabularnewline
\hline
length(List) & Returns the length of List\tabularnewline
\hline
hd(List) & Returns the 1\textsuperscript{st} element of List\tabularnewline
\hline
tl(List) & Returns List with the 1\textsuperscript{st} element removed\tabularnewline
\hline
\end{tabular}
\end{center}

The vertical bar operator (\textbar{}) separates the remainder of a
list.

\texttt{[H | T]  = [1, 2, 3, 4, 5]} $\Rightarrow$ \texttt{H=1} and \texttt{T=[2, 3, 4, 5]} \\
\texttt{[X, Y | Z] = [a, b, c, d, e]} $\Rightarrow$ \texttt{X=a}, \texttt{Y=b} and \texttt{Z=[c, d, e]}

Implicitly a list will end by an empty list, i.e. \texttt{[a, b]} is
the same as \texttt{[a, b | []]}. A list looking like \texttt{[a, b |
c]} is \textbf{badly formed} and should be avoided. Lists lend
themselves naturally to recursive programming, for example the
following functions where the first computes the sum of a list and the
second returns a list where each element has been multiplied by 2.

\begin{erlang}
sum([]) -> 0;
sum([H | T]) -> H + sum(T).

double([]) -> [];
double([H | T]) -> [H*2 | double(T)].
\end{erlang}

The operator \texttt{++} appends the second argument to the first and
returns the resulting list. The operator \texttt{--} produces a list
which is a copy of the first argument except that for each element in
the second argument, the first occurrence of this element (if any) is
removed.

\texttt{[1,2,3] ++ [4,5]} $\Rightarrow$ \texttt{[1,2,3,4,5]}
\texttt{[1,2,3,2,1,2] -- [2,1,2]} $\Rightarrow$ \texttt{[3,1,2]}

A collection of list processing functions can be found in the
\texttt{STDLIB} module lists.


\subsection{Strings}
\label{datatypes:string}
\textbf{Strings} are character strings enclosed within double quotes
but are, in fact, stored as lists.

\texttt{"abcdefghi"} is the same as \texttt{[97,98,99,100,101,102,103,104,105]}

\texttt{""} is the same as \texttt{[]}

Two adjacent strings will be concatenated into one at compile-time and
do not incur any runtime overhead.

\texttt{"string"} \texttt{"42"} $\Rightarrow$ \texttt{"string42"}


\subsection{Binaries}
\label{datatypes:binary}
A binary is a chunk of untyped memory by default a sequence of 8-bit
bytes.

\texttt{<<Elem1,...,ElemN>>}

Each \texttt{ElemX} is specified as
\texttt{Value[:Size][/TypeSpecifierList]}.

\begin{center}
\begin{tabular}{|>{\raggedright}p{45pt}|>{\raggedright}p{28pt}|>{\raggedright}p{81pt}|>{\raggedright}p{147pt}|}
\hline
\multicolumn{4}{|p{297pt}|}{{\large{}Element specification}}\tabularnewline
\hline
\multicolumn{2}{|p{75pt}|}{Value} & Size & TypeSpecifierList\tabularnewline
\hline
\multicolumn{2}{|p{75pt}|}{Should evaluate to an integer, float or binary} & Should
evaluate to an integer & A sequence of optional type specifiers, in any order,
separated by hyphens (-)\tabularnewline
\hline
\multicolumn{4}{|p{297pt}|}{T{\large{}ype specifiers}}\tabularnewline
\hline
Type & \multicolumn{2}{p{120pt}|}{integer \textbar{} float \textbar{} binary} & Default
is integer\tabularnewline
\hline
Signedness & \multicolumn{2}{p{120pt}|}{signed \textbar{} unsigned} & Default is
signed\tabularnewline
\hline
Endianness & \multicolumn{2}{p{120pt}|}{big \textbar{} little \textbar{} native} & CPU
dependent. Default is big\tabularnewline
\hline
Unit & \multicolumn{2}{p{120pt}|}{unit:IntegerLiteral} & Allowed range is 1..256.
Default is 1 for integer and float and 8 for binary\tabularnewline
\hline
\end{tabular}
\end{center}

The value of \texttt{Size} multiplied by the unit gives the number of
bits for the segment. Each segment can consist of zero or more bits
but the total number of bits must be a multiple of \texttt{8}, or a
\texttt{badarg} run-time error will occur. Also, a segment of type
binary must have a size evenly divisible by \texttt{8}.

Binaries cannot be nested.

\begin{erlang}
<<1, 17, 42>>       % <<1, 17, 42>>
<<"abc">>           % <<97, 98, 99>> (The same as <<$a, $b, $c>>)
<<1, 17, 42:16>>    % <<1,17,0,42>>
<<>>                % <<>>
<<15:8/unit:10>>    % <<0,0,0,0,0,0,0,0,0,15>>
<<(-1)/unsigned>>   % <<255>>
\end{erlang}


\section{Escape sequences}
\label{datatypes:escapeseq}
Escape sequences are allowed in strings and quoted atoms.

\begin{center}
\begin{tabular}{|>{\raggedright}p{91pt}|>{\raggedright}p{229pt}|}
\hline
\multicolumn{2}{|p{321pt}|}{E{\large{}scape sequences}}\tabularnewline
\hline
\textbackslash{}b & Backspace\tabularnewline
\hline
\textbackslash{}d & Delete\tabularnewline
\hline
\textbackslash{}e & Escape\tabularnewline
\hline
\textbackslash{}f & Form feed\tabularnewline
\hline
\textbackslash{}n & New line\tabularnewline
\hline
\textbackslash{}r & carriage Return\tabularnewline
\hline
\textbackslash{}s & Space\tabularnewline
\hline
\textbackslash{}t & Tab\tabularnewline
\hline
\textbackslash{}v & Vertical tab\tabularnewline
\hline
\textbackslash{}XYZ, \textbackslash{}XY, \textbackslash{}X & Character with octal
representation XYZ, XY or X\tabularnewline
\hline
\textbackslash{}\textasciicircum{}A .. \textbackslash{}\textasciicircum{}Z & Control
A to control Z\tabularnewline
\hline
\textbackslash{}\textasciicircum{}a .. \textbackslash{}\textasciicircum{}z & Control
A to control Z\tabularnewline
\hline
\textbackslash{}' & Single quote\tabularnewline
\hline
\textbackslash{}\textbf{\texttt{"}} & Double quote\tabularnewline
\hline
\textbackslash{}\textbackslash{} & Backslash\tabularnewline
\hline
\end{tabular}
\end{center}

\section{Type conversions}
There are a number of BIFs for type conversions.

\begin{center}
\begin{tabular}{|>{\raggedright}p{63pt}|>{\raggedright}p{21pt}|>{\raggedright}p{21pt}|>{\raggedright}p{21pt}|>{\raggedright}p{21pt}|>{\raggedright}p{21pt}|>{\raggedright}p{21pt}|>{\raggedright}p{21pt}|>{\raggedright}p{22pt}|}
\hline
\multicolumn{9}{|p{237pt}|}{T{\large{}ype conversions}}\tabularnewline
\hline
 & atom & integer & float & pid & fun & tuple & list & binary\tabularnewline
\hline
atom &  & - & - & - & - & - & X & X\tabularnewline
\hline
integer & - &  & X & - & - & - & X & X\tabularnewline
\hline
float & - & X &  & - & - & - & X & X\tabularnewline
\hline
pid & - & - & - &  & - & - & X & X\tabularnewline
\hline
fun & - & - & - & - &  & - & X & X\tabularnewline
\hline
tuple & - & - & - & - & - &  & X & X\tabularnewline
\hline
list & X & X & X & X & X & X &  & X\tabularnewline
\hline
binary & X & X & X & X & X & X & X & \tabularnewline
\hline
\end{tabular}
\end{center}

The BIF \texttt{float/1} converts integers to floats. The BIFs
\texttt{round/1} and \texttt{trunc/1} convert floats to integers.

The BIFs \texttt{Type\_to\_list/1} and \texttt{list\_to\_Type/1}
convert to and from lists.

The BIFs \texttt{term\_to\_binary/1} and \texttt{binary\_to\_term/1}
convert to and from binaries.

\begin{erlang}
atom_to_list(hello)        % "hello"
list_to_atom("hello")      % hello
float_to_list(7.0)         % "7.00000000000000000000e+00"
list_to_float("7.000e+00") % 7.00000
integer_to_list(77)        % "77"
list_to_integer("77")      % 77
tuple_to_list({a, b ,c})   % [a,b,c]
list_to_tuple([a, b, c])   % {a,b,c}
pid_to_list(self())        % "<0.25.0>"
term_to_binary(<<17>>)     % <<131,109,0,0,0,1,17>>
term_to_binary({a, b ,c})  % <<131,104,3,100,0,1,97,100,0,1,98,100,0,1,99>>
binary_to_term(<<131,104,3,100,0,1,97,100,0,1,98,100,0,1,99>>)  % {a,b,c}
term_to_binary(math:pi())  % <<131,99,51,46,49,52,49,53,57,50,54,53,51,...>>
\end{erlang}



\chapter{Pattern Matching}
\label{patterns}

\section{Variables}
\label{patterns:variables}

\textbf{Variables} are introduced as arguments to a function or as a
result of pattern matching. Variables begin with an uppercase letter
or underscore (\texttt{\_}) and may contain alphanumeric characters,
underscores and at-signs (\texttt{@}). Variables can only be bound
(assigned) once.

\begin{erlang}
Abc
A_long_variable_name
AnObjectOrientatedVariableName
_Height
\end{erlang}

An \textbf{anonymous variable} is denoted by a single underscore (\texttt{\_})
and can be used when a variable is required but its value can be
ignored.

\begin{erlang}
[H|_] = [1,2,3]         % H=1 and the rest is ignored
\end{erlang}

Variables beginning with underscore like \texttt{\_Height} are normal
variables, not anonymous. They are however ignored by the compiler in
the sense that they will not generate any warnings for unused
variables. Thus it is possible to write:

\begin{erlang}
member(_Elem, []) ->
    false.
\end{erlang}

instead of:

\begin{erlang}
member(_, []) ->
    false.
\end{erlang}

which can make for more readable code.

The \textit{scope} for a variable is its function clause. Variables
bound in a branch of an \texttt{if}, \texttt{case}, or
\texttt{receive} expression must be bound in all branches to have a
value outside the expression, otherwise they will be regarded as
\textit{unsafe} (possibly undefined) outside the expression.


\section{Pattern Matching}
A \textbf{pattern} has the same structure as a term but may contain
new unbound variables.

\begin{erlang}
Name1
[H|T]
{error,Reason}
\end{erlang}

Patterns occur in \textit{function heads}, \textit{case},
\textit{receive}, and \textit{try} expressions and in match operator
(\texttt{=}) expressions. Patterns are evaluated through
\textbf{pattern matching} against an expression and this is how
variables are defined and bound.

\begin{erlang}
Pattern = Expr
\end{erlang}

Both sides of the expression must have the same structure. If the
matching succeeds, all unbound variables, if any, in the pattern
become bound. If the matching fails, a \texttt{badmatch} run-time
error will occur.


\begin{minted}{console}
> {A, B} = {answer, 42}.
{answer,42}
> A.
answer
> B.
42
\end{minted}

\subsection{Match operator (\texttt{=}) in patterns}
If \texttt{Pattern1} and \texttt{Pattern2} are valid patterns, then
the following is also a valid pattern:

\begin{erlang}
Pattern1 = Pattern2
\end{erlang}

The \texttt{=} introduces an \textbf{alias} which when matched against an
expression, both \texttt{Pattern1} and \texttt{Pattern2} are matched
against it. The purpose of this is to avoid the reconstruction of terms.

\begin{erlang}
foo({connect,From,To,Number,Options}, To) ->
    Signal = {connect,From,To,Number,Options},
    fox(Signal),
    ...;
\end{erlang}

which can be written more efficiently as:

\begin{erlang}
foo({connect,From,To,Number,Options} = Signal, To) ->
    fox(Signal),
    ...;
\end{erlang}


\subsection{String prefix in patterns}
When matching strings, the following is a valid pattern:

\begin{erlang}
f("prefix" ++ Str) -> ...
\end{erlang}

which is equivalent to and easier to read than:

\begin{erlang}
f([$p,$r,$e,$f,$i,$x | Str]) -> ...
\end{erlang}

You can only use strings as prefix expressions; patterns such as \texttt{Str ++ "postfix"} are not allowed. 

\subsection{Expressions in patterns}
An arithmetic expression can be used within a pattern, provided it
only uses numeric or bitwise operators and its value can be evaluated
to a constant at compile-time.

\begin{erlang}
case {Value, Result} of
    {?Threshold+1, ok} -> ...   % ?Threshold is a macro
\end{erlang}

\subsection{Matching binaries}

\begin{erlang}
Bin = <<1, 2, 3>>               % <<1,2,3>> All elements are 8-bit bytes
<<A, B, C>> = Bin               % A=1, B=2 and C=3
<<D:16, E>> = Bin               % D=258 and E=3
<<F, G/binary>> = Bin           % F=1 and G=<<2,3>>
\end{erlang}

In the last line, the variable \texttt{G} of unspecified size matches the
rest of the binary \texttt{Bin}.

Always put a space between (\texttt{=}) and (\verb|<<|) so as to
avoid confusion with the (\texttt{=<}) operator.



\chapter{Functions}

\section{Function definition}
A function is defined as a sequence of one or more \textbf{function
clauses}. The function name is an atom.

\vspace*{4pt}
\begin{erlang}
Func(Pattern11,...,Pattern1N) [when GuardSeq1] -> Body1;
    ...;
    ...;
Func(PatternK1,...,PatternKN) [when GuardSeqK] -> BodyK.

\end{erlang}
\vspace*{4pt}

The function clauses are separated by semicolons (\texttt{;}) and
terminated by full stop (\texttt{.}). A function clause consists of a
\textbf{clause head} and a \textbf{clause body} separated by an arrow
(\texttt{->}). A clause head consists of the function name (an atom),
arguments within parentheses and an optional guard sequence beginning
with the keyword \texttt{when}.  Each argument is a pattern.  A clause
body consists of a sequence of expressions separated by commas
(\texttt{,}).

\vspace*{4pt}
\begin{erlang}
Expr1,
...,
ExprM
\end{erlang}
\vspace*{4pt}

The number of arguments \texttt{N} is the \textbf{arity} of the
function. A function is uniquely defined by the module name, function name
and arity. Two different functions in the same module with different
arities may have the same name. A function \texttt{Func} in \texttt{Module}
with arity \texttt{N} is often denoted as \texttt{Module:Func/N}.

\vspace*{4pt}
\begin{erlang}
-module(mathStuff).
-export([area/1]).

area({square, Side}) -> Side * Side;
area({circle, Radius}) -> math:pi() * Radius * Radius;
area({triangle, A, B, C}) ->
    S = (A + B + C)/2,
    math:sqrt(S*(S-A)*(S-B)*(S-C)).
\end{erlang}

% push this section onto the next page (orphan line)
\newpage
\section{Function calls}
A function is called using:

\begin{erlang}
[Module:]Func(Expr1, ..., ExprN)
\end{erlang}

\texttt{Module} evaluates to a module name and \texttt{Func} to
a function name or a \textit{fun}. When calling a function in another
module, the module name must be provided and the function must be
exported. This is referred to as a \textbf{fully qualified function name}.

\begin{erlang}
lists:keysearch(Name, 1, List)
\end{erlang}

The module name can be omitted if \texttt{Func} evaluates to the name
of a local function, an imported function, or an auto-imported
BIF.  In such cases, the function is called using an \textbf{implicitly qualified function name}.

Before calling a function the arguments \texttt{ExprX} are
evaluated.  If the function cannot be found, an \texttt{undef} run-time
error will occur. Next the function clauses are scanned sequentially
until a clause is found such that the patterns in the clause head can
be successfully matched against the given arguments and that the guard
sequence, if any, is true. If no such clause can be found, a
\texttt{function\_clause} run-time error will occur.

If a matching clause is found, the corresponding clause body is evaluated,
i.e.~the expressions in the body are evaluated sequentially and the
value of the last expression is returned.

The fully qualified function name must be used when calling a function
with the same name as a BIF (built-in function, see section
\ref{functions:bifs}). The compiler does not allow defining a function
with the same name as an imported function. When calling a local
function, there is a difference between using the implicitly or fully
qualified function name, as the latter always refers to the latest
version of the module (see chapter \ref{code}).


\section{Expressions}
\label{functions:expressions}
An \textbf{expression} is either a term or the invocation of an
operator, for example:

\begin{erlang}
Term
op Expr
Expr1 op Expr2
(Expr)
begin
   Expr1,
   ...,
   ExprM            % no comma (,) before end
end

\end{erlang}

There are both \textit{unary} and \textit{binary} operators. The
simplest form of expression is a term, i.e.~an \textit{integer},
\textit{float}, \textit{atom}, \textit{string}, \textit{list} or
\textit{tuple} and the return value is the term itself. An expression
may contain \textit{macro} or \textit{record} expressions which will
expanded at compile time.

Parenthesised expressions are useful to override operator precedence (see section \ref{functions:expressions:precedence}):

\begin{erlang}
1 + 2 * 3           % 7
(1 + 2) * 3         % 9
\end{erlang}

Block expressions within \texttt{begin...end} can be used to group a
sequence of expressions and the return value is the value of the last
expression \texttt{ExprM}.

All subexpressions are evaluated before the expression itself is
evaluated, but the order in which subexpressions are evaluated is undefined.

Most operators can only be applied to arguments of a certain type. For
example, arithmetic operators can only be applied to integers or
floats. An argument of the wrong type will cause a \texttt{badarg}
run-time error.


\subsection{Term comparisons}
\begin{erlang}
Expr1 op Expr2
\end{erlang}

A \textbf{term comparison} returns a \textit{boolean} value,
in the form of atoms \texttt{true} or \texttt{false}.

\begin{center}
\begin{tabular}{|>{\raggedright}p{40pt}|>{\raggedright}p{105pt}|>{\raggedright}p{26pt}|>{\raggedright}p{124pt}|}
\hline
\multicolumn{4}{|p{297pt}|}{Comparison operators}\tabularnewline
\hline
\texttt{==} & Equal to & \texttt{=<} & Less than or equal to\tabularnewline
\hline
\texttt{/=} & Not equal to & \texttt{<} & Less than\tabularnewline
\hline
\texttt{=:=} & Exactly equal to & \texttt{>}= & Greater than or equal to\tabularnewline
\hline
\texttt{=/=} & Exactly not equal to & \texttt{>} & Greater than\tabularnewline
\hline
\end{tabular}
\end{center}

\begin{erlang}
1==1.0              % true
1=:=1.0             % false
1 > a               % false
\end{erlang}

The arguments may be of different data types. The following order is
defined:

\texttt{number < atom < reference < fun < port < pid < tuple < list < binary}

Lists are compared element by element. Tuples are ordered by size, two
tuples with the same size are compared element by element. When
comparing an integer and a float, the integer is first converted to a
float. In the case of \texttt{=:=} or \texttt{=/=} there is no type conversion.


\subsection{Arithmetic expressions}

\begin{erlang}
op Expr
Expr1 op Expr2
\end{erlang}

An \textbf{arithmetic expression} returns the result after applying
the operator.

\begin{center}
\begin{tabular}{|>{\raggedright}p{35pt}|>{\raggedright}p{145pt}|>{\raggedright}p{128pt}|}
\hline
\multicolumn{3}{|p{309pt}|}{Arithmetic operators}\tabularnewline
\hline
\texttt{+}  & Unary +  & \texttt{Integer \textbar{} Float} \tabularnewline
\hline
\texttt{-}  & Unary -  & \texttt{Integer \textbar{} Float}\tabularnewline
\hline
\texttt{+}  & Addition & \texttt{Integer} \textbar{} \texttt{Float}\tabularnewline
\hline
\texttt{-}  & Subtraction & \texttt{Integer} \textbar{} \texttt{Float}\tabularnewline
\hline
\texttt{*}  & Multiplication & \texttt{Integer} \textbar{} \texttt{Float}\tabularnewline
\hline
\texttt{/}  & Floating point division  & \texttt{Integer} \textbar{} \texttt{Float}\tabularnewline
\hline
\texttt{bnot}  & Unary bitwise not  & \texttt{Integer} \tabularnewline
\hline
\texttt{div}  & Integer division  & \texttt{Integer}\tabularnewline
\hline
\texttt{rem}  & Integer remainder of X/Y  & \texttt{Integer} \tabularnewline
\hline
\texttt{band}  & Bitwise and  & \texttt{Integer}\tabularnewline
\hline
\texttt{bor}  & Bitwise or  & \texttt{Integer} \tabularnewline
\hline
\texttt{bxor}  & Arithmetic bitwise xor  & \texttt{Integer}\tabularnewline
\hline
\texttt{bsl}  & Arithmetic bitshift left  & \texttt{Integer} \tabularnewline
\hline
\texttt{bsr}  & Bitshift right  & \texttt{Integer}\tabularnewline
\hline
\end{tabular}
\end{center}

\begin{erlang}
+1                  % 1
4/2                 % 2.00000
5 div 2             % 2
5 rem 2             % 1
2#10 band 2#01      % 0
2#10 bor 2#01       % 3
\end{erlang}


\subsection{Boolean expressions}

\begin{erlang}
op Expr
Expr1 op Expr2
\end{erlang}

A \textbf{boolean expression} returns the value \texttt{true} or
\texttt{false} after applying the operator.

\begin{center}
\begin{tabular}{|>{\raggedright}p{79pt}|>{\raggedright}p{241pt}|}
\hline
\multicolumn{2}{|p{321pt}|}{Boolean operators}\tabularnewline
\hline
\texttt{not}  & Unary logical not \tabularnewline
\hline
\texttt{and}  & Logical and \tabularnewline
\hline
\texttt{or}  & Logical or \tabularnewline
\hline
\texttt{xor}  & Logical exclusive or\tabularnewline
\hline
\end{tabular}
\end{center}

\begin{erlang}
not true            % false
true and false      % false
true xor false      % true
\end{erlang}


\subsection{Short-circuit boolean expressions}

\begin{erlang}
Expr1 orelse Expr2
Expr1 andalso Expr2
\end{erlang}

These are boolean expressions where \texttt{Expr2} is evaluated only
if necessary. In an \texttt{orelse} expression \texttt{Expr2} will be
evaluated only if \texttt{Expr1} evaluates to false. In an
\texttt{andalso} expression \texttt{Expr2} will be evaluated only if
\texttt{Expr1} evaluates to true.

\begin{erlang}
if A >= 0 andalso math:sqrt(A) > B -> ...

if is_list(L) andalso length(L) == 1 -> ...
\end{erlang}


\subsection{Operator precedences}
\label{functions:expressions:precedence}
In an expression consisting of subexpressions the operators will be
applied according to a defined \textbf{operator precedence} order:

\begin{center}
\begin{tabular}{|>{\raggedright}p{221pt}|>{\raggedright}p{99pt}|}
\hline
\multicolumn{2}{|p{321pt}|}{Operator precedence (from high to low)}\tabularnewline
\hline
\texttt{:} ~ &  \tabularnewline
\hline
\texttt{\#} ~ &  \tabularnewline
\hline
\texttt{Unary + - bnot not ~} &  \tabularnewline
\hline
\texttt{/ * div rem band and}  & Left associative \tabularnewline
\hline
\texttt{+ - bor bxor bsl bsr or xor} & Left associative \tabularnewline
\hline
\texttt{++ --}  & Right associative \tabularnewline
\hline
\texttt{== /= =< < >= > =:= =/=} & \tabularnewline
\hline
\texttt{andalso}  &  \tabularnewline
\hline
\texttt{orelse} &  \tabularnewline
\hline
\texttt{= !}  & Right associative \tabularnewline
\hline
\texttt{catch ~} &  \tabularnewline
\hline
\end{tabular}
\end{center}

The operator with the highest priority is evaluated first. Operators
with the same priority are evaluated according to their
\textbf{associativity}.  The left associative arithmetic operators
are evaluated left to right:

\texttt{6 + 5 * 4 - 3 / 2  \resultingin  6 + 20 - 1.5  \resultingin  26 - 1.5  \resultingin  24.5}


\section{Compound expressions}


\subsection{If}

\begin{erlang}
if
    GuardSeq1 ->
        Body1;
    ...;
    GuardSeqN ->
        BodyN                   % Note no semicolon (;) before end
end
\end{erlang}

The branches of an \texttt{if} expression are scanned sequentially
until a guard sequence \texttt{GuardSeq} which evaluates to
\texttt{true} is found.  The corresponding \texttt{Body} (sequence
of expressions separated by commas) is then evaluated.  The return value of
\texttt{Body} is the return value of the \texttt{if} expression.

If no guard sequence is true, an \texttt{if\_clause} run-time error
will occur. If necessary, the guard expression \texttt{true} can be used in the
last branch, as that guard sequence is always true (known as a ``catch
all'').

\begin{erlang}
is_greater_than(X, Y) ->
    if
        X>Y ->
            true;
        true ->                 % works as an 'else' branch
            false
    end
\end{erlang}

It should be noted that pattern matching in function clauses can be used to replace \texttt{if} cases (most of the time).
Overuse of \texttt{if} sentences withing function bodies is considered a bad Erlang practice.

\subsection{Case}

Case expressions provide for inline pattern matching, similar to the way in which function clauses are matched.

\begin{erlang}
case Expr of
    Pattern1 [when GuardSeq1] ->
        Body1;
        ...;
    PatternN [when GuardSeqN] ->
        BodyN                   % Note no semicolon (;) before end
end
\end{erlang}

The expression \texttt{Expr} is evaluated and the patterns
\texttt{Pattern1}...\texttt{PatternN} are sequentially matched against the result. If a
match succeeds and the optional guard sequence \texttt{GuardSeqX} is
\texttt{true}, then the corresponding \texttt{BodyX} is evaluated. The return value
of \texttt{BodyX} is the return value of the case expression.

If there is no matching pattern with a true guard sequence, a
\texttt{case\_clause} run-time error will occur.

\begin{erlang}
is_valid_signal(Signal) ->
    case Signal of
        {signal, _What, _From, _To} ->
            true;
        {signal, _What, _To} ->
            true;
        _Else ->                % 'catch all'
            false
    end.
\end{erlang}


\subsection{List comprehensions}
List comprehensions are analogous to the \texttt{setof} and
\texttt{findall} predicates in Prolog.

\begin{erlang}
[Expr || Qualifier1,...,QualifierN]
\end{erlang}

\texttt{Expr} is an arbitrary expression, and each \texttt{QualifierX}
is either a \textbf{generator} or a \textbf{filter}. A generator is
written as:

\begin{erlang}
Pattern <- ListExpr
\end{erlang}

where \texttt{ListExpr} must be an expression which evaluates to a
list of terms. A filter is an expression which evaluates to
\texttt{true} or \texttt{false}. Variables in list generator
expressions \textit{shadow} variables in the function clause
surrounding the list comprehension.

The qualifiers are evaluated from left to right, the generators
creating values and the filters constraining them. The list
comprehension then returns a list where the elements are the result of
evaluating \texttt{Expr} for each combination of the resulting values.

\begin{minted}{console}
> [{X, Y} || X <- [1,2,3,4,5,6], X > 4, Y <- [a,b,c]].
[{5,a},{5,b},{5,c},{6,a},{6,b},{6,c}]
\end{minted}


\section{Guard sequences}
A \textbf{guard sequence} is a set of \textbf{guards} separated by
semicolons (\texttt{;}). The guard sequence is \texttt{true} if at
least one of the guards is \texttt{true}.

\begin{erlang}
Guard1; ...; GuardK
\end{erlang}

A \textbf{guard} is a set of \textbf{guard expressions} separated by
commas (\texttt{,}). The guard is \texttt{true} if all guard
expressions evaluate to \texttt{true}.

\begin{erlang}
GuardExpr1, ..., GuardExprN
\end{erlang}

The permitted \textbf{guard expressions} (sometimes called guard
tests) are a subset of valid Erlang expressions, since the
evaluation of a guard expression must be guaranteed to be free of side-effects.

\begin{center}
\begin{tabular}{|>{\raggedright}p{154pt}|>{\raggedright}p{166pt}|}
\hline
\multicolumn{2}{|p{321pt}|}{Valid guard expressions:}\tabularnewline
\hline
\multicolumn{2}{|p{321pt}|}{The atom \texttt{true};}\tabularnewline
\hline
\multicolumn{2}{|p{321pt}|}{Other constants (terms and bound variables), are all regarded
as \texttt{false};}\tabularnewline
\hline
\multicolumn{2}{|p{321pt}|}{Term comparisons;}\tabularnewline
\hline
\multicolumn{2}{|p{321pt}|}{Arithmetic and boolean expressions;}\tabularnewline
\hline
\multicolumn{2}{|p{321pt}|}{Calls to the BIFs specified below.}\tabularnewline
\hline
Type test BIFs & Other BIFs allowed in guards:\tabularnewline
\hline
\texttt{is\_atom/1} & \texttt{abs(Integer} \textbar{} \texttt{Float)}\tabularnewline
\hline
\texttt{is\_constant/1} & \texttt{float(Term)}\tabularnewline
\hline
\texttt{is\_integer/1} & \texttt{trunc(Integer} \textbar{} \texttt{Float)}\tabularnewline
\hline
\texttt{is\_float/1} & \texttt{round(Integer} \textbar{} \texttt{Float)}\tabularnewline
\hline
\texttt{is\_number/1} & \texttt{size(Tuple} \textbar{} \texttt{Binary)}\tabularnewline
\hline
\texttt{is\_reference/1} & \texttt{element(N, Tuple)}\tabularnewline
\hline
\texttt{is\_port/1} & \texttt{hd(List)}\tabularnewline
\hline
\texttt{is\_pid/1} & \texttt{tl(List)}\tabularnewline
\hline
\texttt{is\_function/1} & \texttt{length(List)}\tabularnewline
\hline
\texttt{is\_tuple/1} & \texttt{self()}\tabularnewline
\hline
\texttt{is\_record/2} The 2\textsuperscript{nd} argument is \linebreak{}
the record name & \texttt{node(})\tabularnewline
\hline
\texttt{is\_list/1} & \texttt{node(Pid} \textbar{} \texttt{Ref} \textbar \texttt{Port)}\tabularnewline
\hline
\texttt{is\_binary/1} & \tabularnewline
\hline
\end{tabular}
\end{center}

A small example:

\begin{erlang}
fact(N) when N>0 ->             % first clause head
    N * fact(N-1);              % first clause body
fact(0) ->                      % second clause head
    1.                          % second clause body
\end{erlang}


\section{Tail recursion}
If the last expression of a function body is a function call, a
\textbf{tail recursive} call is performed in such a way that no system
resources (like the call stack) are consumed. This means that an
infinite loop like a server can be programmed provided it only uses
tail recursive calls.

The function \texttt{fact/1} above could be rewritten using tail
recursion in the following manner:

 \begin{erlang}
fact(N) when N>1 -> fact(N, N-1);
fact(N) when N==1; N==0 -> 1.

fact(F,0) -> F;                 % The variable F is used as an accumulator
fact(F,N) -> fact(F*N, N-1).
\end{erlang}


\section{Funs}
\label{functions:funs}
A \textbf{fun} defines a \textit{functional object}. Funs make it
possible to pass an entire function, not just the function name, as an
argument. A `fun' expression begins with the keyword \texttt{fun} and
ends with the keyword \texttt{end} instead of a full stop
(\texttt{.}).  Between these should be a regular function
declaration, except that no function name is specified.

\begin{erlang}
fun
    (Pattern11,...,Pattern1N) [when GuardSeq1] ->
        Body1;
        ...;
    (PatternK1,...,PatternKN) [when GuardSeqK] ->
        BodyK
end
\end{erlang}

Variables in a \texttt{fun} head \textit{shadow} variables in the function
clause surrounding the \texttt{fun} but variables bound in a \texttt{fun} body are local
to the body.  The return value of the expression is the resulting function. The expression
\texttt{fun Name/N is} equivalent to:

\begin{erlang}
fun (Arg1,...,ArgN) -> Name(Arg1,...,ArgN) end
\end{erlang}

The expression \texttt{fun Module:Func/Arity} is also allowed, provided that \texttt{Func} is exported
from \texttt{Module}.

\begin{erlang}
Fun1 = fun (X) -> X+1 end.
Fun1(2)         % 3

Fun2 = fun (X) when X>=1000 -> big; (X) -> small end.
Fun2(2000)      % big
\end{erlang}

Since a \texttt{fun} is anonymous, i.e.~there is no function name in the
definition of the \texttt{fun}, the definition of a recursive \texttt{fun} has to be
done in two steps.  This example shows how to define the function
\texttt{sum(List)} (see section \ref{datatypes:list}) as a \texttt{fun}.

\begin{erlang}
Sum1 = fun ([], _Foo) -> 0;([H|T], Foo) -> H + Foo(T, Foo) end.
Sum = fun (List) -> Sum1(List, Sum1) end.
Sum([1,2,3,4,5])    % 15
\end{erlang}

The definition of \texttt{Sum1} is done in a way such that it takes \textit{itself} as a parameter, matched to \texttt{\_Foo} (empty list) or \texttt{Foo},
which it then calls recursively.  The definition of \texttt{Sum} calls \texttt{Sum1}, also passing \texttt{Sum1} as a parameter.

\section{BIFs --- Built-in functions}
\label{functions:bifs}
The \textbf{built-in functions}, BIFs, are implemented in the C code of
the runtime system and do things that are difficult or impossible to
implement in Erlang. Most of the built-in functions belong to the
module \texttt{erlang} but there are also built-in functions that belong
to other modules like \texttt{lists} and \texttt{ets}. The most
commonly used BIFs belonging to the module \texttt{erlang} are
\textbf{auto-imported}, i.e.~they do not need to be prefixed with the
module name.

\begin{center}
\begin{tabular}{|>{\raggedright}p{103pt}|>{\raggedright}p{217pt}|}
\hline
\multicolumn{2}{|p{321pt}|}{Some useful BIFs}\tabularnewline
\hline
\texttt{date()} & Returns today's date as \texttt{\{Year, Month, Day\}}\tabularnewline
\hline
\texttt{now()} & Returns current time in microseconds. System dependent\tabularnewline
\hline
\texttt{time()} & Returns current time as \texttt{\{Hour, Minute, Second\}} System dependent\tabularnewline
\hline
\texttt{halt()} & Stops the Erlang system\tabularnewline
\hline
\texttt{processes()} & Returns a list of all processes currently known to the system\tabularnewline
\hline
\texttt{process\_info(Pid)} & Returns a dictionary containing information about \texttt{Pid}\tabularnewline
\hline
\texttt{Module:module\_info()} & Returns a dictionary containing information about the code
in Module\tabularnewline
\hline
\end{tabular}
\end{center}

A \textbf{dictionary} is a list of \texttt{\{Key, Value\} terms (see
also section \ref{processes:dicts}).}

\begin{erlang}
size({a, b, c})             % 3
atom_to_list('Erlang')      % "Erlang"
date()                      % {2013,5,27}
time()                      % {01,27,42}
\end{erlang}


\chapter{Processes}
\label{processes}

A \textbf{process} corresponds to one \textit{thread of
  control}. Erlang permits very large numbers of concurrent processes
each executing like it had an own virtual processor. When a process
executing a \texttt{function A} calls another \texttt{function B}, it
will wait until \texttt{function B} is finished and then retrieve its
result. If instead it \textit{spawns} another process executing
\texttt{function B}, both will continue in parallel
(concurrently). \texttt{function A} will not wait for \texttt{function
  B} and the only way they can communicate is through \textit{message
  passing}.

Erlang processes are light-weight with small memory footprint, fast to
create and terminate and the scheduling overhead is low. A
\textbf{process identifier}, \texttt{Pid}, identifies a process. The
BIF \texttt{self/0} returns the \texttt{Pid} of the process itself.


\section{Process creation}
A process is created using the BIF \texttt{spawn/3}.

\begin{erlang}
spawn(Module, Func, [Expr1, ..., ExprN])
\end{erlang}

\texttt{Module} should evaluate to a module name and \texttt{Func} to
a function name in that module. The list of \texttt{Exprs} are the
arguments to the function. \texttt{spawn} creates a new process and
returns the process identifier, \texttt{Pid}. The new process starts
by executing

\begin{erlang}
Module:Func(Expr1, ..., ExprN)
\end{erlang}

The function \texttt{Func} has to be exported even if it is spawned by
another function in the same module. There are other spawn BIFs, for
example \texttt{spawn/4} for spawning a process at another node.


\section{Registered processes}
A process can be associated with a name. The name must be an atom and
is automatically unregistered if the process terminates. Only static
(cyclic) processes should be registered.

\begin{center}
\begin{tabular}{|>{\raggedright}p{117pt}|>{\raggedright}p{204pt}|}
\hline
\multicolumn{2}{|p{321pt}|}{Name registration BIFs}\tabularnewline
\hline
\texttt{register(Name, Pid)}  & Associates the atom Name with the process Pid\tabularnewline
\hline
\texttt{registered()}  & Returns a list of names which have been registered \tabularnewline
\hline
\texttt{whereis(Name)}  & Returns the Pid registered under Name or undefined if the name
is not registered\tabularnewline
\hline
\end{tabular}
\end{center}


\section{Process communication}
Processes communicate by sending and receiving
\textbf{messages}. Messages are sent using the send operator
(\texttt{!}) and are received using \texttt{receive}. Message passing
is asynchronous and reliable, i.e. the message is guaranteed to
eventually reach the recipient, provided that the recipient exists.


\subsection{Send}

\begin{erlang}
Pid ! Expr
\end{erlang}

The send (\texttt{!}) operator sends the value of \texttt{Expr} as a
message to the process specified by \texttt{Pid} where it will be
placed last in its \textbf{message queue}. The value of \texttt{Expr}
is also the return value of the (\texttt{!}) expression. \texttt{Pid}
must evaluate to a process identifier, a registered name or a tuple
\texttt{\{Name,Node\}}, where \texttt{Name} is a registered process at
\texttt{Node} (see chapter \ref{distribution}). The message sending operator
(\texttt{!}) never fails, even if it addresses a non-existing process.


\subsection{Receive}

\begin{erlang}
receive
    Pattern1 [when GuardSeq1] ->
        Body1;
    ...
    PatternN [when GuardSeqN] ->
        BodyN                   % Note no semicolon (;) before end
end
\end{erlang}

This expression receives messages sent to the process using the send
operator (\texttt{!}). The patterns \texttt{PatternX} are sequentially
matched against the first message in time order in the message queue,
then the second, etc. If a match succeeds and the optional guard
sequence \texttt{GuardSeqX} is true, then the message is removed from
the message queue and the corresponding \texttt{BodyX} is
evaluated. It is the order of the pattern clauses that decides the
order in which messages will be received prior to the order in which
they have arrived. This is called \textit{selective receive}. The
return value of \texttt{BodyX} is the return value of the receive
expression.

\texttt{receive} never fails. The process is suspended, possibly
indefinitely, until a message arrives that matches one of the patterns
and with a true guard sequence.


\begin{erlang}
wait_for_onhook() ->
    receive
        onhook ->
            disconnect(),
            idle();
        {connect, B} ->
            B ! {busy, self()},
            wait_for_onhook()
    end.
\end{erlang}


\subsection{Receive with timeout}

\begin{erlang}
receive
    Pattern1 [when GuardSeq1] ->
        Body1;
        ...;
    PatternN [when GuardSeqN] ->
        BodyN
after
    ExprT ->
        BodyT
end
\end{erlang}

\texttt{ExprT} should evaluate to an integer between \texttt{0} and
\texttt{16\#ffffffff} (the value must fit in 32 bits). If no matching
message has arrived within \texttt{ExprT} milliseconds, then
\texttt{BodyT} will be evaluated and its return value becomes the
return value of the receive expression.

\begin{erlang}
wait_for_onhook() ->
    receive
        onhook ->
            disconnect(),
            idle();
        {connect, B} ->
            B ! {busy, self()},
            wait_for_onhook()
    after
        60000 ->
            disconnect(),
            error()
    end.
\end{erlang}

A \texttt{receive...after} expression with no branches can be used to
implement simple timers.

\begin{erlang}
receive
after
    ExprT ->
        BodyT
end
\end{erlang}

\begin{center}
\begin{tabular}{|>{\raggedright}p{47pt}|>{\raggedright}p{273pt}|}
\hline
\multicolumn{2}{|p{321pt}|}{Two special cases for the timeout value \texttt{ExprT}}\tabularnewline
\hline
\texttt{infinity} & This is equivalent to not using a timeout and can be useful for timeout
values that are calculated at run-time\tabularnewline
\hline
\texttt{0} & If there is no matching message in the mailbox, the timeout will occur immediately\tabularnewline
\hline
\end{tabular}
\end{center}


\section{Process termination}
\label{processes:termination}
A process always terminates with an \textbf{exit reason }which may be
any term. If a process terminates normally, for instance if it has run
to the end of its code, then the reason is the atom \texttt{normal}. A process
can terminate itself by calling one of the following BIFs.

\begin{erlang}
exit(Reason)

erlang:error(Reason)

erlang:error(Reason, Args)
\end{erlang}

A process terminates with exit reason \texttt{\{Reason,Stack\}} when a
run-time error occurs.

A process may also be terminated if it receives an exit signal with
another exit reason than \texttt{normal} (see section
\ref{processes:recvexitsignals}).


\section{Process links}
\label{processes:links}
Two processes can be \textbf{linked} to each other. Links are
bidirectional and there can only be one link between two processes. A
process with \texttt{Pid1} can link to a process with \texttt{Pid2}
using the BIF \texttt{link(Pid2)}. The BIF \texttt{spawn\_link(Module,
  Func, Args)} spawns and links a process in one atomic operation.

A link can be removed using the BIF \texttt{unlink(Pid)}.


\subsection{Error handling between processes}
When a process terminates it will send \textbf{exit signals} to all
processes that it is linked to. These in turn will also be terminated
\textit{or handle the exit signal in some way}. This feature can be
used to build hierarchical program structures where some processes are
supervising other processes, for example restarting them if they
terminate abnormally.


\subsection{Sending exit signals}
\label{processes:sendexitsignals}
A process always terminates with an exit reason which is sent as an
exit signal to all linked processes. The BIF \texttt{exit(Pid,
  Reason)} sends an exit signal with the exit reason Reason to
\texttt{Pid}, without affecting the calling process.


\subsection{Receiving exit signals}
\label{processes:recvexitsignals}
If a process receives an exit signal with an exit reason other than
normal it will also be terminated and will send exit signals with the
same exit reason to its linked processes. An exit signal with reason
normal is ignored. This behaviour can be changed using the BIF
\texttt{process\_flag(trap\_exit, true)}.

Then the process is able to \textbf{trap exits}. This means that an
exit signal will be transformed into a message \texttt{\{'EXIT',
  FromPid, Reason\}} which is put into the message queue and can be
handled by the process like a regular message using receive.

However, a call to the BIF \texttt{exit(Pid, kill)} unconditionally
terminates the process \texttt{Pid} regardless whether it is able to
trap exit signals or not.


\section{Monitors}
A process \texttt{Pid1} can create a \textbf{monitor} for
\texttt{Pid2} using the BIF

\begin{erlang}
erlang:monitor(process, Pid2)
\end{erlang}

which returns a reference \texttt{Ref}. If \texttt{Pid2} terminates
with exit reason \texttt{Reason}, a message as follows will be sent to
\texttt{Pid1}.

\begin{erlang}
{'DOWN', Ref, process, Pid2, Reason}
\end{erlang}

If \texttt{Pid2} does not exist, the \texttt{'DOWN'} message is sent
immediately with \texttt{Reason} set to \texttt{noproc}. Monitors are
unidirectional in that if \texttt{Pid1} monitors \texttt{Pid2} then it
will receive a message when \texttt{Pid2} dies but \texttt{Pid2} will
\textbf{not} receive a message when \texttt{Pid1} dies. Repeated calls
to \texttt{erlang:monitor(process, Pid)} will create several,
independent monitors and each one will send a \texttt{'DOWN'} message
when \texttt{Pid} terminates.

A monitor can be removed by calling \texttt{erlang:demonitor(Ref)}. It
is possible to create monitors for processes with registered names,
also at other nodes.


\section{Process priorities}
The BIF \texttt{process\_flag(priority, Prio)} defines the priority of
the current process. \texttt{Prio} may have the value \texttt{normal},
which is the default, or \texttt{high} or \texttt{low} or \texttt{max}. This should be
used very sparingly.


\section{Process dictionary}
\label{processes:dicts}
Each process has its own process dictionary which is a list of
\texttt{\{Key, Value\}} terms.

\begin{center}
\begin{tabular}{|>{\raggedright}p{79pt}|>{\raggedright}p{247pt}|}
\hline
\multicolumn{2}{|p{326pt}|}{Process dictionary BIFs}\tabularnewline
\hline
\texttt{put(Key, Value)} & Saves the Value under the Key or replaces an older value\tabularnewline
\hline
\texttt{get(Key)} & Retrieves the Value stored under Key or undefined\tabularnewline
\hline
\texttt{get()} & Returns the entire process dictionary as a list of \{Key, Value\} terms\tabularnewline
\hline
\texttt{get\_keys(Value)} & Returns a list of keys that have the value Value\tabularnewline
\hline
\texttt{erase(Key)} & Deletes \{Key, Value\}, if any, and returns Key\tabularnewline
\hline
\texttt{erase()} & Returns the entire process dictionary and deletes it\tabularnewline
\hline
\end{tabular}
\end{center}

Process dictionaries could be used to keep global variables within an application but the extensive use of them is usually regarded as poor programming style.


\chapter{Error handling}
\label{errorhandling}

This chapter deals with error handling within a process. Such errors
are known as \textbf{exceptions}.


\section{Exception classes and error reasons}

\begin{center}
\begin{tabular}{|>{\raggedright}p{49pt}|>{\raggedright}p{277pt}|}
\hline
\multicolumn{2}{|p{326pt}|}{Exception classes}\tabularnewline
\hline
\texttt{error} & Run-time error for example when applying an operator to the wrong types of arguments. Run-time errors can be emulated by calling the BIFs \texttt{erlang:error(Reason)} or \texttt{erlang:error(Reason, Args)} \tabularnewline
\hline
\texttt{exit}  & The process calls \texttt{exit(Reason)}, see section \ref{processes:termination}\tabularnewline
\hline
\texttt{throw}  & The process calls \texttt{throw(Expr)}, see section \ref{errorhandling:catchthrow}\tabularnewline
\hline
\end{tabular}
\end{center}

An exception will cause the process to crash, i.e. its execution is
stopped and it is removed from the system. It is also said to
\textit{terminate}. Then exit signals will be sent to any linked
processes.\textit{ }An exception consists of its class, an exit reason
and a stack. The stack trace can be retrieved using the BIF
\texttt{erlang:get\_stacktrace/0}.

Run-time errors and other exceptions can be prevented from causing the
process to terminate by using the expressions catch or try.

For exceptions of class error, for example normal run-time errors, the
\textbf{exit reason} is a tuple \texttt{\{Reason, Stack\}} where
Reason is a term indicating which type of error.

\begin{center}
\begin{tabular}{|>{\raggedright}p{100pt}|>{\raggedright}p{226pt}|}
\hline
\multicolumn{2}{|p{326pt}|}{Exit reasons}\tabularnewline
\hline
\texttt{badarg}  & Argument is of wrong type. \tabularnewline
\hline
\texttt{badarith}  & Argument is of wrong type in an arithmetic expression. \tabularnewline
\hline
\texttt{\{badmatch, Value\}}  & Evaluation of a match expression failed. Value did not match.
\tabularnewline
\hline
\texttt{function\_clause}  & No matching function clause is found when evaluating a function
call. \tabularnewline
\hline
\texttt{\{case\_clause, Value\}}  & No matching branch is found when evaluating a case expression.
Value did not match. \tabularnewline
\hline
\texttt{if\_clause}  & No true branch is found when evaluating an if expression. \tabularnewline
\hline
\texttt{\{try\_clause, Value\}}  & No matching branch is found when evaluating the of section
of a \texttt{try} expression. Value did not match. \tabularnewline
\hline
\texttt{undef}  & The function cannot be found when evaluating a function call\tabularnewline
\hline
\texttt{\{badfun, Fun\}}  & There is something wrong with a Fun\tabularnewline
\hline
\texttt{\{badarity, Fun\}}  & A fun is applied to the wrong number of arguments. Fun describes
it and the arguments\tabularnewline
\hline
\texttt{timeout\_value}  & The timeout value in a receive..after expression is evaluated
to something else than an integer or infinity\tabularnewline
\hline
\texttt{noproc}  & Trying to link to a non-existing process\tabularnewline
\hline
\texttt{\{nocatch, Value\}}  & Trying to evaluate a throw outside a catch. Value is the thrown
term\tabularnewline
\hline
\texttt{system\_limit}  & A system limit has been reached\tabularnewline
\hline
\end{tabular}
\end{center}

Stack is the stack of function calls being evaluated when the error
occurred, given as a list of tuples \texttt{\{Module, Name, Arity\}}
with the most recent function call first.  The most recent function
call tuple may in some cases be \texttt{\{Module, Name, Args\}}


\section{Catch and throw}
\label{errorhandling:catchthrow}
\begin{erlang}
catch Expr
\end{erlang}

This returns the value of \texttt{Expr} unless an exception occurs
during its evaluation. Then the return value will be a tuple
containing information about the exception.

\begin{erlang}
{'EXIT', {Reason, Stack}}
\end{erlang}

Then the exception is \textit{caught}. Otherwise it would terminate
the process. If the exception is caused by a function call
\texttt{exit(Term)} the tuple \texttt{\{'EXIT',Term\}} is returned. If
the exception is caused by calling \texttt{throw(Term)} then
\texttt{Term} will be returned.

\texttt{catch 1+2} $\Rightarrow$ \texttt{3}\\
\texttt{catch 1+a } $\Rightarrow$ \texttt{\{'EXIT',\{badarith,[...]\}\}}

\texttt{catch} has low precedence and catch subexpressions often need
to be enclosed in a block expression or in parentheses.

\texttt{A = (catch 1+2)} $\Rightarrow$ \texttt{3}

The BIF \texttt{throw(Expr)} is used for \textit{non-local} return
from a function. It must be evaluated within a catch, which returns
the result from evaluating \texttt{Expr}.

\texttt{catch begin 1,2,3,throw(four),5,6 end}  $\Rightarrow$  \texttt{four}

If \texttt{throw/1} is not evaluated within a catch, a
\texttt{nocatch} run-time error will occur.

A catch will not prevent a process from terminating due to an exit
signal from another linked process (unless it has been set to trap
exits).


\section{Try}
\label{errorhandling:try}
The try expression is able to distinguish between different exception
classes. The following example emulates \texttt{catch Expr}:

\begin{erlang}
try Expr
catch
    throw:Term -> Term;
    exit:Reason -> {'EXIT', Reason};
    error:Reason -> {'EXIT',{Reason, erlang:get_stacktrace()}}
end
\end{erlang}

The full description of \texttt{try} is as follows:

\begin{erlang}
try Expr [of
    Pattern1 [when GuardSeq1] -> Body1;
    ...;
    PatternN [when GuardSeqN] -> BodyN]
[catch
    [Class1:]ExceptionPattern1 [when ExceptionGuardSeq1] -> ExceptionBody1;
    ...;
    [ClassN:]ExceptionPatternN [when ExceptionGuardSeqN] -> ExceptionBodyN]
[after AfterBody]
end
\end{erlang}

There has to be at least a \texttt{catch} or an \texttt{after}
clause. There may be an \texttt{of} clause following the \texttt{Expr}
which adds a \texttt{case} expression on the value of \texttt{Expr}.

\texttt{try} returns the value of \texttt{Expr} unless an exception
occurs during its evaluation. Then the exception is \textit{caught}
and the patterns \texttt{ExceptionPattern} with the right exception
Class are sequentially matched against the caught exception. An
omitted Class is shorthand for throw. If a match succeeds and the
optional guard sequence \texttt{ExceptionGuardSeq} is true, the
corresponding \texttt{ExceptionBody} is evaluated to become the return
value.

If there is no matching \texttt{ExceptionPattern} of the right Class
with a true guard sequence, the exception is passed on as if
\texttt{Expr} had not been enclosed in a try expression. An exception
occurring during the evaluation of an \texttt{ExceptionBody} it is not
caught.

If none of the of Patterns match, a \texttt{try\_clause} run-time
error will occur.

If defined then \texttt{AfterBody} is always evaluated \textbf{last}
irrespective of whether and error occurred or not. Its return value is
ignored and the return value of the \texttt{try} is the same as
without an \texttt{after} section. \texttt{AfterBody} is evaluated
even if an exception occurs in \texttt{Body} or
\texttt{ExceptionBody}, in which case the exception is passed on.

%% The \texttt{AfterBody} is evaluated after either \texttt{Body} or
%% \texttt{ExceptionBody} no matter which one. The evaluated value of the
%% \texttt{AfterBody} is lost; the return value of the try expression is
%% the same with an after section as without. Even if an exception occurs
%% during evaluation of \texttt{Body} or \texttt{ExceptionBody}, the
%% \texttt{AfterBody} is evaluated. In this case the exception is caught
%% and passed on after the \texttt{AfterBody} has been evaluated, so the
%% exception from the try expression is the same with an after section as
%% without.

An exception that occurs during the evaluation of \texttt{AfterBody}
itself is not caught, so if the \texttt{AfterBody} is evaluated due to
an exception in \texttt{Expr}, \texttt{Body} or
\texttt{ExceptionBody}, that exception is lost and masked by the new
exception.


\chapter{Distributed Erlang}
\label{distribution}

A \textbf{distributed Erlang system} consists of a number of Erlang
runtime systems communicating with each other. Each such runtime
system is called a \textbf{node}.  Nodes can reside on the same
host or on different hosts connected through a network.  The
standard distribution mechanism is implemented using TCP/IP sockets
but other mechanisms can also be implemented.

Message passing between processes on different nodes, as well as links
and monitors, is transparent when using \texttt{Pid}s.  However, registered
names are local to each node.  A registered process at a particular
node is referred to as \texttt{\{Name,Node\}}.

The Erlang Port Mapper Daemon \textbf{epmd} is automatically started
on every host where an Erlang node is started.  It is responsible for
mapping the symbolic node names to machine addresses.


\section{Nodes}
A \textbf{node} is an executing Erlang runtime system which has been
given a name, using the command line flag \texttt{-name} (long name)
or \texttt{-sname} (short name).

The format of the node name is an atom \texttt{Name@Host} where
\texttt{Name} is the name given by the user and \texttt{Host} is the
full host name if long names are used, or the first part of the host
name if short names are used. \texttt{node()} returns the name of the
node. Nodes using long names cannot communicate with nodes using
short names.


\section{Node connections}
The nodes in a distributed Erlang system are fully connected. The
first time the name of another node is used, a connection attempt to
that node will be made. If a node A connects to node B, and node B has
a connection to node C, then node A will also try to connect to node
C. This feature can be turned off using the command line flag:

\texttt{~~~~-connect\_all false}

If a node goes down, all connections to that node are removed. The
BIF:

\begin{erlang}
erlang:disconnect(Node)
\end{erlang}

disconnects \texttt{Node}. The BIF \texttt{nodes()} returns the list of
currently connected (visible) nodes.


\section{Hidden nodes}
It is sometimes useful to connect to a node without also connecting to
all other nodes. For this purpose, a \textbf{hidden node} may be
used. A hidden node is a node started with the command line flag
\texttt{-hidden}.  Connections between hidden nodes and other nodes must be set
up explicitly. Hidden nodes do not show up in the list of nodes
returned by \texttt{nodes()}. Instead, \texttt{nodes(hidden)} or \texttt{nodes(connected)} must
be used. A hidden node will not be included in the set of nodes that
the module global keeps track of.

A \textbf{C node} is a C program written to act as a hidden node in a
distributed Erlang system. The library \texttt{erl\_interface}
contains functions for this purpose.


\section{Cookies}
Each node has its own \textbf{magic cookie}, which is an atom. The
Erlang network authentication server (auth) reads the cookie in the
file \texttt{\$HOME/.erlang.cookie}. If the file does not exist, it
will be created with a random string as content.

% FRMB CHECK: the implication here is that "erlang:set_cookie(node(), Cookie)"
% sets *this* node's cookie, as well as the cookie that will be used to connect
% to other nodes, if not explicitly set otherwise.

The permissions of the file must be set to octal 400 (read-only
by user).  The cookie of the local node may also be set using the BIF
\texttt{erlang:set\_cookie(node(), Cookie)}.

The current node is only allowed to communicate with another node
\texttt{Node2} if it knows its cookie.  If this is different from the current node (whose cookie will
be used by default) it must be explicitly set with the BIF \texttt{erlang:set\_cookie(Node2, Cookie2)}.


\section{Distribution BIFs}

\begin{center}
\begin{tabular}{|>{\raggedright}p{156pt}|>{\raggedright}p{170pt}|}
\hline
\multicolumn{2}{|p{326pt}|}{Distribution BIFs}\tabularnewline
\hline
\texttt{node()}  & Returns the name of the current node. Allowed in guards\tabularnewline
\hline
\texttt{is\_alive()}  & Returns true if the runtime system is a node and can connect to
other nodes, false otherwise\tabularnewline
\hline
\texttt{erlang:get\_cookie()}  & Returns the magic cookie of the current node\tabularnewline
\hline
\texttt{set\_cookie(Node, Cookie)} & Sets the magic cookie used when connecting to \texttt{Node}.
If \texttt{Node} is the current node, \texttt{Cookie} will be used when connecting to all new nodes\tabularnewline
\hline
\texttt{nodes()}  & Returns a list of all visible nodes to which the current node is connected
to\tabularnewline
\hline
\texttt{nodes(connected\textbar{}hidden)}  & Returns a list not only of visible nodes,
but also hidden nodes and previously known nodes, etc. \tabularnewline
\hline
\texttt{monitor\_node(Node,}\break\texttt{\phantom{xxxx}true\textbar{}false)}  & Monitors the status of \texttt{Node}. A message
\texttt{\{nodedown, Node\}} is received if the connection to it is lost\tabularnewline
\hline
\texttt{node(Pid\textbar{}Ref\textbar{}Port)}  & Returns the node where the argument is
located\tabularnewline
\hline
\texttt{erlang:disconnect\_node(Node)}  & Forces the disconnection of \texttt{Node}\tabularnewline
\hline
\texttt{spawn[\_link\textbar{}\_opt](Node,}\break\texttt{\phantom{xxxx}Module, Function, Args)}  & Creates a process
at a remote node\tabularnewline
\hline
\texttt{spawn[\_link\textbar{}\_opt](Node, Fun)}  & Creates a process at a remote node\tabularnewline
\hline
\end{tabular}
\end{center}


\section{Distribution command line flags}

\begin{center}
\begin{tabular}{|>{\raggedright}p{102pt}|>{\raggedright}p{224pt}|}
\hline
\multicolumn{2}{|p{326pt}|}{Distribution command line flags}\tabularnewline
\hline
\texttt{-connect\_all false}  & Only explicit connection set-ups will be used\tabularnewline
\hline
\texttt{-hidden}  & Makes a node into a hidden node\tabularnewline
\hline
\texttt{-name Name}  & Makes a runtime system into a node, using long node names\tabularnewline
\hline
\texttt{-setcookie Cookie}  & Same as calling \linebreak{}
\texttt{erlang:set\_cookie(node(), Cookie))}\tabularnewline
\hline
\texttt{-sname Name}  & Makes a runtime system into a node, using short node names\tabularnewline
\hline
\end{tabular}
\end{center}


\section{Distribution modules}
There are several modules available which are useful for distributed programming:

\begin{center}
\begin{tabular}{|>{\raggedright}p{93pt}|>{\raggedright}p{233pt}|}
\hline
\multicolumn{2}{|p{326pt}|}{Kernel modules useful for distribution}\tabularnewline
\hline
\texttt{global}  & A global name registration facility\tabularnewline
\hline
\texttt{global\_group}  & Grouping nodes to global name registration groups\tabularnewline
\hline
\texttt{net\_adm}  & Various net administration routines\tabularnewline
\hline
\texttt{net\_kernel}  & Erlang networking kernel\tabularnewline
\hline
\multicolumn{2}{|p{326pt}|}{STDLIB modules useful for distribution}\tabularnewline
\hline
\texttt{slave}  & Start and control of slave nodes\tabularnewline
\hline
\end{tabular}
\end{center}


\chapter{Ports and Port Drivers}
\label{ports}
\textbf{Ports} provide a byte-oriented interface to external programs
and communicate with Erlang processes by sending and receiving lists
of bytes as messages. The Erlang process that creates a port is called
the \textbf{port owner} or the \textbf{connected process} of the
port. All communication to and from the port should go via the port
owner. If the port owner terminates, so will the port (and the
external program, if it has been programmed correctly).

The external program forms another OS process. By default, it should
read from standard input (file descriptor 0) and write to standard
output (file descriptor 1). The external program should terminate when
the port is closed.


\section{Port Drivers}
Drivers are normally programmed in C and are dynamically linked to the
Erlang runtime system. The linked-in driver behaves like a port and is
called a \textbf{port driver}. However, an erroneous port driver might
cause the entire Erlang runtime system to leak memory, hang or crash.


\section{Port BIFs}

\begin{center}
\begin{tabular}{|>{\raggedright}p{161pt}|>{\raggedright}p{165pt}|}
\hline
\multicolumn{2}{|p{326pt}|}{Port creation BIF}\tabularnewline
\hline
\texttt{open\_port(PortName, PortSettings)} & Returns a \textbf{port identifier} \texttt{Port} as the result of opening a new Erlang port. Messages can be sent to and received from a port identifier, just like a Pid. Port identifiers can also be linked to or registered under a name using \texttt{link/1} and \texttt{register/2}. \tabularnewline
\hline
\end{tabular}
\end{center}

\texttt{PortName} is usually a tuple \texttt{\{spawn,Command\}} where
the string \texttt{Command} is the name of the external program. The
external program runs outside the Erlang workspace unless a port
driver with the name \texttt{Command} is found. If the driver is
found, it will be started.

\texttt{PortSettings} is a list of settings (options) for the
port. The list typically contains at least a tuple
\texttt{\{packet,N\}} which specifies that data sent between the port
and the external program are preceded by an N-byte length
indicator. Valid values for \texttt{N} are 1, 2 or 4. If binaries
should be used instead of lists of bytes, the option \texttt{binary}
must be included.

The port owner Pid communicates with Port by sending and receiving
messages. (Any process could send the messages to the port, but
messages from the port will always be sent to the port owner).

\begin{center}
\begin{tabular}{|>{\raggedright}p{115pt}|>{\raggedright}p{211pt}|}
\hline
\multicolumn{2}{|p{326pt}|}{Messages sent to a port}\tabularnewline
\hline
\texttt{\{Pid, \{command, Data\}\}}  & Sends Data to the port. \tabularnewline
\hline
\texttt{\{Pid, close\}}  & Closes the port. Unless the port is already closed, the port replies
with \texttt{\{Port, closed\}} when all buffers have been flushed and the port really closes.
\tabularnewline
\hline
\texttt{\{Pid,\{connect,NewPid\}\}}  & Sets the port owner of \texttt{Port} to \texttt{NewPid}. Unless the port is already closed, the port replies with \texttt{\{Port, connected\}} to the old port owner. Note that the old port owner is still linked to the port, but the new port
owner is not. \tabularnewline
\hline
\end{tabular}
\end{center}

Data must be an I/O list. An I/O list is a binary or a (possibly deep)
list of binaries or integers in the range 0..255.

\begin{center}
\begin{tabular}{|>{\raggedright}p{121pt}|>{\raggedright}p{204pt}|}
\hline
\multicolumn{2}{|p{326pt}|}{Messages received from a port}\tabularnewline
\hline
\texttt{\{Port, \{data, Data\}\}}  & Data is received from the external program\tabularnewline
\hline
\texttt{\{Port, closed\}}  & Reply to \texttt{Port ! \{Pid,close\}}\tabularnewline
\hline
\texttt{\{Port, connected\}}  & Reply to \texttt{Port ! \{Pid,\{connect, NewPid\}\}} \tabularnewline
\hline
\texttt{\{'EXIT', Port, Reason\}}  & If Port has terminated for some reason. \tabularnewline
\hline
\end{tabular}
\end{center}

Instead of sending and receiving messages, there are also a number of
BIFs that can be used. These can be called by any process, not only
the port owner.

\begin{center}
\begin{tabular}{|>{\raggedright}p{146pt}|>{\raggedright}p{180pt}|}
\hline
\multicolumn{2}{|p{326pt}|}{Port BIFs}\tabularnewline
\hline
\texttt{port\_command(Port, Data)}  & Sends Data to Port\tabularnewline
\hline
\texttt{port\_close(Port)}  & Closes Port\tabularnewline
\hline
\texttt{port\_connect(Port, NewPid)}  & Sets the port owner of \texttt{Port} to \texttt{NewPid}. The old port owner Pid stays linked to the port and has to call \texttt{unlink(Port)} if this is not desired. \tabularnewline
\hline
\texttt{erlang:port\_info(Port, Item)}  & Returns information as specified by \texttt{Item}\tabularnewline
\hline
\texttt{erlang:ports()}  & Returns a list of all ports on the current node\tabularnewline
\hline
\end{tabular}
\end{center}

There are some additional BIFs that only apply to port drivers:
\texttt{port\_control/3} and \texttt{erlang:port\_call/3}.

\chapter{Code loading}
\label{code}

Erlang supports code updating in a running system. Code replacement is
performed at module level.

The code of a module can exist in two versions in a system:
\textbf{current} and \textbf{old}. When a module is
loaded into the system for the first time, the code becomes
\textit{current}. If a new instance of the module is loaded, the code
of the previous instance becomes \textit{old} and the new instance
becomes \textit{current}. Normally a module is automatically loaded
the first time a function in it is called. If the module is already
loaded then it must explicitly be loaded again to a new version.

Both old and current code are valid, and may be used
concurrently. Fully qualified function calls will always refer to the
current code. However, the old code may still be run by other
processes.

If a third instance of the module is loaded, the code server will
remove (\textit{purge}) the old code and any processes lingering in it
are terminated. Then the third instance becomes \textit{current} and
the previously current code becomes \textit{old}.

To change from old code to current code, a process must make a fully
qualified function call.

\begin{erlang}
-module(mod).
-export([loop/0]).

loop() ->
    receive
        code_switch ->
            mod:loop();
        Msg ->
            ...
            loop()
    end.
\end{erlang}

To make the process change code, send the message
\texttt{code\_switch} to it. The process then will make a fully
qualified call to \texttt{mod:loop()} and change to the current
code. Note that \texttt{mod:loop/0} must be exported.

\chapter{Macros}
\label{macros}

\section{Defining and using macros}

\begin{erlang}
-define(Const, Replacement).
-define(Func(Var1, ..., VarN), Replacement).
\end{erlang}

A \textbf{macro} must be defined before it is used but a macro
definition may be placed anywhere among the attributes and function
declarations of a module. If a macro is used in several modules it is
advisable to put the macro definition in an include file. A macro is
used as follows:

\begin{erlang}
?Const
?Func(Arg1,...,ArgN)
\end{erlang}

Macros are expanded during compilation. A macro reference
\texttt{?Const} is replaced by \texttt{Replacement} like this:

\begin{erlang}
-define(TIMEOUT, 200).
...
call(Request) ->
    server:call(refserver, Request, ?TIMEOUT).
\end{erlang}

is expanded to:

\begin{erlang}
call(Request) ->
    server:call(refserver, Request, 200).
\end{erlang}

A macro reference \texttt{?Func(Arg1, ..., ArgN)} will be replaced by
\texttt{Replacement}, where all occurrences of a variable \texttt{VarX}
from the macro definition are replaced by the corresponding argument
\texttt{ArgX}.

\begin{erlang}
-define(MACRO1(X, Y), {a, X, b, Y}).
...
bar(X) ->
    ?MACRO1(a, b),
    ?MACRO1(X, 123).
\end{erlang}

will be expanded to:

\begin{erlang}
bar(X) ->
    {a, a, b, b},
    {a, X, b, 123}.
\end{erlang}

To view the result of macro expansion, a module can be compiled with
the \texttt{`P'} option:

\begin{erlang}
compile:file(File, ['P']).
\end{erlang}

This produces a listing of the parsed code after preprocessing and
parse transforms, in the file \texttt{File.P}.


\section{Predefined macros}

\begin{center}
\begin{tabular}{|>{\raggedright}p{103pt}|>{\raggedright}p{223pt}|}
\hline
\multicolumn{2}{|p{326pt}|}{P{\large{}redefined} macros}\tabularnewline
\hline
\texttt{?MODULE} & The name of the current module\tabularnewline
\hline
\texttt{?MODULE\_STRING} & The name of the current module, as a string\tabularnewline
\hline
\texttt{?FILE} & The file name of the current module\tabularnewline
\hline
\texttt{?LINE} & The current line number\tabularnewline
\hline
\texttt{?MACHINE} & The machine name, 'BEAM'\tabularnewline
\hline
\end{tabular}
\end{center}


\section{Flow Control in Macros}

\begin{erlang}
-undef(Macro).      % This inhibits the macro definition.

-ifdef(Macro).
    %% Lines that are evaluated if Macro was defined
-else.
    %% If the condition was false, these lines are evaluated instead.
-endif.
\end{erlang}

\texttt{ifndef(Macro)} can be used instead of \texttt{ifdef} and means
the opposite.

\begin{erlang}
-ifdef(debug).
-define(LOG(X), io:format("{~p,~p}:~p~n",[?MODULE,?LINE,X])).
-else.
-define(LOG(X), true).
-endif.
\end{erlang}

If \texttt{debug} is defined when the module is compiled,
\texttt{?LOG(Arg)} will expand to a call to \texttt{io:format/2} and
provide the user with some simple trace output.


\section{Stringifying Macro Arguments}
\texttt{??Arg}, where \texttt{Arg} is a macro argument expands to the
argument in the form of a string.

\begin{erlang}
-define(TESTCALL(Call), io:format("Call ~s: ~w~n", [??Call, Call])).

?TESTCALL(myfunction(1,2)),
?TESTCALL(you:function(2,1)),
\end{erlang}

results in:

\begin{erlang}
io:format("Call ~s: ~w~n", ["myfunction(1,2)", m:myfunction(1,2)]),
io:format("Call ~s: ~w~n", ["you:function(2,1)", you:function(2,1)]),
\end{erlang}

That is, a trace output with both the function called and the
resulting value.


\chapter{Further Reading and Resources}

Following websites provide in-depth explanation of topics and concepts briefly covered in this document:

\begin{itemize}
	\item Official Erlang documentation: \url{http://www.erlang.org/doc/}
	\item Learn You Some Erlang for Great Good: \url{http://learnyousomeerlang.com/}
	\item Tutorials section at Erlang Central: \url{https://erlangcentral.org/wiki/index.php?title=Category:HowTo}
\end{itemize}

Still have questions? erlang-questions mailing list (\url{http://erlang.org/mailman/listinfo/erlang-questions}) is a good place for general discussions about Erlang/OTP, the language, implementation, usage and beginners questions.



\end{document}
