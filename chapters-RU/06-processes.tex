\chapter{Processes}
\label{processes}

A \textbf{process} corresponds to one \textit{thread of control}.
Erlang permits very large numbers of concurrent processes,
each executing like it had an own virtual processor.  When a process
executing \texttt{functionA} calls another \texttt{functionB}, it
will wait until \texttt{functionB} is finished and then retrieve its
result. If instead it \textit{spawns} another process executing
\texttt{functionB}, both will continue in parallel
(concurrently). \texttt{functionA} will not wait for \texttt{functionB}
and the only way they can communicate is through \textit{message passing}.

Erlang processes are light-weight with a small memory footprint, fast to
create and shut-down, and the scheduling overhead is low.  A
\textbf{process identifier}, \texttt{Pid}, identifies a process. The
BIF \texttt{self/0} returns the \texttt{Pid} of the calling process.


\section{Process creation}
A process is created using the BIF \texttt{spawn/3}.

\begin{erlang}
spawn(Module, Func, [Expr1, ..., ExprN])
\end{erlang}

\texttt{Module} should evaluate to a module name and \texttt{Func} to
a function name in that module. The list \texttt{Expr1}$...$\texttt{ExprN} are the
arguments to the function.  \texttt{spawn} creates a new process and
returns the process identifier, \texttt{Pid}. The new process starts
by executing:

\begin{erlang}
Module:Func(Expr1, ..., ExprN)
\end{erlang}

The function \texttt{Func} has to be exported even if it is spawned by
another function in the same module. There are other spawn BIFs, for
example \texttt{spawn/4} for spawning a process on another node.


\section{Registered processes}
A process can be associated with a name. The name must be an atom and
is automatically unregistered if the process terminates. Only static
(cyclic) processes should be registered.

\begin{center}
\begin{tabular}{|>{\raggedright}p{117pt}|>{\raggedright}p{204pt}|}
\hline
\multicolumn{2}{|p{321pt}|}{Name registration BIFs}\tabularnewline
\hline
\texttt{register(Name, Pid)}  & Associates the atom \texttt{Name} with the process \texttt{Pid}\tabularnewline
\hline
\texttt{registered()}  & Returns a list of names which have been registered \tabularnewline
\hline
\texttt{whereis(Name)}  & Returns the \texttt{Pid} registered under \texttt{Name} or \texttt{undefined} if the name
is not registered\tabularnewline
\hline
\end{tabular}
\end{center}


\section{Process communication}
Processes communicate by sending and receiving
\textbf{messages}. Messages are sent using the send operator
(\texttt{!}) and are received using \texttt{receive}. Message passing
is asynchronous and reliable, i.e.~the message is guaranteed to
eventually reach the recipient, provided that the recipient exists.

\subsection{Send}
\begin{erlang}
Pid ! Expr
\end{erlang}

The send (\texttt{!}) operator sends the value of \texttt{Expr} as a
message to the process specified by \texttt{Pid} where it will be
placed last in its \textbf{message queue}.  The value of \texttt{Expr}
is also the return value of the (\texttt{!}) expression. \texttt{Pid}
must evaluate to a process identifier, a registered name or a tuple
\texttt{\{Name,Node\}}, where \texttt{Name} is a registered process at
\texttt{Node} (see chapter \ref{distribution}). The message sending operator
(\texttt{!}) never fails, even if it addresses a non-existent process.


\subsection{Receive}

\begin{erlang}
receive
    Pattern1 [when GuardSeq1] ->
        Body1;
    ...
    PatternN [when GuardSeqN] ->
        BodyN                   % Note no semicolon (;) before end
end
\end{erlang}

This expression receives messages sent to the process using the send
operator (\texttt{!}). The patterns \texttt{PatternX} are sequentially
matched against the first message in time order in the message queue,
then the second and so on.  If a match succeeds and the optional guard
sequence \texttt{GuardSeqX} is true, then the message is removed from
the message queue and the corresponding \texttt{BodyX} is
evaluated.  It is the order of the pattern clauses that decides the
order in which messages will be received prior to the order in which
they arrive.  This is called \textit{selective receive}. The
return value of \texttt{BodyX} is the return value of the receive
expression.

\texttt{receive} never fails. The process may be suspended, possibly
indefinitely, until a message arrives that matches one of the patterns
and with a true guard sequence.

\newpage
\begin{erlang}
wait_for_onhook() ->
    receive
        onhook ->
            disconnect(),
            idle();
        {connect, B} ->
            B ! {busy, self()},
            wait_for_onhook()
    end.
\end{erlang}


\subsection{Receive with timeout}

\begin{erlang}
receive
    Pattern1 [when GuardSeq1] ->
        Body1;
        ...;
    PatternN [when GuardSeqN] ->
        BodyN
after
    ExprT ->
        BodyT
end
\end{erlang}

\texttt{ExprT} should evaluate to an integer between \texttt{0} and
\texttt{16\#ffffffff} (the value must fit in 32 bits). If no matching
message has arrived within \texttt{ExprT} milliseconds, then
\texttt{BodyT} will be evaluated and its return value becomes the
return value of the receive expression.

\begin{erlang}
wait_for_onhook() ->
    receive
        onhook ->
            disconnect(),
            idle();
        {connect, B} ->
            B ! {busy, self()},
            wait_for_onhook()
    after
        60000 ->
            disconnect(),
            error()
    end.
\end{erlang}

A \texttt{receive...after} expression with no branches can be used to
implement simple timeouts.

\begin{erlang}
receive
after
    ExprT ->
        BodyT
end
\end{erlang}

\begin{center}
\begin{tabular}{|>{\raggedright}p{47pt}|>{\raggedright}p{273pt}|}
\hline
\multicolumn{2}{|p{321pt}|}{Two special cases for the timeout value \texttt{ExprT}}\tabularnewline
\hline
\texttt{infinity} & This is equivalent to not using a timeout and can be useful for timeout
values that are calculated at run-time\tabularnewline
\hline
\texttt{0} & If there is no matching message in the mailbox, the timeout will occur immediately\tabularnewline
\hline
\end{tabular}
\end{center}


\section{Process termination}
\label{processes:termination}
A process always terminates with an \textbf{exit reason} which may be
any term.  If a process terminates normally, i.e.~it has run
to the end of its code, then the reason is the atom \texttt{normal}. A process
can terminate itself by calling one of the following BIFs.

\begin{erlang}
exit(Reason)

erlang:error(Reason)

erlang:error(Reason, Args)
\end{erlang}

A process terminates with the exit reason \texttt{\{Reason,Stack\}} when a
run-time error occurs.

A process may also be terminated if it receives an exit signal with
a reason other than \texttt{normal} (see section \ref{processes:recvexitsignals}).


\section{Process links}
\label{processes:links}
Two processes can be \textbf{linked} to each other. Links are
bidirectional and there can only be one link between two distinct processes (unique \texttt{Pid}s). A
process with \texttt{Pid1} can link to a process with \texttt{Pid2}
using the BIF \texttt{link(Pid2)}.  The BIF \texttt{spawn\_link(Module, Func, Args)}
spawns and links a process in one atomic operation.

A link can be removed using the BIF \texttt{unlink(Pid)}.


\subsection{Error handling between processes}
When a process terminates it will send \textbf{exit signals} to all
processes that it is linked to.  These in turn will also be terminated
\textit{or handle the exit signal in some way}. This feature can be
used to build hierarchical program structures where some processes are
supervising other processes, for example restarting them if they
terminate abnormally.


\subsection{Sending exit signals}
\label{processes:sendexitsignals}
A process always terminates with an exit reason which is sent as an
exit signal to all linked processes. The BIF \texttt{exit(Pid, Reason)} sends
an exit signal with the reason \texttt{Reason} to \texttt{Pid}, without
affecting the calling process.


\subsection{Receiving exit signals}
\label{processes:recvexitsignals}
If a process receives an exit signal with an exit reason other than
\texttt{normal} it will also be terminated, and will send exit signals with the
same exit reason to its linked processes.  An exit signal with reason
\texttt{normal} is ignored.  This behaviour can be changed using the BIF
\texttt{process\_flag(trap\_exit, true)}.

The process is then able to \textbf{trap exits}.  This means that an
exit signal will be transformed into a message \texttt{\{'EXIT', FromPid, Reason\}} which is
put into the process's mailbox and can be handled by the process like a regular
message using \texttt{receive}.

However, a call to the BIF \texttt{exit(Pid, kill)} unconditionally
terminates the process \texttt{Pid} regardless whether it is able to
trap exit signals or not.


\section{Monitors}
A process \texttt{Pid1} can create a \textbf{monitor} for
\texttt{Pid2} using the BIF:

\begin{erlang}
erlang:monitor(process, Pid2)
\end{erlang}

which returns a reference \texttt{Ref}. If \texttt{Pid2} terminates
with exit reason \texttt{Reason}, a message as follows will be sent to
\texttt{Pid1}:

\begin{erlang}
{'DOWN', Ref, process, Pid2, Reason}
\end{erlang}

If \texttt{Pid2} does not exist, the \texttt{'DOWN'} message is sent
immediately with \texttt{Reason} set to \texttt{noproc}.  Monitors are
unidirectional in that if \texttt{Pid1} monitors \texttt{Pid2} then it
will receive a message when \texttt{Pid2} dies but \texttt{Pid2} will
\textbf{not} receive a message when \texttt{Pid1} dies. Repeated calls
to \texttt{erlang:monitor(process, Pid)} will create several,
independent monitors and each one will be sent a \texttt{'DOWN'} message
when \texttt{Pid} terminates.

A monitor can be removed by calling \texttt{erlang:demonitor(Ref)}. It
is possible to create monitors for processes with registered names,
also at other nodes.


\section{Process priorities}
The BIF \texttt{process\_flag(priority, Prio)} defines the priority of
the current process. \texttt{Prio} may have the value \texttt{normal},
which is the default, \texttt{low}, \texttt{high} or \texttt{max}. 

Modifying a process's priority is discouraged and should only be done in 
special circumstances.  A problem that requires changing process priorities
can generally be solved by another approach.


\section{Process dictionary}
\label{processes:dicts}
Each process has its own process dictionary which is a list of
\texttt{\{Key, Value\}} terms.

\begin{center}
\begin{tabular}{|>{\raggedright}p{79pt}|>{\raggedright}p{247pt}|}
\hline
\multicolumn{2}{|p{326pt}|}{Process dictionary BIFs}\tabularnewline
\hline
\texttt{put(Key, Value)} & Saves the \texttt{Value} under the \texttt{Key} or replaces an older value\tabularnewline
\hline
\texttt{get(Key)} & Retrieves the value stored under \texttt{Key} or \texttt{undefined}\tabularnewline
\hline
\texttt{get()} & Returns the entire process dictionary as a list of \texttt{\{Key, Value\}} terms\tabularnewline
\hline
\texttt{get\_keys(Value)} & Returns a list of keys that have the value \texttt{Value}\tabularnewline
\hline
\texttt{erase(Key)} & Deletes \texttt{\{Key, Value\}}, if any, and returns \texttt{Key}\tabularnewline
\hline
\texttt{erase()} & Returns the entire process dictionary and deletes it\tabularnewline
\hline
\end{tabular}
\end{center}

Process dictionaries could be used to keep global variables within an application,
but the extensive use of them for this is usually regarded as poor programming style.

