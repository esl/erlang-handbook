\chapter{Дальнейшие материалы для чтения}

Следующие вебсайты предлагают более подробное объяснение тем и концепций, кратко
описанных в данном документе:

\section{Русскоязычные ресурсы}

\begin{itemize}
\item Русские новости из мира Erlang \url{http://erlanger.ru}
\item Группа \texttt{erlang-russian} и \texttt{erlang-in-ukraine} на сервере 
Google Groups.
\item Книга Ф. Чезарини "<Программирование в Erlang">, русский перевод
\url{http://dmkpress.com/catalog/computer/programming/functional/978-5-94074-617-1/}
\end{itemize}

Также обратите внимание на русскоязычный канал \texttt{erlang} на сервере 
\url{http://jabber.ru}

\section{Англоязычные ресурсы}

\begin{itemize}
\item Официальная документация по Erlang: \url{http://www.erlang.org/doc/}
\item Learn You Some Erlang for Great Good: \url{http://learnyousomeerlang.com/}
\item Раздел с уроками на Erlang Central: 
	\url{https://erlangcentral.org/wiki/index.php?title=Category:HowTo}
\end{itemize}

Ещё вопросы? Список рассылки \texttt{erlang-questions} (адрес для подписки
\url{http://erlang.org/mailman/listinfo/erlang-questions}) является хорошим местом
для неспешных общих дискуссий об Erlang/OTP, языке, реализации, использовании и
вопросы новичков. Если вы не планируете писать в рассылку, её можно прочесть без 
подписки на сервере Google Groups, группа \texttt{erlang-programming}.

Также обратите внимание на англоязычный IRC канал \texttt{\#erlang} в сети 
Freenode.
